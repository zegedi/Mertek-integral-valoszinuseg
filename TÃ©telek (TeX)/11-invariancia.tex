\documentclass[
%parspace, % Add vertical space between paragraphs
%noindent, % No indentation of first lines in each paragraph
%nohyp,	   % No hyphenation of words
%twoside,  % Double sided format
%draft,    % Quicker draft compilation without rendering images
%final,    % Set final to hide todos
]{elteikthesis}[2024/04/26]


% The minted package is also supported for source highlighting
% See elteikthesis_minted.tex for example
%\usepackage[newfloat]{minted}
\usepackage{enumitem}

% Document's metadata
\title{Mérték, integrál, valószínűség} % title
\subtitle{11. Vizsgatétel}

% The document
\begin{document}
	
	% Set document language
	\documentlang{hungarian}
	
	\section{Invariancia}
	
	Legyen \( 1 \leq p \in \N \) egy tetszőleges kitevő.
	
	\begin{statement}{Eltolás és tükrözés invariancia}{}
		Legyen \( A \in \Omega_p \) tetszőleges Borel-halmaz, 
		valamint \( \vb{a} \in \R^p \) egy vektor. Ekkor
		\begin{enumerate}
			\item
			\emph{Eltolás invariancia:}
			\( \vb{a} + A \in \Omega_p \) 
			\quad és \quad 
			\( \mu_p( \vb{a} + A ) = \mu_p(A) \).
		\end{enumerate}
	\end{statement}
	
	\newpage
	\section{Példa nem Borel-mérhető halmazra}
	
	\begin{statement}{}{}
		Létezik olyan \( A \subseteq \R \) halmaz, ami nem Borel-mérhető, 
		azaz \( A \notin \Omega_1. \)
	\end{statement}
	\begin{proof}
		Vezessük be az alábbi relációt
		\[
			x \sim y
			\quad \ratio\Longleftrightarrow \quad
			x - y \in \Q.
		\]
		Ekkor fennállnak az alábbi állítások.
		\begin{enumerate}
			\item A \( \sim \) ekvivalenciareláció.
			\item Bármely \( A \) ekvivalenciaosztály esetén \( A \cap [0, 1) \neq \emptyset \).
		\end{enumerate}
		Ezt felhasználva legyen \( K \) az a halmaz, 
		ami minden ekvivalenciaosztályból pontosan egy elemet tartalmaz és más eleme nincsen.
		Ekkor
		\[
			\R = \bigcup_{r \in \Q} (r + K)
		\]
		és ez páronként diszjunkt felbontás.
		
		Végül indirekt tegyük fel, hogy \( K \in \Omega_1 \). Ekkor
		\[
			\mu_1( \R ) = 
			+\infty =
			\sum_{r \in \Q} \mu_1(r + K) = 
			\sum_{r \in \Q} \mu_1(K).
		\]
		Innen egyszerűen adódik, hogy \( \mu_1(K) > 0 \). Ugyanakkor
		\[
			B \coloneq \Q \cap [0, 1)
			\quad \Longrightarrow \quad
			\bigcup_{r \in B} (r + K) \subseteq [0, 2)
			\quad \Longrightarrow \quad
			\mu_1(B) \leq 2.
		\]
		Viszont
		\[
			\mu_1(B) =
			\sum_{r \in B} \mu_1(r + K) =
			\sum_{r \in B} \mu_1(K) =
			+\infty.
		\]
	\end{proof}
	
\end{document}