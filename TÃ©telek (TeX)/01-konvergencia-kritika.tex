\documentclass[
%parspace, % Add vertical space between paragraphs
%noindent, % No indentation of first lines in each paragraph
%nohyp,	   % No hyphenation of words
%twoside,  % Double sided format
%draft,    % Quicker draft compilation without rendering images
%final,    % Set final to hide todos
]{elteikthesis}[2024/04/26]


% The minted package is also supported for source highlighting
% See elteikthesis_minted.tex for example
%\usepackage[newfloat]{minted}
\usepackage{enumitem}
\usepackage{pgfplots}
\usetikzlibrary{arrows.meta}


% Document's metadata
\title{Mérték, integrál, valószínűség} % title
\subtitle{1. Vizsgatétel}

% The document
\begin{document}
	
	% Set document language
	\documentlang{hungarian}
	
	\section{Függvények lokális oszcillációja}
	
	\begin{definition}{Oszcilláció halmazon, lokális oszcilláció}{oszcilláció}
		Legyen \( \funcin{f}{\R}{\R} \), és
		\( A \subseteq \R \) olyan halmaz, hogy \( A \cap \dom{f} \neq \emptyset \).
		\marginnote{
			Legyen \( \func{f}{[a, b]}{\R} \) korlátos függvény,
			\[
				\tau \coloneq \{ a = x_0 < \cdots < x_n = b \}
			\]
			egy felosztás. Ekkor
			\[
				\omega(f, \tau) \coloneq 
				\sum_{ I \in \mathcal{F}(\tau) } \mathcal{O}(f, I) {\cdot} \abs{ I }
			\]
			az \( f \) függvény \emph{oszcillációs összege}.
		}
		Ekkor
		\[
			\mathcal{O}(f, A) \coloneq
			\sup \Bigl\{ \abs\big{f(x) - f(y)} \ \colon\, x, y \in A \cap \dom{f} \Bigr\}
		\]
		az \( f \) függvény \emph{oszcillációja} az \( A \) halmazon.
		Továbbá egy \( z \in \dom{f} \) helyen
		\[
			o_z(f) \coloneq
			\inf \Bigl\{ 
			\mathcal{O}(f, I) \ \colon\, I \subset \R \text{ intervallum},\, z \in \inter(I) 
			\Bigr\}
		\]
		az \( f \) függvény \emph{lokális oszcillációja} a \( z \) pontban.
	\end{definition}
	
	\begin{lemma}{Lokális oszcilláció és a folytonosság kapcsolata}{}
		Legyen \( \funcin{f}{\R}{\R} \), valamint \( z \in \dom{f} \) egy adott pont.
		Ekkor
		\[
			f \in \ContAt{z}
			\quad \iff \quad
			o_z(f) = 0.
		\]
	\end{lemma}
	\begin{proof}\,\\[3pt]
		\Ifstep Ha az \( f \) függvény folytonos \( z \)-ben, 
		akkor tetszőleges \( \varepsilon > 0 \)-hoz
		\[
			\exists \delta > 0 \colon \quad
			\abs\big{f(x) - f(z)} < \varepsilon
			\quad \bigl( x \in \dom{f},\  \abs{x - z} < \delta \bigr).
		\]
		Legyen \( I \coloneq (z - \delta, z + \delta) \).
		Ekkor minden \( x, t \in I \cap \dom{f} \) esetén igaz, hogy
		\[
			\abs\big{f(x) - f(t)} \leq
			\abs\big{f(x) - f(z)} + \abs\big{f(t) - f(z)} <
			2 \varepsilon
			\quad \Longrightarrow \quad
			\smash{\uwave{\mathcal{O}(f, I) < 2 \varepsilon.}}
		\]
		Ebből következik, hogy \( 0 \leq o_z(f) < 2\varepsilon \), ahonnan \( o_z(f) = 0 \) adódik.
		
		% ----------------------------------------------------------------------------
		
		\vspace{6pt}
		\hrule
		\vspace{6pt}
		
		% ----------------------------------------------------------------------------
		
		\Backifstep Most tegyük fel, hogy \( o_z(f) = 0 \), vagyis definíció szerint
		\[
			o_z(f) =
			\inf \Bigl\{ 
			\mathcal{O}(f, I) \ \colon\, I \subset \R \text{ intervallum},\, z \in \inter(I) 
			\Bigr\} = 0.
		\]
		Ekkor bármilyen \( \varepsilon > 0 \)-hoz megadható olyan \( I \subset \R \) intervallum, hogy
		\[
			z \in \inter(I) 
			\qquad \text{és} \qquad
			\mathcal{O}(f, I) < \varepsilon.
		\]
		Mivel \( z \) belső pontja az \( I \)-nek, ezért létezik olyan \( \delta > 0 \) sugár, amivel
		\[
			K_\delta(z) \coloneq (z - \delta, z + \delta) \subset I.
		\]
		Ekkor bármely \( x \in \dom{f} \cap K_\delta(z) \) pontban
		\[
			\abs\big{f(x) - f(z)} \leq \mathcal{O}(f, I) < \varepsilon
			\qquad \Longrightarrow \qquad
			\smash{\uwave{f \in \ContAt{z}.}}
		\]
	\end{proof}
	
	\newpage
	\section{Konvergencia}
	
	\begin{definition}{Függvénysorozat, pontonkénti konvergencia}{}
		Azt mondjuk, hogy az \( (f_n) \) egy \emph{függvénysorozat}, 
		ha egy \( \mathcal{D} \neq \emptyset \) halmazzal
		\[
			\func{f_n}{\mathcal{D}}{\R} \qquad (n \in \N).
		\]
		Az \( (f_n) \) függvénysorozat \emph{pontonként konvergens}, 
		ha az \( \bigl( f_n(x) \bigr) \) sorozat minden \( x \in \mathcal{D} \) esetén konvergens. 
		Ekkor az \( f \) \emph{határfüggvénye} legyen
		\[
			f(x) \coloneq \lim_{n \to \infty} f_n(x) \qquad (x \in \mathcal{D}).
		\]
	\end{definition}
	
%	\begin{notes}
%		\item Az általunk vizsgált függvénysorozatok nagyon speciálisak, tudniillik
%		\[
%			\dom{f_n} = [a, b] \qquad (n \in \N).
%		\]
%		
%		\item Az \( (f_n) \) függvénysorozat \( x \in \mathcal{D} \) pontbeli konvergenciája azt jelenti, hogy
%		\[
%			\forall \varepsilon > 0 \text{-hoz }\
%			\exists N \in \N,\
%			\forall n \in \N \ \text{ és } \ n > N \colon \qquad
%			\abs\big{ f_n(x) - f(x) } < \varepsilon.
%		\]
%	\end{notes}
	
	\noindent 
	Legyen adott az \( f_n \in \Riem{[a, b]} \ (n \in \N) \) függvényeknek a konvergens sorozata.\\
	
	\noindent
	\underline{\textbf{Kérdések:}}
	
	\begin{enumerate}[label=\arabic*.]
		\item\label{eq:kérdések-01}
		Konvergens-e az integrálokból képzett \( \left( \int_a^b f_n \right) \) számsorozat?
		
		\item\label{eq:kérdések-02}
		Igaz-e, hogy az \( f \) határfüggvény Riemann-integrálható?
		
		\item\label{eq:kérdések-03}
		Ha az előbbi két kérdésre igen a válasz, akkor fennáll-e az
		\marginnote{
			Másképp fogalmazva teljesül-e az
			\[
				\int_a^b \lim_{n \to \infty} f_n = \lim_{n \to \infty} \int_a^b f_n
			\]
			felcserélhetőség?
		}
		\[
			\int_a^b f = \lim_{n \to \infty} \int_a^b f_n
		\]
		egyenlőség?
	\end{enumerate}
	
	\noindent
	\underline{\textbf{Válaszok:}} Minden további feltétel nélkül ezek nem teljesülnek.
	Ellenpéldák.
	
	\begin{enumerate}[label=\alph*)]
		\item 
		Legyen \( (r_n) \) a \( [0, 1] \) intervallumbeli racionális számoknak egy sorozata, és
		\[
			f_n(x) \coloneq 
			\begin{cases}
				1, & \text{ha } x \in    \{ r_0, \dots, r_n \} \\
				0, & \text{ha } x \notin \{ r_0, \dots, r_n \}
			\end{cases}
			\quad \bigl( x \in [0, 1],\ n \in \N \bigr).
		\]
		Ezek a függvények mind Riemann-integrálhatóak, továbbá
		\[
			D(x) \coloneq 
			\lim_{n \to \infty} f_n(x) =
			\begin{cases}
				1, \quad \text{ha } x \in    \Q \\
				0, \quad \text{ha } x \notin \Q
			\end{cases}
			\quad \bigl( x \in [0, 1] \bigr).
		\]
		Ez pedig a híres \emph{Dirichlet-függvény}, amiről ismeretes, 
		\marginnote{
			Vagyis \aref{eq:kérdések-02} kérdésre nem a válasz!
		}
		hogy \( D \notin \Riem{[0, 1]} \).
		
		\item 
		Jelöljön \( \func{(a_n)}{\N}{\R} \) egy számsorozatot, és legyen (lásd \ref{fig:konvergencia-ellenpélda-01}. ábra)
		\marginnote{
			\centering
			
			% Define the variable n.
			\pgfmathsetmacro{\an}{1.5}
			\pgfmathsetmacro{\overN}{0.45}
			
			\begin{tikzpicture}[scale=0.7]
				\begin{axis}[
					axis lines = middle,
					font=\large,
					xlabel = $x$,
					ylabel = $y$,
					xmin=0,
					xmax=1.1,
					ymin=0,
					ymax=1.6,
					xtick={\overN, 1}, 
					xticklabels={$\nicefrac{1}{n}$, 1}, 
					ytick={\an},
					yticklabels={$a_n$}, 
					xlabel style={xshift=75pt, yshift=-6pt},
					ylabel style={xshift=-7pt, yshift=60pt},
					samples=100,
					enlarge y limits = true,
					]
					
					% Fill in the region under the graph.
					\draw[fill=gray!20] (axis cs:0, 0) rectangle (axis cs:\overN, \an);
					
					% Add the Dasherd line.
					% \addplot[thick, gray, dashed] coordinates {(\overN,0) (\overN, \an)};
					
					% Plot for the function a_n (for 0 <= x < 1/n).
					\addplot[ultra thick, blue] coordinates {(0,\an) (\overN,\an)};
					
					% Plot for the function 0 (for 1/n <= x < 1).
					\addplot[ultra thick, blue] coordinates {(\overN, 0) (1, 0)};
					
					% Adding a point at the origin (0, 0) 
					\addplot[only marks, blue, thin] coordinates {(0, \an)};
					\addplot[only marks, blue, thin, fill=white, mark options={line width=1pt}] coordinates {(\overN, \an)};
					\addplot[only marks, blue, thin] coordinates {(\overN, 0)};
					\addplot[only marks, blue, thin] coordinates {(1, 0)};
					
					% Adding a point at the origin (0, 0).
					% \draw[only marks, black] (axis cs:0, 0) node[anchor=west] {0};
				\end{axis}
			\end{tikzpicture}
			\captionof{figure}{Az \( (f_n) \) sorozat egyik tagja.}
			\label{fig:konvergencia-ellenpélda-01}
		}[-\baselineskip]
		\[
			f_n(x) \coloneq
			\begin{cases}
				a_n  & \text{ha } x \in [0, \nicefrac{1}{n}) \\
				0    & \text{ha } x \in [\nicefrac{1}{n}, 1]
			\end{cases}
			\qquad (1 \leq n \in \N).
		\]
		Ekkor \( (f_n) \) pontonként konvergens és a határfüggvénye \( f \equiv 0 \). Nyilván
		%
		\begin{alignat*}{3}
			f \in \Riem{[0, 1]}&
			\qquad \text{ és } \qquad &&
			\int_0^1 \lim_{n \to \infty} f_n(x) \dd{x} =
			\int_0^1 0 \dd{x} =
			0.\\
			%
			\intertext{%
				Mivel \( f_n \) szakaszonként folytonos minden \( 1 \leq n \in \N \) indexre,
				ezért}
			%
			f_n \in \Riem{[0, 1]}&
			\qquad \text{ és } \qquad&&
			\int_0^1 f_n(x) \dd{x} =
			\int_0^{\frac{1}{n}} a_n \dd{x} + \int_{\frac{1}{n}}^1 0 \dd{x} =
			\frac{a_n}{n}.
		\end{alignat*}
		Vagyis az \( (a_n) \) sorozattól függően nem biztos 
		\ref{eq:kérdések-01} és \ref{eq:kérdések-03} teljesülni fog.
	\end{enumerate}
	
	\newpage
	\begin{definition}{Egyenletes konvergencia}{}
		Azt mondjuk, hogy az \( (f_n) \) függvénysorozat \emph{egyenletesen konvergál} az
		\[
			\func{f}{\mathcal{D}}{\R}
		\]
		függvényhez, 
		ha bármely \( \varepsilon > 0 \)-hoz van olyan \( N \in \N \) küszöbindex, hogy
		\[
%			\forall \varepsilon > 0 \text{-hoz }\
%			\exists N \in \N \
			\forall n \in \N,\ n > N\ \text{és} \ \forall x \in \mathcal{D} \colon \qquad
			\abs\big{ f_n(x) - f(x) } < \varepsilon.
		\]
	\end{definition}
		
	\begin{theorem}{}{}
		Legyen \( \func{f_n}{[a, b]}{\R} \) egy függvénysorozat \( (n \in \N) \).
		Tegyük fel, hogy
		%
		\begin{enumerate}[label=\roman*)]
			\item\label{eq:integrálható-függvénysorozat-01}
			minden \( n \in \N \) index esetén \( f_n \in \Riem{[a, b]} \),
			
			\item\label{eq:integrálható-függvénysorozat-02}
			az \( (f_n) \) egyenletesen konvergál az \( f \coloneq \lim(f_n) \) határfüggvényhez.
		\end{enumerate}
		%
		Ekkor \( f \in \Riem{[a, b]} \) és az 
		integrálok \( \left( \int_a^b f_n \right) \) sorozata konvergens, valamint
		%
		\marginnote{
			Másképp fogalmazva teljesül, hogy
			\[
				\int_a^b \lim_{n \to \infty} f_n = \lim_{n \to \infty} \int_a^b f_n.
			\]
		}
		\[
			\int_a^b f = \lim_{n \to \infty} \int_a^b f_n.
		\]
	\end{theorem}
	\begin{proof}\,\\[6pt]
		%
		\fbox{\( 1. \)} 
		Legyen \( I \subseteq [a, b] \) egy intervallum és \( n \in \N \). 
		Ekkor minden \( x, y \in I \)-re
		\[
			\abs\big{ f(x) - f(y) } \leq
			\abs\big{ f(x) - f_n(x) }   + 
			\abs\big{ f_n(x) - f_n(y) } + 
			\abs\big{ f_n(y) - f(y) }.
		\]
		Továbbá \aref{eq:integrálható-függvénysorozat-02} feltétel miatt
		minden \( \varepsilon > 0 \)-hoz van olyan \( N \in \N \) küszöbindex:
%		Mivel az \( (f_n) \) függvénysorozat egyenletesen konvergens,
%		ezért minden \( \varepsilon > 0 \)-hoz van olyan \( N \in \N \) küszöbindex, hogy
		\[
			\abs\big{ f_n(t) - f(t) } < \varepsilon
			\qquad (t \in I,\ N < n \in \N).
		\]
		Ebből következik, hogy
		\[
			\abs\big{ f(x) - f(y) } < 2 \varepsilon + \abs\big{ f_n(x) - f_n(y) }
			\qquad (x, y \in I, \ N < n \in \N).
		\]
		Innen szuprémumot véve kapjuk, hogy
		\begin{equation}\label{eq:oszcillációs-becslés}
			\mathcal{O}(f, I) < 2 \varepsilon + \mathcal{O}(f_n, I)
			\qquad (N < n \in \N).
			\tag{\( * \)}
		\end{equation}
		
		% ----------------------------------------------------------------------------
		
		% \vspace{6pt}
		\hrule
		\vspace{6pt}
		
		% ----------------------------------------------------------------------------		
		
		\fbox{\( 2. \)}
		Ekkor a \eqref{eq:oszcillációs-becslés} becslés felhasználásával
		\marginnote{
			Az alkalmazott becslések részletesen:
			\begin{align*}
				\omega(f, \tau)
				&\overset{\text{\ref{eq:oszcillációs-becslés}}}{<}
				\sum_{I \in \mathcal{F}(\tau)} 
				\bigl( 2 \varepsilon + \mathcal{O}(f_n, I) \bigr) {\cdot} \abs{I} \\[3pt]
				&= 2\varepsilon \sum_{\mathclap{I \in \mathcal{F}(\tau)}} \abs{I}
				 + \sum_{\mathclap{I \in \mathcal{F}(\tau)}} 
				   \mathcal{O}(f_n, I) {\cdot} \abs{I} \\[3pt]
				&= 2\varepsilon (b-a) + \omega(f_n, \tau) \\[3pt]
				&< 2\varepsilon (b-a) + \varepsilon.
			\end{align*}
		}
		bármilyen \( \tau \subset [a, b] \) felosztás esetén
		\[
			\omega(f, \tau) = 
			\sum_{I \in \mathcal{F}(\tau)} \mathcal{O}(f, I) {\cdot} \abs{I} <
			2\varepsilon (b - a) + \omega(f_n, \tau)
			\qquad (N < n \in \N).
		\]
		Mivel az \( f_n \) integrálható, 
		ezért megadható olyan \( \tau \subset [a, b] \) felosztás, hogy
		\[
			\omega(f_n, \tau) < \varepsilon
			\qquad \Longrightarrow \qquad
			\smash{\uwave{\omega(f, \tau) < \varepsilon \bigl( 2(b-a) + 1 \bigr)}}.
		\]
		Következésképpen \( f \in \Riem{[a, b]} \).
		
		% ----------------------------------------------------------------------------
		
		\vspace{6pt}
		\hrule
		\vspace{6pt}
		
		% ----------------------------------------------------------------------------
		
		\fbox{\( 3. \)} Amennyiben \( N < n \in \N \) egy tetszőleges index, akkor
		\[
			\abs\Bigg{ \int_a^b f_n - \int_a^b f } \leq
		%	\abs\Bigg{ \int_a^b \bigl( f_n - f \bigr) } \leq
			\int_a^b \abs\big{ f_n - f } <
			\int_a^b \varepsilon = 
			\varepsilon (b - a).
		%	\qquad (N < n \in \N).
		\]
		Ez pontosan azt jelenti, hogy az \( \left( \int_a^b f_n \right) \) integrálsorozat konvergens.
	\end{proof}
	
	\newpage
	\section{Teljesség}
	
	Legyen \( f, g \in \Riem{[a, b]} \) és értelmezzük az \( f \) és \( g \) 
	függvények ``távolságát'' a
	\[
		\varrho(f, g) \coloneq \int_a^b \abs\big{f(x) - g(x)} \dd{x}.
	\]
	leképezés segítségével. Megjegyezzük, hogy \( \varrho \) egy úgynevezett félmetrika.\\
	
	\noindent
	\underline{\textbf{Kérdés:}} 
	Amennyiben az \( f_n \in \Riem{[a, b]} \ (n \in \N) \) függvénysorozatra igaz, hogy
	\[
		\int_a^b \abs{f_n - f_m} \longrightarrow 0 \qquad (m, n \to \infty),
	\]
	akkor van-e olyan \( f \in \Riem{[a, b]} \) függvény, amelyre
	\[
		\int_a^b \abs{f_n - f} \longrightarrow 0 \qquad (n \to \infty)
	\]
	teljesül? Ezt nevezzük \emph{Cauchy-kritériumnak}.\\
	
	\noindent
	\underline{\textbf{Válasz:}} Nem, mert megmutatható az alábbi ellenpélda.
	\marginnote{
		\centering
		
		% Define the variable n.
		\pgfmathsetmacro{\n}{5}
		\pgfmathsetmacro{\x}{0.5}
		\pgfmathsetmacro{\fx}{1/(sqrt(\x))}
		\pgfmathsetmacro{\sqrtN}{sqrt(\n)}
		\pgfmathsetmacro{\overN}{0.2}
		
		\begin{tikzpicture}[scale=0.7]
			\begin{axis}[
				axis lines = middle,
				font=\large,
				xlabel = $x$,
				ylabel = $y$,
				xmin=0,
				xmax=1.1,
				ymin=0,
				ymax=2.5,
				xtick={\overN, \x, 1}, 
				xticklabels={$\nicefrac{1}{n}$, $x$, 1}, 
				ytick={0, 1, \fx, \sqrtN},
				yticklabels={$0$, $1$, $\dfrac{1}{\sqrt{x}}$, $\sqrt{n}$}, 
				xlabel style={xshift=75pt, yshift=-6pt},
				ylabel style={xshift=-7pt, yshift=60pt},
				samples=100,
				enlarge y limits = true,
				]
				
				\addplot[thick, gray, dashed] coordinates {(\overN,0) (\overN,\sqrtN)};
				\addplot[thick, gray, dashed] coordinates {(0,1) (1,1)};
				
				\addplot[thick, gray] coordinates {(\x, 0) (\x,\fx)};
				\addplot[thick, gray] coordinates {(0,\fx) (\x,\fx)};
				
				% Plot for the function 1/sqrt(x) (for 1/n < x < 1)
				\addplot[ultra thick, blue, domain=\overN:1] {1/sqrt(x)};
				
				% Plot for the function sqrt(n) (for 0 < x < 1/n)
				\addplot[ultra thick, blue] coordinates {(0,\sqrtN) (\overN,\sqrtN)};
				
				% Adding a point at the origin (0, 0) 
				\addplot[only marks, blue, thin] coordinates {(\overN, \sqrtN)};
				
				% Adding a point at the origin (0, 0) 
				\addplot[only marks, blue, thin] coordinates {(\x, \fx)};
			\end{axis}
		\end{tikzpicture}
		\captionof{figure}{Az \( (f_n) \) sorozat egyik tagja.}
		\label{fig:cauchy-ellenpélda-01}
	}[-2\baselineskip]
	Legyen (lásd \ref{fig:cauchy-ellenpélda-01}. ábra)
	\[
		f_n(x) \coloneq 
		\begin{cases}
			\sqrt{n}            & \quad \text{ha } x \in [0, \nicefrac{1}{n}) \\[6pt]
			\dfrac{1}{\sqrt{x}} & \quad \text{ha } x \in [\nicefrac{1}{n}, 1]
		\end{cases}
		\qquad (n \in \N).
	\]
	Ekkor szakaszonkénti integrálással adódik, hogy
	\[
		\int_a^b \abs\big{f_n - f_m} \longrightarrow 0 \qquad (m, n \to \infty).
	\]
	Indirekt tegyük fel, hogy az \( (f_n) \) függvénysorozat konvergens és legyen 
	%
	\begin{equation*}\label{eq:teljesség-01}
		f \in \Riem{[0, 1]}, \quad
		\int_0^1 \abs\big{f_n - f} \longrightarrow 0
		\qquad (n \to \infty).
		\tag{\( ** \)}
	\end{equation*}
	%
	Ekkor minden olyan \( 0 < x < 1 \) helyen, ahol az \( f \) folytonos, szükségképpen
	\[
		f(x) = \dfrac{1}{\sqrt{x}}.
	\]
	Ugyanis indirekt tegyük fel, hogy valamilyen \( \xi \in (0, 1) \) folytonossági helyen
	\[
		f(\xi) \neq \dfrac{1}{\sqrt{\xi}}
		\qquad \Longrightarrow \qquad 
		3\varepsilon \coloneq \abs\bigg{ f(\xi) - \dfrac{1}{\sqrt{\xi}} } > 0.
	\]
	Mivel \( f \in \ContAt{\xi} \),
	\marginnote{
		\centering
		
		\pgfmathsetmacro{\xiValue}{0.5}
		\pgfmathsetmacro{\deltaValue}{0.25}
%		\pgfmathsetmacro{\epsilonValue}{0.20}
		\pgfmathsetmacro{\intervalLower}{\xiValue - \deltaValue}
		\pgfmathsetmacro{\intervalUpper}{\xiValue + \deltaValue}
		\pgfmathsetmacro{\xiFunctionValueA}{0.6}
		\pgfmathsetmacro{\xiFunctionValueB}{1/sqrt(\xiValue)}
		\pgfmathsetmacro{\epsilonValue}{abs(\xiFunctionValueA - \xiFunctionValueB)/3}
		
		\begin{tikzpicture}[scale=0.7]
			\begin{axis}[
				axis lines = middle,
				font=\large,
				xlabel = $x$,
				ylabel = $y$,
				xmin=0, xmax=1.1, ymin=0, ymax=2,
				xtick={0, \intervalLower, \xiValue, \intervalUpper, 1}, 
				xticklabels={0, $\xi - \delta$, $\xi$, $\xi + \delta$, 1}, 
				ytick={\xiFunctionValueA, \xiFunctionValueB},
				yticklabels={$f(\xi)$, $\dfrac{1}{\sqrt{\xi}}$}, 
				xlabel style={xshift=75pt, yshift=-6pt},
				ylabel style={xshift=-7pt, yshift=60pt},
				samples=100,
				enlarge y limits = true,
				]
				
				
				% Adding a point at the origin (0, 0) 
				\addplot[thick, gray, dashed] coordinates 
				{(\intervalLower, 0) (\intervalLower, \xiFunctionValueB + \epsilonValue)};
				
				\addplot[thick, gray, dashed] coordinates 
				{(\intervalUpper, 0) (\intervalUpper, \xiFunctionValueB + \epsilonValue)};
				
				\addplot[thick, gray, dashed] coordinates 
				{(0, \xiFunctionValueA - \epsilonValue) (\intervalLower, \xiFunctionValueA - \epsilonValue)};
				
				\addplot[thick, gray, dashed] coordinates 
				{(0, \xiFunctionValueA + \epsilonValue) (\intervalLower, \xiFunctionValueA + \epsilonValue)};
				
				\addplot[thick, gray, dashed] coordinates 
				{(0, \xiFunctionValueB - \epsilonValue) (\intervalLower, \xiFunctionValueB - \epsilonValue)};
				
				\addplot[thick, gray, dashed] coordinates 
				{(0, \xiFunctionValueB + \epsilonValue) (\intervalLower, \xiFunctionValueB + \epsilonValue)};
				
				\draw[decorate, decoration={brace,amplitude=4pt,mirror}] 
				(axis cs:\intervalUpper + 0.01,\xiFunctionValueB - \epsilonValue) -- 
				(axis cs:\intervalUpper + 0.01,\xiFunctionValueB) 
				node[midway,xshift=10pt] {$\varepsilon$};
				
				\draw[decorate, decoration={brace,amplitude=4pt}] 
				(axis cs:\intervalUpper + 0.01,\xiFunctionValueB + \epsilonValue) -- 
				(axis cs:\intervalUpper + 0.01,\xiFunctionValueB) 
				node[midway,xshift=10pt] {$\varepsilon$};
				
				\draw[decorate, decoration={brace,amplitude=4pt,mirror}] 
				(axis cs:\intervalUpper + 0.01,\xiFunctionValueA - \epsilonValue) -- 
				(axis cs:\intervalUpper + 0.01,\xiFunctionValueA) 
				node[midway,xshift=10pt] {$\varepsilon$};
				
				\draw[decorate, decoration={brace,amplitude=4pt}] 
				(axis cs:\intervalUpper + 0.01,\xiFunctionValueA + \epsilonValue) -- 
				(axis cs:\intervalUpper + 0.01,\xiFunctionValueA) 
				node[midway,xshift=10pt] {$\varepsilon$};
				
				\draw[Bracket-, ultra thick, blue] 
				(axis cs:\intervalLower,0) -- (axis cs:\intervalLower + 0.01,0);
				\draw[-Bracket, ultra thick, blue] 
				(axis cs:\intervalUpper - 0.01,0) -- (axis cs:\intervalUpper,0);
				
				\draw[fill=gray!20, draw=gray, thick] 
				(axis cs:\intervalLower, \xiFunctionValueA + \epsilonValue) rectangle 
				(axis cs:\intervalUpper, \xiFunctionValueA - \epsilonValue);
				
				\draw[fill=gray!20, draw=gray, thick] 
				(axis cs:\intervalLower, \xiFunctionValueB + \epsilonValue) rectangle 
				(axis cs:\intervalUpper, \xiFunctionValueB - \epsilonValue);
				
				\addplot[only marks, blue, thick] coordinates {(\xiValue, \xiFunctionValueA)};
				\addplot[only marks, blue, thick] coordinates {(\xiValue, \xiFunctionValueB)};
				
				\addplot[ultra thick, blue, dashed] coordinates 
				{(0, \xiFunctionValueA) (\xiValue, \xiFunctionValueA)};
			
				\addplot[ultra thick, blue, dashed] coordinates 
				{(0, \xiFunctionValueB) (\xiValue, \xiFunctionValueB)};
			\end{axis}
		\end{tikzpicture}
		\captionof{figure}{Az \( f(\xi) \) érték környezete.}
		\label{fig:cauchy-ellenpélda-02}
	}[-7.82\baselineskip]
	ezért megadható olyan \( \delta > 0 \) sugár, amivel (lásd \ref{fig:cauchy-ellenpélda-02}. ábra)
	\[
		t \in [\xi - \delta,\, \xi + \delta] \subset [0, 1]
		\qquad \Longrightarrow \qquad 
		\abs\bigg{ f(t) - \dfrac{1}{\sqrt{t}} } \geq \varepsilon.
%		\qquad \bigl( t \in [\xi - \delta,\, \xi + \delta] \subset \bigr).
	\]
	Legyen \( N \in \N \) olyan küszöbindex, hogy \( 1 / N < \delta - \xi \).
	Ekkor minden \( n > N \) indexre
	\[
		\int\limits_0^1 \abs\big{f_n - f} \geq
		\int\limits_{\xi - \delta}^{\xi + \delta} \abs\bigg{ f(t) - \dfrac{1}{\sqrt{t}} } \dd{t} \geq
		\int\limits_{\xi - \delta}^{\xi + \delta} \varepsilon \dd{t} =
		2\delta {\cdot} \varepsilon > 0.
	\]
	Vagyis az alábbi ellentmondásra jutunk
	\[
		\int_0^1 \abs\big{f_n - f} \centernot{\longrightarrow} 0 
		\qquad (n \to \infty).
	\]
	Térjünk vissza a \eqref{eq:teljesség-01} ponthoz.
	\marginnote{
		\begin{theo*}[Lebesgue-kritérium]{\phantomsection\label{th:lebesgue-kritérium}}
			A
			\[
				g \in \Riem{[a,b]}
			\]
			feltétel azzal ekvivalens, hogy az
			\[
				\mathcal{A}_g \coloneq
				\setc[\big]{ x \in [a,b] }{ g \notin \ContAt{x} }
			\]
			halmaza nullamértékű és \( g \) korlátos.
		\end{theo*}
	}[-1.6\baselineskip]
	A \hyperref[th:lebesgue-kritérium]{Lebesgue-kritérium} miatt az \( f \in \Riem{[0, 1]} \) függvény szakadási helyeinek a halmaza nullamértékű. Mivel tetszőleges \( r \in (0, 1) \) esetén a \( (0, r) \)
	intervallum nem nullamértékű, ezért ez tartalmaz folytonossági pontot.
	Azaz van olyan \( \xi_r \in (0, r) \) pont, hogy
	\[
		f \in \ContAt{\xi_r}
		\qquad \Longrightarrow \qquad 
		f(\xi_r) = 
		\frac{1}{\sqrt{ \xi_r }} >
		\frac{1}{\sqrt{ r }} \longrightarrow +\infty
		\qquad (r \to 0).
	\]
	Ebből következik, hogy az \( f \) függvény nem korlátos, ezért \( f \notin \Riem{[0, 1]} \).
\end{document}