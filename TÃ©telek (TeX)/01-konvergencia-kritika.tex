\documentclass[
%parspace, % Add vertical space between paragraphs
%noindent, % No indentation of first lines in each paragraph
%nohyp,	   % No hyphenation of words
%twoside,  % Double sided format
%draft,    % Quicker draft compilation without rendering images
%final,    % Set final to hide todos
]{elteikthesis}[2024/04/26]


% The minted package is also supported for source highlighting
% See elteikthesis_minted.tex for example
%\usepackage[newfloat]{minted}
\usepackage{enumitem}

\makeatletter
\def\biggg{\bBigg@{3}}
\def\bigggm{\mathrel\biggg}
\def\bigggl{\mathopen\biggg}
\def\bigggr{\mathclose\biggg}
\def\Biggg{\bBigg@{3.5}}
\def\Bigggm{\mathrel\Biggg}
\def\Bigggl{\mathopen\Biggg}
\def\Bigggr{\mathclose\Biggg}
\makeatother

% Document's metadata
\title{Mérték, integrál, valószínűség} % title
\subtitle{1. Vizsgatétel}

% The document
\begin{document}
	
	% Set document language
	\documentlang{hungarian}
	%\documentlang{english}
	
	\section{Függvények lokális oszcillációja}
	
	\begin{definition}{Oszcilláció halmazon, lokális oszcilláció}{oszcilláció}
		Legyen \( \funcin{f}{\R}{\R} \), és
		\( A \subseteq \R \) olyan halmaz, hogy \( A \cap \dom{f} \neq \emptyset \).
		Ekkor
		\[
			\mathcal{O}(f, A) \coloneq
			\sup \Bigl\{ \abs\big{f(x) - f(y)} \ \colon\, x, y \in A \cap \dom{f} \Bigr\}
		\]
		az \( f \) függvény \emph{oszcillációja} az \( A \) halmazon.
		Továbbá egy \( z \in \dom{f} \) helyen
		\[
			o_z(f) \coloneq
			\inf \Bigl\{ 
			\mathcal{O}(f, I) \ \colon\, I \subset \R \text{ intervallum},\, z \in \inter(I) 
			\Bigr\}
		\]
		az \( f \) függvény \emph{lokális oszcillációja} a \( z \) pontban.
	\end{definition}
	
	\begin{lemma}{Lokális oszcilláció és a folytonosság kapcsolata}{}
		Legyen \( \funcin{f}{\R}{\R} \), valamint \( z \in \dom{f} \) egy adott pont.
		Ekkor
		\[
			f \in \ContAt{z}
			\quad \iff \quad
			o_z(f) = 0.
		\]
	\end{lemma}
	\begin{proof}\,\\[3pt]
		\Ifstep Ha az \( f \) függvény folytonos \( z \)-ben, 
		akkor tetszőleges \( \varepsilon > 0 \)-hoz
		\[
			\exists \delta > 0 \colon \quad
			\abs\big{f(x) - f(z)} < \varepsilon
			\quad \bigl( x \in \dom{f},\  \abs{x - z} < \delta \bigr).
		\]
		Legyen \( I \coloneq (z - \delta, z + \delta) \).
		Ekkor minden \( x, t \in I \cap \dom{f} \) esetén igaz, hogy
		\[
			\abs\big{f(x) - f(t)} \leq
			\abs\big{f(x) - f(z)} + \abs\big{f(t) - f(z)} <
			2 \varepsilon
			\quad \Longrightarrow \quad
			\smash{\uwave{\mathcal{O}(f, I) < 2 \varepsilon.}}
		\]
		Ebből következik, hogy \( 0 \leq o_z(f) < 2\varepsilon \), ahonnan \( o_z(f) = 0 \) adódik.
		
		% ----------------------------------------------------------------------------
		
		\vspace{6pt}
		\hrule
		\vspace{6pt}
		
		% ----------------------------------------------------------------------------
		
		\Backifstep Most tegyük fel, hogy \( o_z(f) = 0 \), vagyis definíció szerint
		\[
			o_z(f) =
			\inf \Bigl\{ 
			\mathcal{O}(f, I) \ \colon\, I \subset \R \text{ intervallum},\, z \in \inter(I) 
			\Bigr\} = 0.
		\]
		Ekkor bármilyen \( \varepsilon > 0 \)-hoz megadható olyan \( I \subset \R \) intervallum, hogy
		\[
			z \in \inter(I) 
			\qquad \text{és} \qquad
			\mathcal{O}(f, I) < \varepsilon.
		\]
		Mivel \( z \) belső pontja az \( I \)-nek, ezért létezik olyan \( \delta > 0 \) sugár, amivel
		\[
			K_\delta(z) \coloneq (z - \delta, z + \delta) \subset I.
		\]
		Ekkor bármely \( x \in \dom{f} \cap K_\delta(z) \) pontban
		\[
			\abs\big{f(x) - f(z)} \leq \mathcal{O}(f, I) < \varepsilon
			\qquad \Longrightarrow \qquad
			\smash{\uwave{f \in \ContAt{z}.}}
		\]
	\end{proof}
	
	\newpage
	\section{Konvergencia}
	
%	Az \( f \) függvényt sorozatnak nevezzük, amennyiben \( \dom{f} = \N \). Ekkor a sorozat
%	\[
%		f_n \coloneq f(n) \qquad (n \in \N)
%	\]
%	helyettesítési értékét az \( f \) sorozat \( n \)-edik (vagy \( n \) indexű) tagjának nevezzük, és a sorozat jelölésére a függvényeknél használt \( f \) szimbólum helyett az \( (f_n) \) jelölés használjuk.
%	
%	Attól függően, hogy az egyes tagok milyen halmazból vannak, különböző jelzőket használunk a sorozat megnevezésekor. Például azt mondjuk, hogy \( (f_n) \) számsorozat, ha minden tagja szám. Vagy azt mondjuk, hogy \( (f_n) \) halmazsorozat, ha minden tagja halmaz.
%	Hasonló elv szerint értelmezzük a függvénysorozatokat.
	
	A továbbiakban legyen \( [a, b] \subset \R \) egy korlátos és zárt intervallum.
	
	\begin{definition}{Függvénysorozat, pontonkénti konvergencia}{}
		Azt mondjuk, hogy az \( (f_n) \) egy \emph{függvénysorozat}, ha
		\[
			\func{f_n}{[a, b]}{\R} \qquad (n \in \N).
		\]
		Az \( (f_n) \) függvénysorozat \emph{pontonként konvergens}, 
		ha minden \( x \in [a,b] \) helyen az \( \bigl( f_n(x) \bigr) \) számsorozat konvergens. 
		Ekkor az \( f \) \emph{határfüggvénye}
		\[
			f(x) \coloneq \lim_{n \to \infty} f_n(x) \qquad \bigl( x \in [a, b] \bigr).
		\]
	\end{definition}
	
	\begin{notes}
		\item Az általunk vizsgált függvénysorozatok nagyon speciálisak, tudniillik
		\[
			\dom{f_n} = [a, b] \qquad (n \in \N).
		\]
		
		\item Az \( (f_n) \) függvénysorozat \( x \in [a, b] \) pontbeli konvergenciája azt jelenti, hogy
		\[
			\forall \varepsilon > 0 \text{-hoz }\
			\exists N \in \N,\
			\forall n \in \N \ \text{ és } \ n > N \colon \qquad
			\abs\big{ f_n(x) - f(x) } < \varepsilon.
		\]
	\end{notes}
	
	\noindent 
	Legyen adott az \( f_n \in \Riem{[a, b]} \ (n \in \N) \) függvényeknek a konvergens sorozata.\\
	
	\noindent
	\underline{\textbf{Kérdések:}}
	
	\begin{enumerate}
		\item Konvergens-e az integrálokból képzett \( \left( \int_a^b f_n \right) \) számsorozat?
		\item Igaz-e, hogy az \( f \) határfüggvény Riemann-integrálható?
		\item Ha az előbbi két kérdésre igen a válasz, akkor fennáll-e az
		\marginnote{
			Másképp fogalmazva teljesül-e az
			\[
				\int_a^b \lim_{n \to \infty} f_n = \lim_{n \to \infty} \int_a^b f_n
			\]
			felcserélhetőség?
		}
		\[
			\int_a^b f = \lim_{n \to \infty} \int_a^b f_n
		\]
		egyenlőség?
	\end{enumerate}
	
	\noindent
	\underline{\textbf{Válaszok:}} Minden további feltétel nélkül ezek nem teljesülnek.
	
	Megmutatjuk, hogy egy Riemann-integrálható tagokból álló \( (f_n) \) függvénysorozat pontonkénti konvergenciája nem elégséges a határfüggvény Riemann-integrálhatóságához.
	
	Legyen \( (r_n) \) a \( [0, 1] \) intervallumbeli racionális számoknak egy sorozata megszámlálhatóan végtelen [0,1]-beli racionális szám van, ezért sorozatba rendezhetőek), továbbá legyen
	\marginnote{
		Ismert megszámlálhatóan végtelen \( [0, 1] \)-beli racionális szám van, 
		ezért ezek sorozatba rendezhetőek.
	}
	\[
		f_n(x) \coloneq 
		\begin{cases}
			1, & \text{ha } x \in    \{ r_0, \dots, r_n \} \\
			0, & \text{ha } x \notin \{ r_0, \dots, r_n \}
		\end{cases}
		\quad \bigl( x \in [0, 1],\ n \in \N \bigr).
	\]
	Ekkor minden \( n \in \N \) index esetén az \( f_n \) függvény csak véges sok pontban nem nulla, ezért \( f_n \in \Riem{[0, 1]} \).
	Továbbá az \( (f_n) \) pontonként konvergens és a határfüggvénye
	\[
		D(x) \coloneq 
		\lim_{n \to \infty} f_n(x) =
		\begin{cases}
			1, \quad \text{ha } x \in    \Q \\
			0, \quad \text{ha } x \notin \Q
		\end{cases}
		\quad \bigl( x \in [0, 1] \bigr).
	\]
	Ez pedig a híres \emph{Dirichlet-függvény} és ismeretes, hogy \( D \notin \Riem{[0, 1]} \).
	
	Megmutatjuk, hogy egy Riemann-integrálható tagokból álló \( (f_n) \) függvénysorozat pontonkénti konvergenciája nem elégséges a határátmenet és az integrálás operátornak a felcserélhetőségéhez.
	
	Jelöljön \( \func{(a_n)}{\N}{\R} \) egy számsorozatot, és tekintsük a következő függvénysorozatot.
	\[
		f_n(x) \coloneq
		\begin{cases}
			a_n , & \text{ha } 0   \leq x < 1/n \\
			0   , & \text{ha } 1/n \leq x < 1   \\
		\end{cases}
		\qquad (1 \leq n \in \N).
	\]
	Ekkor \( (f_n) \) pontonként konvergens és a határfüggvénye \( f \equiv 0 \). Nyilvánvaló, hogy
	%
	\begin{alignat*}{3}
		f \in \Riem{[0, 1]}&
		\qquad \text{ és } \qquad &&
		\int\limits_0^1 \lim_{n \to \infty} f_n(x) \dd{x} =
		\int\limits_0^1 0 \dd{x} =
		0.\\
		%
		\intertext{%
			Mivel \( f_n \) szakaszonként folytonos minden \( 1 \leq n \in \N \) indexre, ezért}
		%
		f_n \in \Riem{[0, 1]}&
		\qquad \text{ és } \qquad&&
		\int\limits_0^1 f_n(x) \dd{x} =
		\int\limits_0^{\frac{1}{n}} a_n \dd{x} + \int\limits_{\frac{1}{n}}^1 0 \dd{x} =
		\frac{a_n}{n}.
	\end{alignat*}
	%
	Ennek az integrálsorozatnak a határértéke pedig függ az \( (a_n) \) megválasztásától:
	\[
		\lim_{n \to \infty} \int\limits_0^1 f_n(x) \dd{x} =
		\lim_{n \to \infty} \frac{a_n}{n} =
		\begin{cases}
			       c \in \R, & \text{ha } a_n \coloneq c {\cdot} n  \\
			\hfill + \infty, & \text{ha } a_n \coloneq n^2 \\
			\hfill \nexists, & \text{ha } a_n \coloneq (-1)^n {\cdot} n.
		\end{cases}
	\]
	
	\begin{definition}{Egyenletes konvergencia}{}
		Azt mondjuk, hogy az \( (f_n) \) függvénysorozat \emph{egyenletesen konvergál} az
		\[
			\func{f}{[a, b]}{\R}
		\]
		határfüggvényhez, amennyiben 
		\[
			\forall \varepsilon > 0 \text{-hoz }\
			\exists N \in \N \
			\forall n \in \N,\ n > N\ \text{ és } \forall x \in [a, b] \colon \qquad
			\abs\big{ f_n(x) - f(x) } < \varepsilon.
		\]
	\end{definition}
	
	\newpage
		
	\begin{theorem}{}{}
		Legyen \( \func{f_n}{[a, b]}{\R} \) egy függvénysorozat \( (n \in \N) \).
		Tegyük fel, hogy
		%
		\begin{enumerate}[label=(\roman*)]
			\item minden \( n \in \N \) index esetén \( f_n \in \Riem{[a, b]} \),
			\item az \( (f_n) \) egyenletesen konvergál az \( f \coloneq \lim(f_n) \) határfüggvényhez.
		\end{enumerate}
		%
		Ekkor \( f \in \Riem{[a, b]} \) és az 
		integrálok \( \left( \int_a^b f_n \right) \) sorozata konvergens, valamint
		%
		\marginnote{
			Másképp fogalmazva teljesjül, hogy
			\[
				\int_a^b \lim_{n \to \infty} f_n = \lim_{n \to \infty} \int_a^b f_n.
			\]
		}
		\[
			\int_a^b f = \lim_{n \to \infty} \int_a^b f_n.
		\]
	\end{theorem}
	\begin{proof}\,\\[6pt]
		%
		\fbox{\( 1. \)} 
		Legyen \( I \subseteq [a, b] \) egy intervallum és \( n \in \N \). Ekkor
		\[
			\abs\big{ f(x) - f(y) } \leq
			\abs\big{ f(x) - f_n(x) }   + 
			\abs\big{ f_n(x) - f_n(y) } + 
			\abs\big{ f_n(y) - f(y) }
		%	\quad (x, y \in I).
		\]
		bármely \( x, y \in I \) esetén fennáll.
		Mivel az \( (f_n) \) függvénysorozat egyenletesen konvergens,
		ezért minden \( \varepsilon > 0 \)-hoz van olyan \( N \in \N \) küszöbindex, hogy
		\[
			\abs\big{ f_n(t) - f(t) } < \varepsilon
			\qquad (t \in I,\ N < n \in \N).
		\]
		Ebből következik, hogy
		%
		\begin{equation}\label{eq:oszcillációs-becslés}
			\abs\big{ f(x) - f(y) } < 2 \varepsilon + \abs\big{ f_n(x) - f_n(y) }
			\qquad (x, y \in I, \ N < n \in \N).
			\tag{\( * \)}
		\end{equation}
		
		% ----------------------------------------------------------------------------
		
		% \vspace{6pt}
		\hrule
		\vspace{6pt}
		
		% ----------------------------------------------------------------------------		
		
		\fbox{\( 2. \)} 
		%
		Vegyünk egy \( \tau = \{ a = x_0 < x_1 < \dots < x_s = b \} \) felosztást és vizsgáljuk az
		\[
			\omega(f, \tau) = \sum_{k=0}^{s-1} O_k(f) {\cdot} \abs{I_k}
		\]
		oszcillációs összeget, ahol tetszőleges \( k = 0,1,\dots,s-1 \) mellett legyen
		\[
			I_k \coloneq [x_k, x_{k + 1}], \qquad
			O_k(f) \coloneq \sup 
			\Bigl\{ \abs\big{f(x) - f(y)} \ \colon\, x, y \in I_k \Bigr\}.
		%	\quad (k = 0,1,\dots,s-1).
		\]
		Ha most veszünk egy rögzített \( N < n \in \N \) indexet, 
		akkor \eqref{eq:oszcillációs-becslés} alapján
		\[
			O_k(f) < 2\varepsilon + O_k(f_n) \qquad (k = 0,1,\dots,s-1).
		\]
		Ennél fogva teljesül az alábbi becslés:
		%
		\marginnote{
			\begin{align*}
				\omega(f, \tau) 
			%	&=    \sum_{k=0}^{s-1} O_k(f) {\cdot} \abs{I_k} \\
				&\leq \sum_{k=0}^{s-1} \bigl( 2\varepsilon + O_k(f_n) \bigr) {\cdot} \abs{I_k} \\
				&=    2 \varepsilon \sum_{k=0}^{s-1}    \abs{I_k} 
				 +    \sum_{k=0}^{s-1} O_k(f_n) {\cdot} \abs{I_k} \\
				&= 2\varepsilon (b - a) + \omega(f_n, \tau).
			\end{align*}
			és itt
			\[
				\sum_{k=0}^{s-1} \abs{I_k} =
				\sum_{k=0}^{s-1} (x_{k + 1} - x_k) =
				b - a.
			\]
		}
		\[
			\omega(f, \tau) \leq
			%	&=    \sum_{k=0}^{s-1} O_k(f) {\cdot} \abs{I_k} \\
			\sum_{k=0}^{s-1} \bigl( 2\varepsilon + O_k(f_n) \bigr) {\cdot} \abs{I_k} =
			2\varepsilon (b - a) + \omega(f_n, \tau).
		\]
		Mivel az \( f_n \) integrálható, ezért megadható olyan \( \mu \subset [a, b] \) felosztás, hogy
		\[
			\omega(f_n, \mu) < \varepsilon
			\qquad \Longrightarrow \qquad
			\smash{\uwave{\omega(f, \mu) \leq \varepsilon \bigl( 2(b-a) + 1 \bigr)}}.
		\]
		Következésképpen \( f \in \Riem{[a, b]} \).
		
		% ----------------------------------------------------------------------------
		
		\vspace{6pt}
		\hrule
		\vspace{6pt}
		
		% ----------------------------------------------------------------------------
		
		\fbox{\( 3. \)} Amennyiben \( N < n \in \N \) egy tetszőleges index, akkor
		\[
			\abs\Bigg{ \int_a^b f_n - \int_a^b f } \leq
		%	\abs\Bigg{ \int_a^b \bigl( f_n - f \bigr) } \leq
			\int_a^b \abs\big{ f_n - f } <
			\int_a^b \varepsilon = 
			\varepsilon (b - a).
		%	\qquad (N < n \in \N).
		\]
		Ez pontosan azt jelenti, hogy az \( \left( \int_a^b f_n \right) \) integrálsorozat konvergens és
		\[
			\lim_{n \to \infty} \int_a^b f_n = \int_a^b f.
		\]
	\end{proof}
	
	\section{A teljesség kérdése}
	
	Legyen \( f, g \in \Riem{[a, b]} \) és értelmezzük az \( f \) és \( g \) függvények ``távolságát'' a
	\[
		\varrho(f, g) \coloneq \int_a^b \abs\big{f(x) - g(x)} \dd{x}.
	\]
	leképezés segítségével.
	
	Legyen
	\[
		f_n(x) \coloneq 
		\begin{cases}
			\sqrt{n}           , & \text{ha }           0  \leq x \leq \dfrac{1}{n} \\[12pt]
			\dfrac{1}{\sqrt{x}}, & \text{ha } \dfrac{1}{n}    < x \leq 1
		\end{cases}
		\qquad (n \in \N).
	\]
	
	Indirekt tegyük fel, hogy az \( (f_n) \) függvénysorozat konvergens és legyen 
	\[
		\int_0^1 \abs\big{f_n - f} \longrightarrow 0
		\qquad (n \to \infty).
	\]
	Ekkor minden olyan \( 0 < x < 1 \) helyen, ahol az \( f \) folytonos, szükségképpen
	\[
		f(x) = \dfrac{1}{\sqrt{x}}.
	\]
	Különben tegyük fel, hogy valamilyen \( \xi \in (0, 1) \) folytonossági helyen
	\[
		f(\xi) \neq \dfrac{1}{\sqrt{\xi}}
		\qquad \Longrightarrow \qquad 
		\abs\bigg{ f(\xi) - \dfrac{1}{\sqrt{\xi}} } > 0.
	\]
	Mivel \( f \in \ContAt{\xi} \), ezért bármely \( \varepsilon > 0 \)-hoz van olyan \( \delta > 0 \)
	Ekkor a folytonosság alapján elmondható, valamilyen \( \delta > 0 \) sugárral 
	\( [\xi - \delta, \xi + \delta] \subset \)
	\[
		\abs\bigg{ f(t) - \dfrac{1}{\sqrt{t}} } \geq \varepsilon
		\qquad \bigl( t \in [\xi - \delta,\, \xi + \delta] \bigr).
	\]
	Legyen \( N \in \N \) olyan küszöbindex, hogy \( \)
	\[
		\int\limits_0^1 \abs\big{f_n - f} \geq
		\int\limits_{\xi - \delta}^{\xi + \delta} \abs\big{f_n(t) - f(t)} \dd{t} \geq
		\int\limits_{\xi - \delta}^{\xi + \delta} \varepsilon \dd{t} =
		2\delta {\cdot} \varepsilon > 0.
	\]
	vagyis
	\[
		\int_0^1 \abs\big{f_n - f} \nrightarrow 0 
		\qquad (n \to \infty).
	\]
\end{document}