\documentclass[
%parspace, % Add vertical space between paragraphs
%noindent, % No indentation of first lines in each paragraph
%nohyp,	   % No hyphenation of words
%twoside,  % Double sided format
%draft,    % Quicker draft compilation without rendering images
%final,    % Set final to hide todos
]{elteikthesis}[2024/04/26]


% The minted package is also supported for source highlighting
% See elteikthesis_minted.tex for example
%\usepackage[newfloat]{minted}
\usepackage{enumitem}

% Document's metadata
\title{Mérték, integrál, valószínűség} % title
\subtitle{20. Vizsgatétel}


% The document
\begin{document}
	
	% Set document language
	\documentlang{hungarian}
	
	\section{Emlékeztető}
	
	\begin{definition}{Súlyfüggvény}{}
		Legyen \( (X, \Omega, \mu) \) mértéktér, 
		\( f \in \StepFuncPlus \) egy adott függvény. Ekkor a
		\[
		\func{\mu_f}{\Omega}{[0, +\infty]}, \qquad 
		\mu_f(A) \coloneq 
		\int_A f \dd{\mu} \coloneq 
		\int f {\cdot} \chi_A \dd{\mu}
		\]
		leképezést \emph{súlyfüggvénynek} nevezzük.
		%	az \( f \) függvény \emph{integrálja} az \( A \) halmazon.
	\end{definition}
	
	\begin{notes}
		\item Könnyen igazolható, hogy \( \mu_f \) mérték.
		
		\item Amennyiben az \( A \in \Omega \) halmaz nullamértékű, akkor
		\[
			f {\cdot} \chi_A = 0 \ \mu \text{-m.m.}
			\quad \Longrightarrow \quad
			\mu_f(A) = 
			%	\int_A f \dd{\mu} =
			\int f {\cdot} \chi_A \dd{\mu} =
			0.
		\]
	\end{notes}
	
	\begin{definition}{Abszolút folytonosság}{}
		Legyen \( (X, \Omega) \) mérhető tér, 
		valamint \( \func{\mu, \nu}{\Omega}{[0, +\infty]} \) két mérték.\\[6pt]
		Azt mondjuk, 
		hogy \( \nu \) \emph{abszolút folytonos} \( \mu \)-re nézve (jelben \( \nu \ll \mu \)), ha
		\[
			\mu(A) = 0
			\quad \Longrightarrow \quad
			\nu(A) = 0
			\qquad (A \in \Omega).
		\]
	\end{definition}
	
	\begin{lemma}{}{}
		Legyen \( (X, \Omega) \) mérhető tér, 
		\( \func{\mu, \nu}{\Omega}{[0, +\infty]} \) két mérték, ahol \( \nu \) véges.\\[6pt]
		Ekkor \( \nu \ll \mu \) azzal ekvivalens, hogy
		bármely \( \varepsilon > 0 \)-hoz van olyan \( \delta > 0 \):
		\[
		%			\forall \varepsilon > 0 \text{-hoz},\
		%			\exists \delta > 0,
		%			\forall A \in \Omega \colon \
		%			\mu(A) < \delta \quad \longrightarrow \quad \nu(A) < \delta.
		\mu(A) < \delta
		\quad \Longrightarrow \quad
		\nu(A) < \varepsilon
		\qquad (A \in \Omega).
		\]
	\end{lemma}
	
	\section{Radon--Nikodym-tétel}
	
	Korábban megjegyeztük, hogy amennyiben \( (X, \Omega, \mu) \) mértéktér és
	\( f \in \StepFuncPlus \), akkor igaz a
	\[
		\mu(A) = 0
		\quad \Longrightarrow \quad
		\mu_f(A) = 0
		\qquad (A \in \Omega)
	\]
	implikáció, vagyis a \( \mu_f \) mérték abszolút folytonos a \( \mu \) mértékre.
	A most tárgyalásra kerülő Radon--Nikodym-tétel azt állítja, hogy amennyiben a \( \mu \) szigma-véges mérték, akkor \( \mu_f \) az egyetlen olyan mérték, ami eleget tesz ennek.
	
	\newpage
	\begin{theorem}{Radon--Nikodym-tétel}{radon-nikodym}
		Legyen \( (X, \Omega) \) mérhető tér, 
		valamint \( \func{\mu, \nu}{\Omega}{[0, +\infty)} \) mérték. Ekkor
		\[
			\nu \ll \mu
			\quad \iff \quad
			\exists f \in \StepFuncPlus, \text{ hogy } \nu = \mu_f.
		\]
	\end{theorem}
	\begin{proof}
		Tehát feltesszük, hogy a \( \mu, \nu \) mértékek végesek.
		Legyen
		\marginnote{
			Az \( \mathcal{L} \) halmaz valóban nem üres, mert
			\[
				g \equiv 0
				\quad \Longrightarrow \quad
				g \in \mathcal{L}.
			\]
		}
		\[
			\mathcal{L} \coloneq 
			\setc[\big]{ g \in \StepFuncPlus }{ \mu_g \leq \nu } \neq \emptyset.
		\]
		Könnyen belátható módon fennállnak az alábbi állítások.
	%	Ekkor az \( \mathcal{L} \) halmaz nem üres, 
	%	ugyanis a \( g \equiv 0 \) függvény eleme ennek.
		\begin{enumerate}
			\item\label{eq:radon-nikodym-01}
			Tetszőleges \( g, h \in \mathcal{L} \) függvény esetén
			\marginnote{
				Lásd: \( 
					\max \{ g, h \} =
					g {\cdot} \chi_{\{ g \geq h \}} + h {\cdot} \chi_{\{ g < h \}}
				\).
			}
			\( \max \{ g, h \} \in \mathcal{L} \).
			
			\item\label{eq:radon-nikodym-02}
			Létezik az alábbi véges ``integrál-szuprémum''
			\marginnote{
				Ugyanis tetszőleges \( g \in \mathcal{L} \) függvényre
				\[
					\int g \dd{\mu} =
				%	\int g \cdot \chi_X \dd{\mu} =
					\mu_g(X) \leq
					\nu(X) <
					+\infty.
				\]
			}
			\[
				\gamma \coloneq 
				\sup \setc[\bigg]{ \int g \dd{\mu} }{ g \in \mathcal{L} } < +\infty.
			\]
		\end{enumerate}
		
		\vspace{-6pt}
		\hrule
		\vspace{9pt}
		
		\Aref{eq:radon-nikodym-01} és \ref{eq:radon-nikodym-02} állítások alapján 
		megadható olyan	\( (f_n) \) függvénysorozat, hogy
		\[
			f_n \in \mathcal{L}, \quad 
			f_n \leq f_{n + 1}, \quad
			\lim_{n \to \infty} \int f_n \dd{\mu} = \gamma 
			\qquad (n \in \N).
		\]
		Ekkor a \hyperref[th:beppo-levi]{Beppo Levi-tétel} miatt elmondható,
		\marginnote{
			\begin{theo*}[Beppo Levi-tétel]\phantomsection\label{th:beppo-levi}
				Amennyiben
				\[
					g_n \in \StepFuncPlus, \quad 
					g_n \leq g_{n + 1} \quad (n \in \N),
				\]
				akkor \( g \coloneq \lim(g_n) \) szintén \( \StepFuncPlus \)-beli és
				\[
					\int g \dd{\mu} =
					\lim_{n \to \infty} \int g_n \dd{\mu}.
				\]
			\end{theo*}
		}[-1.4\baselineskip]
		hogy az \( (f_n) \) konvergens, azaz
		\[
			f \coloneq \lim(f_n) \in \StepFuncPlus.
		\]
		Továbbá bármilyen \( A \in \Omega \) halmaz esetén az \( (f_n {\cdot} \chi_A) \) sorozat monoton növő és \( \StepFuncPlus \)-beli, ezért szintén a 
		\hyperref[th:beppo-levi]{Beppo Levi-tétel} felhasználásával kapjuk, hogy
		\[
			\lim_{n \to \infty} (f_n {\cdot} \chi_A ) = 
			f { \cdot } \chi_A \in \StepFuncPlus.
		\]
		Továbbá a határátmenet és az integrálás felcseresével
		\[
			\mu_f(A) = 
			\int f {\cdot} \chi_A \dd{\mu} =
		%	\int \lim_{n \to \infty} f_n {\cdot} \chi_A \dd{\mu} =
			\lim_{n \to \infty} \int f_n {\cdot} \chi_A \dd{\mu} =
			\lim_{n \to \infty} \mu_{f_n}(A) \leq
			\nu(A).
		\]
		Speciálisan megmutattuk, hogy \( \mu_f \leq \nu \).
		
		\vspace{6pt}
		\hrule
		\vspace{6pt}
		
		Belátjuk, hogy \( \nu = \mu_f \). Ez azzal ekvivalens, hogy
		\marginnote{
			Hiszen \( A \in \Omega \) esetén \( A \subseteq X \), ezért
			\[
				0 \leq \sigma(A) \leq \sigma(X) = 0.
			\]
		}
		\[
			\sigma \coloneq \nu - \mu_f = 0 
			\quad \iff \quad
			\sigma(X) = 0.
		\]
		Ekkor nyilvánvaló, hogy \( \sigma \) egy véges mérték.
		Továbbá \( \sigma \leq \nu \), ezért \( \nu \ll \mu \) következtében \( \sigma \ll \mu \).
		\marginnote{
			Ha ugyanis \( A \in \Omega, \ \mu(A) = 0 \), akkor
			\[
				0 \leq \sigma(A) \leq \nu(A) = 0.
			\]
		}
%		\[
%			\mu(A) = 0
%			\quad \Longrightarrow \quad
%			\nu(A) = 0
%			\quad \Longrightarrow \quad
%			\sigma(A) = 0
%			\qquad (A \in \Omega).
%		\]
%		Tehát \( \sigma \) abszolút folytonos a \( \mu \)-re nézve.
		Továbbá indirekt módon tegyük fel, hogy
		\[
			\sigma(X) > 0
			\quad \Longrightarrow \quad
			\mu(X) > 0.
		\]
		Ezt követően bizonyítás nélkül elfogadjuk az alábbi lemmát.
		\begin{lemma}{B-lemma}{radon-nikodym-b}
			Megadható olyan \( Y \in \Omega \) halmaz és \( \beta > 0 \) szám, 
			amellyel %\( A \in \Omega \) esetén
			\[
				\sigma(Y) > \beta \cdot \mu(Y), \qquad
			%	\qquad \text{és} \qquad
				\sigma(A \cap Y) \geq \beta \cdot \mu(A \cap Y)
				\qquad (A \in \Omega).
			\]
		\end{lemma}
		
		\newpage
		Ekkor a \hyperref[lem:radon-nikodym-b]{B-lemma} alapján \( \mu(Y) >  0 \).
		Ellenkező esetben a \( \sigma \ll \mu \) abszolút folytonosság miatt ellentmondanánk az előbb említett lemma első pontjának.
		Továbbá tekintsük az \( \StepFuncPlus \)-beli
		\[
			F \coloneq f + \beta \cdot \chi_Y 
		\]
		függvényt. Ekkor tetszőleges \( A \in \Omega \) mellett
		\begin{align*}
			\mu_F(A)
			&= \int F {\cdot} \chi_A \dd{\mu} 
			 = \int f {\cdot} \chi_A \dd{\mu} + \int \beta {\cdot} \chi_{A \cap Y} \dd{\mu} \\[3pt]
			&= \mu_f(A) + \beta {\cdot} \mu(A \cap Y)
			\leq \mu_f(A) + \sigma(A) = \nu(A).
		\end{align*}
		Ezek szerint \( F \in \mathcal{L} \), ahonnan az alábbi ellentmondás adódik:
		\[
			\gamma \geq 
			\int F \dd{\mu} = 
		%	\int f \dd{\mu} + \beta \int \chi_Y \dd{\mu} =
			\int f \dd{\mu} + \beta \cdot \mu(Y) > \gamma.
		\]
		
		
	%	Mindez azt jelenti, hogy \( f \in \mathcal{L} \).
	\end{proof}
	
	\begin{notes}
		\item
		A \hyperref[th:radon-nikodym]{Radon--Nikodym-tétel} abban az esetben is igaz marad,
		amikor a
		\[
			\func{\mu, \nu}{\Omega}{[0, +\infty]}
		\]
		mértékek végessége helyett azt követeljük meg, hogy \( \mu \) szigma-véges legyen.
		
		Ez utóbbi azt jelenti, hogy létezik olyan \( A_n \in \Omega \ (n \in \N) \)
		páronként diszjunkt mérhető halmazokból álló sorozat, amellyel
		\[
			X = \bigcup_{n = 0}^{\infty} A_n
			\quad \text{és} \quad
			\mu(A_n) < +\infty \qquad (n \in \N).
		\]
		
		\item
		A \hyperref[th:radon-nikodym]{Radon--Nikodym-tételben} bevezetett jelölésekkel az alábbi teljesül.
		
		\begin{theo*}
			Legyen \( (X, \Omega) \) mérhető tér, 
			\( \func{\mu}{\Omega}{[0, +\infty]} \) szigma-véges mérték. 
			Ekkor tetszőleges \( f,g \in \StepFuncPlus(\mu) \) esetén
			\[
				\mu_f = \mu_g
				\qquad \Longrightarrow \qquad
				f = g \quad \mu \text{-m.m.}
			\]
		\end{theo*}
		
	\end{notes}
	
	
%	\newpage
%	
%	\begin{theorem}{Radon--Nikodym-tétel}{}
%		Legyen \( (X, \Omega) \) mérhető tér, 
%		valamint \( \func{\mu, \nu}{\Omega}{[0, +\infty)} \) mérték. Ekkor
%		\[
%		\nu \ll \mu
%		\quad \iff \quad
%		\exists f \in \StepFuncPlus, \text{ hogy } \nu = \mu_f.
%		\]
%	\end{theorem}
%	\begin{proof}
%		Ha \( \mu, \nu \) véges mértékek, akkor \( (\mu + \nu) \) is véges mérték. Legyen
%		\[
%			\mathcal{H} \coloneq \Lebesgue[2](\mu + \nu)
%		\]
%		Következésképpen értelmezhető az alábbi funkcionál
%		\[
%			\func{\varphi}{\mathcal{H}}{\R}, \qquad 
%			\varphi(h) \coloneq \int h \dd{\mu}.
%		\]
%		Ezen \( \varphi \) funkcionál korlátos, ugyanis a Hölder-egyenlőtlenséget felhasználva
%		\[
%			\abs\big{ \varphi(h) } =
%			\abs{ \int h \cdot 1 \dd{\mu} } \leq
%			\int \abs{h} \cdot \abs{1} \dd{\mu} \leq
%			\int \abs{h} \cdot \abs{1} \dd{(\mu + \nu)}
%		\]
%	\end{proof}
	
	
\end{document} 