\documentclass[
%parspace, % Add vertical space between paragraphs
%noindent, % No indentation of first lines in each paragraph
%nohyp,	   % No hyphenation of words
%twoside,  % Double sided format
%draft,    % Quicker draft compilation without rendering images
%final,    % Set final to hide todos
]{elteikthesis}[2024/04/26]


% The minted package is also supported for source highlighting
% See elteikthesis_minted.tex for example
%\usepackage[newfloat]{minted}
\usepackage{enumitem}

% Document's metadata
\title{Mérték, integrál, valószínűség} % title
\subtitle{\circled{15} Vizsgatétel}


% The document
\begin{document}
	
	% Set document language
	\documentlang{hungarian}
	
	\section{Beppo Levi tétele}
	
	\begin{theorem}{Beppo Levi}{beppo-levi}
		Legyen \( \func{(f_n)}{\N}{\StepFuncPlus} \) egy monoton növekedő függvénysorozat. 
		\marginnote{
			Tehát az \( (f_n) \) sorozatra
			\[
				\int \lim_{n \to \infty} f_n \dd{\mu} =
				\lim_{n \to \infty} \int f_n \dd{\mu}.
			\]
		}
		Ekkor
		\[
			f \coloneq \lim(f_n) \in \StepFuncPlus
			\qquad \text{és} \qquad
			\int f \dd{\mu} = \lim_{n \to \infty} \int f_n \dd{\mu}.
		\]
	\end{theorem}
	\begin{proof}
		Mivel az \( (f_n) \) sorozat tagjai valamennyien \( \StepFuncPlus \)-ban vannak,
		\marginnote{
			Tehát minden \( n \in \N \) indexhez van olyan
			\[
				\func{(f_{kn})}{\N}{\StepFuncPos}
			\]
			sorozat, ami monoton növekedve tart a
			\[
				\lim_{k \to \infty} f_{kn} = f_n
			\]
			határfüggvényhez.
		}
		így
		\[
			\begin{NiceArray}{cccccc}
				f_{00} & f_{01} & \Cdots & f_{0n} & \Cdots & \nearrow \ f_0 \\
				f_{10} & f_{11} & \Cdots & f_{1n} & \Cdots & \nearrow \ f_1 \\
				\Vdots & \Vdots &        & \Vdots &        & \Vdots         \\
				f_{n0} & f_{n1} & \Cdots & f_{nn} & \Cdots & \nearrow \ f_n \\
			\end{NiceArray}
			\qquad (n \in \N).
		\]
		Tekintsük az alábbi függvénysorozatot:
		\[
			g_n \coloneq \max \setc[\big]{ f_{ij} }{ i,j = 0,\dots,n } \qquad (n \in \N).
		\]
		Ekkor nyilvánvaló, hogy fennállnak a soron következő állítások:
		\[
			g_n \in \StepFuncPos, \quad
			g_n \leq g_{n + 1}, \quad
			g_n \leq f_n \leq f \qquad (n \in \N).
		\]
		Az utóbbi egyenlőtlenség következménye, hogy
		\[
			\lim(g_n) \leq \lim(f_n) = f.
		\]
		Ugyanakkor \( (g_n) \) monoton nő, így tetszőleges \( i, j = 0, \dots, n \) index esetén
		\[
			f_{ij} \leq g_n \leq \lim(g_n).
		\]
		Amennyiben vesszük az \( i, j \to \infty \) határátmenetet, akkor
		\[
			f = \lim(f_n) \leq \lim(g_n).
		\]
		Tehát \( f = \lim(g_n) \),
		ami definíció szerint egy \( \StepFuncPlus \)-beli függvény, továbbá
		\marginnote{
			Kihasználva az
			\[
				g_n \leq f_n \leq f \qquad (n \in \N)
			\]
			egyenlőtlenséget és a monotonitást.
		}
		\[
			\int f \dd{\mu} = 
			\lim_{n \to \infty} \int g_n \dd{\mu} \leq
			\lim_{n \to \infty} \int f_n \dd{\mu} \leq
			\int f \dd{\mu}.
		\]
	\end{proof}
	
	\noindent 
	
	\newpage
	\section{Következmények}
	
	Fogalmazzuk meg a Beppo Levi--tételt tetszőleges \( \StepFuncPlus \)-beli függvénysor esetén!
	
	\begin{theorem}{Beppo Levi--tétel függvénysorokra}{}
		Legyen \( \func{(h_n)}{\N}{\StepFuncPlus} \) egy tetszőleges függvénysorozat. Ekkor
		\[
			\sum_{n=0}^{\infty} h_n \in \StepFuncPlus
			\qquad \text{és} \qquad
			\int \sum_{n=0}^{\infty} h_n \dd{\mu} =
			\sum_{n=0}^{\infty} \int h_n \dd{\mu}.
		\]
	\end{theorem}
	\begin{proof}
		Tekintsük az alábbi részletösszeg-sorozatot:
		\[
			f_n \coloneq \sum_{k=0}^{n} h_k \qquad (n \in \N).
		\]
		Ekkor \( (f_n) \) egy \( \StepFuncPlus \)-beli, 
		monoton növekedő függvényekből álló sorozat, 
		így a \hyperref[th:beppo-levi]{Beppo Levi--tétel} alkalmazásával adódik, 
		hogy az \( (f_n) \) konvergens és
		\[
			\int \sum_{n=0}^{\infty} h_n \dd{\mu} =
			\int \lim_{n \to \infty} f_n \dd{\mu} =
			\lim_{n \to \infty} \int f_n \dd{\mu} =
			\sum_{n=0}^{\infty} \int h_n \dd{\mu}.
		\]
	\end{proof}
	
	
	
	\begin{definition}{Majdnem mindenhol terminológia}{}
		Legyen \( (X, \Omega, \mu) \) egy mértéktér, 
		valamint \( T \) egy tulajdonság az \( X \) elemein.\\[3pt]
		Ekkor \( T \) \emph{majdnem mindenhol} igaz, 
		ha megadható olyan \( A \in \Omega \), hogy
		\begin{enumerate}
			\item \( \mu(A) = 0 \),
			\item minden \( x \in X \setminus A \) esetén a \( T \) igaz \( x \)-ben.
		\end{enumerate}
	\end{definition}
	
	\begin{notes}
		\item
		Használatos még a \( T \) \emph{majdnem mindenütt} igaz megnevezés is,
		illetve ezek rövidítésére a \( T \) \emph{\( \boldsymbol{\mu} \)-m.m.} 
		és (egyetlen mérték esetén) a \( T \) \emph{m.m.} szimbólum.
		
		\item 
		\textbf{\textcolor{red}{Figyelem!}}
		A most bevezetett majdnem mindenhol terminológia azt jelenti, hogy 
		\textit{egy alkalmas nullamértékű halmaz pontjait leszámítva a \( T \) tulajdonság mindenhol teljesül}.
		
		Azt nem állíthatjuk, hogy egy nullamértékű \( A \) halmaz pontjaiban a \( T \) hamis
		és egyébként meg mindenhol igaz.
		Csupán azt követeljük meg, hogy az
		\[
			N \coloneq \setc[\big]{ x \in X }{ T \text{ nem igaz } x \text{-ben} }
		\]
		halmaz lefedhető legyen egy nullamértékű \( A \) halmazzal.
		\marginnote{Vagyis legyen \( N \subseteq A \) és \( \mu(A) = 0 \).}
		
		\item 
		Előfordulhat, hogy \( N \notin \Omega \).
		Viszont, ha a \( \mu \) mérték teljes és \( T \) \( \mu \)-m.m., akkor
		\[
			N \in \Omega
			\qquad \text{és} \qquad
			\mu(N) = 0.
		\]
		
	\end{notes}

	
	\newpage
	\begin{theorem}{}{}
		 Legyen \( (X, \Omega, \mu) \) egy mértéktér, 
		 valamint \( f, g \in \StepFuncPlus \) adott függvények.
		 \begin{enumerate}[label=\arabic*.]
		 	\item\label{eq:majdnem-mindenhol-01}
		 	\( \displaystyle \int f \dd{\mu} = 0 \iff f = 0 \) majdnem mindenhol.
		 	
		 	\item\label{eq:majdnem-mindenhol-02}
		 	Ha \( \displaystyle \int f \dd{\mu}  \) véges, 
		 	akkor \( \abs{f} < +\infty \) majdnem mindenhol.
		 	
		 	\item\label{eq:majdnem-mindenhol-03}
		 	Ha \( f = g \) majdnem mindenhol, 
		 	akkor \( \displaystyle \int f \dd{\mu} = \int g \dd{\mu} \).
		 \end{enumerate}
	\end{theorem}
	\begin{proof}
		\Aref{eq:majdnem-mindenhol-01} állítás bizonyítása.\\[6pt]
		%
		\Ifstep Tekintsük a \( \{ f > q \} \) nívóhalmazt, ahol \( q > 0 \) rögzített. Ekkor
		\[
			f \geq q \cdot \chi_{\{ f > q \}}
			\qquad \Longrightarrow \qquad
			0 = 
			\int f \dd{\mu} \geq 
			q \cdot \mu \bigl( \{ f > q \} \bigr) \geq 0.
		\]
		Ugyanakkor
		\[
			\{ f > 0 \} = 
			\bigcup_{n = 1}^{\infty} \biggl\{ f > \frac{1}{n} \biggr\}
			\qquad \Longrightarrow \qquad
			\mu \bigl( \{ f > 0 \} \bigr) = 0.
		\]
		
		\Backifstep Tekintsük az \( f_n \coloneq n {\cdot} \chi_{\{ f > 0 \}}  \ (n \in \N) \) függvénysorozatot. Ekkor
		\[
			f_n \in \StepFuncPlus, \quad
			f_n \leq f_{n + 1}, \quad
			f \leq \sup_{n} f_n = \lim( f_n )
			\qquad (n \in \N).
		\]
		Alkalmazva a \hyperref[th:beppo-levi]{Beppo Levi--tételt} azt kapjuk, hogy
		\[
			0 \leq
			\int f \dd{\mu} =
			\lim_{n \to \infty} \int f_n \dd{\mu} = 
			\lim_{n \to \infty} \Bigl( n \cdot \mu \bigl( \{ f > 0 \} \bigr) \Bigr) = 0.
		\]
		
		\vspace{6pt}
		\hrule
		\vspace{9pt}
		
		\Aref{eq:majdnem-mindenhol-02} állítás bizonyítása.
		Tetszőleges \( q > 0 \) számmal, hogy \( f \geq q {\cdot} \chi_{\{ f = +\infty \}} \).
		\[
			0 \leq
			q \cdot \mu\bigl( \{ f = +\infty \} \bigr) \leq 
			\int f \dd{\mu}
			< +\infty,
		\]
		ahonnan \( \mu \bigl( \{ f = +\infty \} \bigr) = 0 \).
		
		\vspace{6pt}
		\hrule
		\vspace{9pt}
		
		\Aref{eq:majdnem-mindenhol-03} állítás bizonyítása.
		Tekintsük az \( f \) függvény alábbi felbontását:
		\marginnote{
			Ugyanezen oknál fogva
			\[
				g = g \cdot \chi_{\{ f = g \}} + g \cdot \chi_{\{ f \neq g \}}.
			\]
			Ekkor az integrál additivitása miatt
			\[
				\int g \dd{\mu} =
				\int g {\cdot} \chi_{\{ f = g \}} \dd{\mu},
			\]
			hiszen az \( f = g \) m.m. feltétel alapján
			\[
				\int g {\cdot} \chi_{\{ f \neq g \}} \dd{\mu} = 0.
			\]
		}
		\[
			f = f \cdot \chi_{\{ f = g \}} + f \cdot \chi_{\{ f \neq g \}}
		\]
		Mivel az integrál additív, ennél fogva
		\[
			\int f \dd{\mu} =
			\int f {\cdot} \chi_{\{ f = g \}} \dd{\mu} +
			\int f {\cdot} \chi_{\{ f \neq g \}} \dd{\mu}.
		\]
		Mivel az \( f {\cdot} \chi_{\{ f \neq g \}} = 0 \) majdnem mindenhol igaz, 
		ezért \aref{eq:majdnem-mindenhol-01} állítás miatt
		\[
			\int f \dd{\mu} =
			\int f {\cdot} \chi_{\{ f = g \}} \dd{\mu} =
			\int g {\cdot} \chi_{\{ f = g \}} \dd{\mu} =
			\int g \dd{\mu}.
		\]
	\end{proof}
	
	\begin{lemma}{Fatou--lemma}{}
		Legyen \( (X, \Omega, \mu) \) mértéktér, 
		valamint \( \func{(h_n)}{\N}{\StepFuncPlus} \) egy függvénysorozat.
		\begin{enumerate}[label=\alph*)]
			\item\label{eq:fatou-01}
			Ekkor
			\[
				\liminf(h_n) \in \StepFuncPlus
				\qquad \text{és} \qquad
				\int \liminf_{n \to \infty} h_n \dd{\mu} \leq 
				\liminf_{n \to \infty} \int h_n \dd{\mu}.
			\]
			
			\item\label{eq:fatou-02}
			Tegyük fel, hogy létezik olyan \( F \in \StepFuncPlus \) függvény, amire
			\[
				\int F \dd{\mu} < +\infty
				\qquad \text{és} \qquad
				h_n \leq F \ \mu \text{-m.m.} \qquad (n \in \N).
			\]
			Ekkor
			\[
				\limsup(h_n) \in \StepFuncPlus
				\qquad \text{és} \qquad
				\limsup_{n \to \infty} \int h_n \dd{\mu} \leq 
				\int \limsup_{n \to \infty} h_n \dd{\mu}.
			\]
		\end{enumerate}
	\end{lemma}
	
	\begin{proof}
		\Aref{eq:fatou-01} állítás bizonyításához legyen
		\[
			\liminf(h_n) = 
			\lim_{n \to \infty} \Bigl( \,\inf_{k \geq n} h_k \Bigr) \eqcolon
			\lim( f_n ).
		\]
		Itt \( (f_n) \) egy \( \StepFuncPlus \)-beli monoton növő sorozat, 
		ezért a \hyperref[th:beppo-levi]{Beppo Levi--tétel} alapján
		\[
			\int \liminf_{n \to \infty} h_n \dd{\mu} =
			\int \lim_{n \to \infty} f_n \dd{\mu} =
			\lim_{n \to \infty} \int f_n \dd{\mu}.
		\]
		Mivel \( f_n \leq h_k \ (k \geq n) \),
		\marginnote{
			Ugyanis minden \( k \geq n \) indexre
			\[
				\int f_n \dd{\mu} \leq \int h_k \dd{\mu}.
			\]
		}
		ezért az integrál monotonitása miatt
		\[
			\lim_{n \to \infty} \int f_n \dd{\mu} \leq
			\lim_{n \to \infty} \biggl( \inf_{k \geq n} \int h_k \dd{\mu} \biggr) =
			\liminf_{n \to \infty} \int h_n \dd{\mu}.
		\]
		
		\vspace{6pt}
		\hrule
		\vspace{9pt}
		
		\Aref{eq:fatou-02} állítás bizonyítását nem részletezzük.\\[6pt]
		Csupán megjegyezzük, hogy az \( F \) függvény \( \mu \)-m.m. majorálja a \( (h_n) \) sorozat tagjait, vagyis egy alkalmas \( A \in \Omega \) mérhető halmazzal
		\[
			\mu(A) = 0, \quad 
			h_n(x) \leq F(x) < +\infty \qquad (n \in \N, \ x \in X \setminus A).
		\]
		Ezek után alkalmazzuk \aref{eq:fatou-01} állítást az
		\[
			F - h_n {\cdot} \chi_C \in \StepFuncPlus \qquad (n \in \N)
		\]
		sorozaton, ahol \( C \coloneq X \setminus A \).
	\end{proof}
	
	
	
\end{document} 