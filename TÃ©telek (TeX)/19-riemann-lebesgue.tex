\documentclass[
%parspace, % Add vertical space between paragraphs
%noindent, % No indentation of first lines in each paragraph
%nohyp,	   % No hyphenation of words
%twoside,  % Double sided format
%draft,    % Quicker draft compilation without rendering images
%final,    % Set final to hide todos
]{elteikthesis}[2024/04/26]


% The minted package is also supported for source highlighting
% See elteikthesis_minted.tex for example
%\usepackage[newfloat]{minted}
\usepackage{enumitem}

% Document's metadata
\title{Mérték, integrál, valószínűség} % title
\subtitle{\circled{19} Vizsgatétel}


% The document
\begin{document}
	
	% Set document language
	\documentlang{hungarian}
	
	\section{Emlékeztető}
	
	Alkalmazni fogjuk az alábbi tételt.
	
	\begin{theorem}{Kis Lebesgue-tétel}{kis-lebesgue}
		Legyen \( (X, \Omega, \mu) \) mértéktér, \( \mu \) véges, 
		valamint az \( f_n \in \Lebesgue \ (n \in \N) \) egy olyan függvénysorozat, 
		amelyre a következők igazak:
		\begin{enumerate}[label=\roman*)]
			\item
			majdnem mindenhol létezik a \( \lim(f_n) \) pontonkénti limesz;
			
			\item
			megadható olyan \( C \) konstans, hogy minden \( n \in \N \) indexre
			\[
			%	\abs{f_n} \leq F \qquad \mu \text{-m.m.}
			\abs\big{f_n(x)} \leq C \qquad (\mu \text{-m.m. } x \in X).
			\]
		\end{enumerate}
		Ekkor van olyan \( f \in \Lebesgue \) függvény, 
		hogy majdnem mindenhol \( f = \lim(f_n) \) és
		\[
		%	f = \lim_{n \to \infty} f_n \quad \mu \text{-m.m.}
		%	\qquad \text{és} \qquad
		\int f \dd{\mu} = \lim_{n \to \infty} \int f_n \dd{\mu}.
		\]
	\end{theorem}
	
	\section{Riemann--Lebesgue-tétel}
	
	A továbbiakban feltesszük, hogy adottak az \( a, b \in \R, \ a < b \) számok.
	Ekkor
	\[
		\Omega \coloneq 
		\bigl\{ A \subseteq [a, b] \ \colon \, A \in \widehat{\Omega}_1 \bigr\}
	\]
	a \textit{Lebesgue-mérhető halmazok} szigma-algebrája, valamint legyen 
	\[
		\func{\mu}{\Omega}{[0, +\infty)}, \quad
		\mu(A) \coloneq \widehat{\mu}_1(A) \quad (A \in \Omega)
	\]
	a \textit{Lebesgue-mérték}.
	Mivel \( \mu \) véges, ezért bármely \( 1 \leq p \leq q \leq +\infty \) mellett
	\[
		\Lebesgue[\infty][a, b] \subset
		\Lebesgue[q][a, b] \subset
		\Lebesgue[p][a, b] \subset
		\Lebesgue[1][a, b].
	\]
	a szigorú értelemben vett tartalmazások.
	
	\begin{theorem}{Riemann--Lebesgue-tétel}{}
		Bármely \( \func{f}{[a, b]}{\R} \) Riemann-integrálható függvényre
		\( f \in \Lebesgue[\infty][a, b] \), és
		\[
			\text{Riemann-integrál} \ \longrightarrow \
			\int_a^b f(x) \dd{x} = \int f \dd{\mu}
			\ \longleftarrow \ \text{Lebesgue-integrál}.
		\]
	\end{theorem}
	
	\begin{proof}
		Vegyünk egy \( \tau \coloneq \{ x_0, \dots, x_s \} \subseteq [a, b] \) felosztást és legyen
		\[
			\delta_\tau \coloneq \max \setc[\big]{ x_{i + 1} - x_i }{ i = 0,\dots, s-1 }
		\]
		az felosztás úgynevezett \emph{finomsága}. Továbbá legyen
		\[
			\begin{alignedat}{2}
				m_i &\coloneq \inf && \setc[\big]{ f(x) }{ x_i \leq x \leq x_{i + 1} } \\
				M_i &\coloneq \sup && \setc[\big]{ f(x) }{ x_i \leq x \leq x_{i + 1} }
			\end{alignedat}
			\qquad (i = 0,\dots,s-1).
		\]
		
		\newpage
		
		Vezessük be az alábbi függvényeket
		\begin{align*}
			\varphi &\coloneq
			\sum_{i=0}^{s-2} m_i     {\cdot} \chi_{[x_i, x_{i+1})}
			+                m_{s-1} {\cdot} \chi_{[x_{s-1}, x_s]} \\[6pt]
			\Phi &\coloneq
			\sum_{i=0}^{s-2} M_i     {\cdot} \chi_{[x_i, x_{i+1})}
			+                M_{s-1} {\cdot} \chi_{[x_{s-1}, x_s]}.
		\end{align*}
		Nyilvánvaló, hogy fennállnak az alábbi állítások
		\[
			\varphi, \Phi \in \StepFunc, \quad
			\varphi \leq f \leq \Phi, \quad
			\int \varphi \dd{\mu} = s(f, \tau), \quad
			\int \Phi \dd{\mu} = S(f, \tau).
		\]
		Továbbá az \( \StepFunc \)-beli integrál definíció miatt világos, hogy
		\begin{align*}
			\int \varphi \dd{\mu} &= 
			s(f, \tau) = 
			\sum_{i=0}^{s-1} m_i {\cdot} \bigl( x_{i+1} - x_i \bigr), \\
			\int \Phi \dd{\mu} &= 
			S(f, \tau) = 
			\sum_{i=0}^{s-1} M_i {\cdot} \bigl( x_{i+1} - x_i \bigr).
		\end{align*}
		
		\vspace{6pt}
		\hrule
		\vspace{9pt}
		
		Vegyünk egy minden határon túl finomodó felosztássorozat, azaz legyen
		\[
			\tau_n \subset [a, b] \text{ felosztás}, \quad
			\tau_n \subset \tau_{n + 1} \quad (n \in \N), \qquad
			\lim_{n \to \infty} \delta_{\tau_n} = 0.
		\]
		A Riemann-integrálról tanultak alapján elmondható, hogy
		\[
%			s(f, \tau_n) \nearrow \int\limits_a^b f(x) \dd{x}
%			\qquad \text{és} \qquad
%			S(f, \tau_n) \searrow \int\limits_a^b f(x) \dd{x}
			s(f, \tau_n), \ S(f, \tau_n) \longrightarrow  \int_a^b f(x) \dd{x}
			\qquad (n \to \infty).
		\]
		Következésképpen a megfelelő Lebesgue-integrálsorozatok is
		\[
			\int \varphi_n \dd{\mu}, \
			\int \Phi_n    \dd{\mu} \longrightarrow  \int_a^b f(x) \dd{x}
			\qquad (n \to \infty).
		\]
		Az itt szereplő \( (\varphi_n) \) függvénysorozat monoton növekedő,
		míg a \( (\Phi_n) \) sorozat monoton csökken, továbbá
		\begin{equation}\label{eq:riemann-lebesgue-01}
			\varphi_n \leq f \leq \Phi_n
			\qquad (n \in \N).
			\tag{\( * \)}
		\end{equation}
		Mivel két lépcsősfüggvény különbsége továbbra is lépcsősfüggvény, 
		ezért a \( (\Phi_n - \varphi_n) \) sorozat \( \StepFuncPos \)-beli függvényekből álló monoton csökkenő sorozat.
		\[
			\Psi \coloneq 
			\lim_{n \to \infty} (\Phi_n - \varphi_n) =
			\lim_{n \to \infty} \Phi_n - \lim_{n \to \infty} \varphi_n
		\]
		függvény nemnegatív és mérhető, tehát \( \Psi \in \StepFuncPlus \). 
		\marginnote{
			Mivel \( \Phi_0 \in \StepFunc \), ezért korlátos is.
		}
		Illetve adott \( C \in \R \) esetén
		\[
			\Phi_n - \varphi_n \leq \Phi_0 \leq C \qquad (n \in \N)
		\]
		is fennáll, 
		ezért az \( n \to \infty \) határátmenetet véve \( \Psi \leq C \).
		
		Ekkor a \hyperref[th:kis-lebesgue]{kis Lebesgue-tétel} alkalmazásával kapjuk, hogy
		\[
			\int \Psi \dd{\mu} =
			\lim_{n \to \infty} \int (\Phi_n - \varphi_n) \dd{\mu} =
			\int_a^b f(x) \dd{x} - \int_a^b f(x) \dd{x} =
			0.
		\]
		Innen kapjuk, hogy \( \Psi = 0 \) m.m.
		Ugyanakkor a \eqref{eq:riemann-lebesgue-01} becslés alapján
		\begin{equation*}\label{eq:riemann-lebesgue-02}
			f = g \coloneq \lim(\varphi_n) \quad \mu \text{-m.m.}
			\tag{\( ** \)}
		\end{equation*}
		
		\vspace{6pt}
		\hrule
		\vspace{9pt}
				
		Igazoljuk, hogy \( f \in \Lebesgue[\infty][a, b] \).
		Ehhez azt kell megmutatni, hogy az \( f \) korlátos és mérhető.
		Nyilván az előbbi fennáll, hiszen csakis a korlátos függvények Riemann-integrálhatóak.
		Utóbbihoz megmutatjuk, hogy minden \( A \subseteq \R \) Borel-halmazra
		\[
			f^{-1}[A] \in \Omega.
		\]
		Ugyanis
		\[
			f^{-1}[A] =
			\bigl( f^{-1}[A] \cap \{ f = g \} \bigr) \cup 
			\bigl( f^{-1}[A] \cap \{ f \neq g \} \bigr).
		\]
		Ekkor \eqref{eq:riemann-lebesgue-02} szerint van olyan 
		\( B \in \Omega \) nullamértékű halmaz, 
		hogy \( \{ f \neq g \} \subseteq B \).
		Felhasználva a \( \mu \) mérték teljességét
		\[
			\underbrace{f^{-1}[A] \cap \{ f = g \}}_{\in \Omega} =
			\underbrace{g^{-1}[A]}_{\in \Omega} \cap \underbrace{\{ f = g \}}_{\in \Omega}
			\quad \text{és} \quad
			\underbrace{f^{-1}[A] \cap \{ f \neq g \}}_{\in \Omega} \subseteq B.
		\]
		Tehát \( f^{-1}[A] \in \Omega \) valóban fennáll, vagyis az \( f \) mérhető.
%		\begin{equation*}
%			\begin{tikzpicture}
%				\matrix (m) [
%					matrix of math nodes,
%					row sep=1em,
%					column sep=-0.5em,
%					text height=1.5ex, text depth=0.25ex
%					] {
%					\underbrace{f^{-1}[A] \cap \{ f = g \}} & = & g^{-1}[A] & \cap & \{ f = g \} \\
%					\in	\Omega &  & \in \Omega & & \in \Omega \\
%				};
%				\path[->]
%				%	(m-1-1) edge node[above] {$L$} (m-1-2)
%					(m-2-1) edge node[left]  {} (m-1-1)
%					(m-2-3) edge node[right] {} (m-1-3)
%					(m-2-5) edge node[right] {} (m-1-5);
%			\end{tikzpicture}
%		\end{equation*}

		\vspace{6pt}
		\hrule
		\vspace{9pt}

		Már csak az integrálok egyenlőségét kell megmutatni.
		Ehhez ismételten a \hyperref[th:kis-lebesgue]{kis Lebesgue-tételt} alkalmazzuk, 
		most viszont az \( (\varphi_n) \) függvénysorozatra.
		\[
			\int f \dd{\mu} =
			\lim_{n \to \infty} \int \varphi_n \dd{\mu} =
			\int_a^b f(x) \dd{x}.
		\]
	\end{proof}
	
	\begin{notes}

		\item
		Megmutatjuk, hogy nem minden Lebesgue-integrálható függvény Riemann-integrálható, 
		tehát a Lebesgue-integrál valóban kiterjesztése a Riemann-féle megközelítésnek.
		
		Ugyanis jól ismert, hogy a Dirichlet-féle
		\[
			\func{f}{[0, 1]}{\R}, \qquad
			f(x) =
			\left\{
			\begin{aligned}
				1, & \quad x \in \Q \\
				0, & \quad x \notin \Q
			\end{aligned}
			\right.
		\]
		függvény nem Riemann-integrálható. 
		Ugyanakkor ha vesszük a \( [0, 1] \)-beli racionális számoknak egy 
		\( (r_n) \) sorozatát, akkor az
		\[
			\func{f_n}{[0, 1]}{\R}, \quad
			f_n(x) =
			\left\{
			\begin{aligned}
				1, & \quad x \in \{ r_0, \dots, r_n \} \\
				0, & \quad x \notin \{ r_0, \dots, r_n \}
			\end{aligned}
			\right.
			\quad (x \in [0, 1], \ n \in \N).
		\]
		függvénysorozatról nyilvánvaló, hogy
		\[
			f_n \in \StepFuncPlus, \quad
			f_n \leq f_{n + 1} \quad (n \in \N), \quad
			f = \lim(f_n).
		\]
		Ezért az \( f \) függvény Lebesgue-integrálja
		\[
			\int f \dd{\mu} =
			\lim_{n \to \infty} \int f_n \dd{\mu} = 
			\lim_{n \to \infty} 0 = 
			0.
		\]
	\end{notes}
	
\end{document} 