\documentclass[
%parspace, % Add vertical space between paragraphs
%noindent, % No indentation of first lines in each paragraph
%nohyp,	   % No hyphenation of words
%twoside,  % Double sided format
%draft,    % Quicker draft compilation without rendering images
%final,    % Set final to hide todos
]{elteikthesis}[2024/04/26]


% The minted package is also supported for source highlighting
% See elteikthesis_minted.tex for example
%\usepackage[newfloat]{minted}
\usepackage{enumitem}

\makeatletter
\def\biggg{\bBigg@{3}}
\def\bigggm{\mathrel\biggg}
\def\bigggl{\mathopen\biggg}
\def\bigggr{\mathclose\biggg}
\def\Biggg{\bBigg@{3.5}}
\def\Bigggm{\mathrel\Biggg}
\def\Bigggl{\mathopen\Biggg}
\def\Bigggr{\mathclose\Biggg}
\makeatother

% Document's metadata
\title{Mérték, integrál, valószínűség} % title
\date{2024} % year of defense

% Author's metadata
\author{Egedi Viktor}
\degree{programtervező informatikus BSc}

% Superivsor(s)' metadata
\supervisor{Dr. Simon Péter} % internal supervisor's name
\affiliation{egyetemi tanár} % internal supervisor's affiliation

% University's metadata
\university{Eötvös Loránd Tudományegyetem} % university's name
\faculty{Informatikai Kar} % faculty's name
\department{Numerikus Analízis Tanszék} % department's name
\city{Budapest} % city
\logo{elte_cimer_szines} % logo

% The document
\begin{document}
	
	% Set document language
	\documentlang{hungarian}
	
	A továbbiakban \( X \) egy tetszőleges halmazt jelöl, aminek
	\[
		\powerset{X} \coloneq \setc[\big]{A \text{ halmaz}}{A \subseteq X}
	\]
	az úgynevezett \emph{hatványhalmaza}.
	
	\begin{definition}{Szigma-algebra}{}
		Azt mondjuk, hogy \( \Omega \subseteq \powerset{X} \) egy \emph{szigma-algebra} 
		(\emph{\( \boldsymbol{\sigma} \)-algebra}), ha
		\begin{enumerate}[label=\( \Sigma \)\arabic*.]
			\item\label{ax:sigma-algebra-01}
			\( X \in \Omega \),
			
			\item\label{ax:sigma-algebra-02}
			\( A \in \Omega \quad \Longrightarrow \quad A^c \in \Omega \),
			\marginnote{
				Itt \( A^c = X \setminus A \) jelöli a komplementert.
			}
			
			\item\label{ax:sigma-algebra-03}
			\( A_n \in \Omega \ \ (n \in \N) 
			   \quad \Longrightarrow \quad 
			   \bigcup_{n=0}^{\infty} A_n \in \Omega 
			\).
		\end{enumerate}
	\end{definition}
	
	\begin{stat*}
		Amennyiben \( \Omega \subseteq \powerset{X} \) egy \( \sigma \)-algebra, akkor
		\begin{enumerate}
			\item
			\( \emptyset \in \Omega \),
			
			\item
			\( A, B \in \Omega \quad \Longrightarrow \quad A \cup B,\ A \cap B,\ A \setminus B \in \Omega \),
			
			\item
			 tetszőleges \( A_0, \dots, A_n \in \Omega \) véges halmazsorozat és 
			\( B_n \in \Omega \ (n \in \N) \) esetén
			\[
				\bigcup_{k=0}^{n}      A_k \in \Omega, \qquad
				\bigcap_{k=0}^{n}      A_k \in \Omega, \qquad
				\bigcap_{n=0}^{\infty} B_n \in \Omega.
			\]
		\end{enumerate}
	\end{stat*}
	\begin{proof*}\,
		\begin{enumerate}
			\item 
			A \ref{ax:sigma-algebra-01} és \ref{ax:sigma-algebra-02} axióma alapján
			\( \emptyset = X \setminus X = X^c \in \Omega \).
			
			\item Az unióra való zártság bizonyításához alkalmazzuk a
			\ref{ax:sigma-algebra-03} axiómát az
			\marginnote{
				Ugyanis
				\[
					A \cup B = 
					A \cup B \cup \emptyset \cup \emptyset \cup \cdots \in \Omega.
				\]
			}
			\[
				A_1 \coloneq A, \ A_2 \coloneq B, \ A_n \coloneq \emptyset
				\qquad (2 \leq n \in \N)
			\]
			halmazsorozaton. Ezt, valamint a De Morgan-azonosság felhasználva
			\[
				(A \cap B)^c = A^c \cup B^c \in \Omega
				\quad \xLongrightarrow{\ \text{\ref{ax:sigma-algebra-02}} } \quad
				A \cap B \in \Omega.
			\]
			Végül a különbség képzést a komplementerrel és metszettel kifejezve
			\[
				A \setminus B = A \cap B^c \in \Omega.
			\]
		\end{enumerate}
	\end{proof*}
	
	\begin{example}
		\item Amennyiben \( A \in \powerset{X} \) egy tetszőleges halmaz, akkor az
		\[
			\{ \emptyset, X \}, \qquad \{ \emptyset, A, A^c, X \}, \qquad \powerset{X}
		\]
		halmazrendszerek nyilván mind \( \sigma \)-algebrák. Az elsőt \emph{triviális}, 
		a másodikat pedig az \( A \) halmazt tartalmazó \emph{legszűkebb \( \sigma\)-algebrának} szokás nevezni.
	\end{example}
	
	\begin{definition}{Generált szigma-algebra}{}
		Legyen \( Y \subseteq \powerset{X} \) egy halmazrendszer, és tekintsük az
		\[
			\Sigma \coloneq \Sigma_Y \coloneq 
			\setc[\big]{ \Omega \subseteq \powerset{X} }
					   { \Omega \text{ szigma-algebra}, Y \subseteq \Omega }
		\]
		rendszert. Ekkor az \( Y \) halmazrendszer által \emph{generált szigma-algebra}
		\[
			\Omega(Y) \coloneq \bigcap_{\Omega \in \Sigma} \Omega.
		\]
	\end{definition}
	
	Könnyen belátható, hogy \( \Omega(Y) \) valóban szigma-algebra. 
	Továbbá az is elmondható, hogy \( \Omega(Y) \) a legszűkebb olyan szigma-algebra, ami tartalmazza az \( Y \) halmazrendszer minden elemét, azaz
	\[
		\Omega \subseteq \powerset{X} \text{ szigma-algebra}
		\quad \Longrightarrow \quad
		\Omega(Y) \subseteq \Omega.
	\]
	
	\begin{definition}{Halmazgyűrű}{}
		Azt mondjuk, hogy \( \ring{G} \subseteq \powerset{X} \) egy \emph{halmazgyűrű} (röviden \emph{gyűrű}), ha
		%
		\begin{enumerate}[label=\( \ring{G} \)\arabic*.]
			\item\label{ax:halmazgyűrű-01}
			\( \ring{G} \neq \emptyset \),
			
			\item\label{ax:halmazgyűrű-02}
			\( A, B \in \ring{G} \quad \Longrightarrow \quad A \cup B \in \ring{G} \),
			
			\item\label{ax:halmazgyűrű-03}
			\( A, B \in \ring{G} \quad \Longrightarrow \quad A \setminus B \in \ring{G} \).
		\end{enumerate}
	\end{definition}
	
	\begin{stat*}
		Amennyiben \( \ring{G} \subseteq \powerset{X} \) egy halmazgyűrű, akkor
		\begin{enumerate}
			\item
			\( \emptyset \in \ring{G} \),
			
			\item
			tetszőleges \( A_0, \dots, A_n \in \ring{G} \) véges halmazsorozat esetén
			\[
				\bigcup_{k=0}^{n} A_k \in \ring{G} \qquad \text{és} \qquad
				\bigcap_{k=0}^{n} A_k \in \ring{G}.
			\]
		\end{enumerate}
	\end{stat*}
	\begin{proof*}\,
		\begin{enumerate}
			\item Mivel \ref{ax:halmazgyűrű-01} alapján \( \ring{G} \) nem üres, 
			ezért egy \( A \in \ring{G} \) halmazra alkalmazva a \ref{ax:halmazgyűrű-03} axiómát.
			\[
				\emptyset = A \setminus A \in \ring{G}.
			\]
			
			\item Elég azt igazolni, hogy \( \ring{G} \) zárt a metszetképzésre.
			Ez valóban így van, ugyanis
			\[
				A, B \in \ring{G}
				\quad \xLongrightarrow{\ \text{\ref{ax:halmazgyűrű-03}} } \quad
				A \cap B = A \setminus (A \setminus B) \in \ring{G}.
			\]
			Innen teljes indukcióval könnyen belátható mindkét állítás.
		\end{enumerate}
	\end{proof*}
	
	Tehát a halmazgyűrű egy olyan halmazrendszer
	
	\begin{definition}{Generált halmazgyűrű}{}
		Legyen \( Y \subseteq \powerset{X} \) egy halmazrendszer, és tekintsük a
		\[
			G \coloneq G_Y \coloneq 
			\setc[\big]{ \ring{G} \subseteq \powerset{X} }
			{ \ring{G} \text{ gyűrű}, Y \subseteq \ring{G} }
		\]
		rendszert. Ekkor az \( Y \) halmazrendszer által \emph{generált halmazgyűrű}
		\[
			\ring{G}(Y) \coloneq \bigcap_{\ring{G} \in G} \ring{G}.
		\]
	\end{definition}
\end{document}