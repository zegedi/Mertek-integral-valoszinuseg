\documentclass[
%parspace, % Add vertical space between paragraphs
%noindent, % No indentation of first lines in each paragraph
%nohyp,	   % No hyphenation of words
%twoside,  % Double sided format
%draft,    % Quicker draft compilation without rendering images
%final,    % Set final to hide todos
]{elteikthesis}[2024/04/26]


% The minted package is also supported for source highlighting
% See elteikthesis_minted.tex for example
%\usepackage[newfloat]{minted}
\usepackage{enumitem}

\makeatletter
\def\biggg{\bBigg@{3}}
\def\bigggm{\mathrel\biggg}
\def\bigggl{\mathopen\biggg}
\def\bigggr{\mathclose\biggg}
\def\Biggg{\bBigg@{3.5}}
\def\Bigggm{\mathrel\Biggg}
\def\Bigggl{\mathopen\Biggg}
\def\Bigggr{\mathclose\Biggg}
\makeatother

% Document's metadata
\title{Mérték, integrál, valószínűség} % title
\subtitle{4. Vizsgatétel}

% The document
\begin{document}
	
	% Set document language
	\documentlang{hungarian}
	
	\section{Halmazstruktúrák}
	
	A továbbiakban \( X \) egy tetszőleges halmazt jelöl, aminek a \emph{hatványhalmaza}
	\[
		\powerset{X} \coloneq \setc[\big]{A \text{ halmaz}}{A \subseteq X}.
	\]
	
	\begin{definition}{Szigma-algebra}{}
		Azt mondjuk, hogy \( \Omega \subseteq \powerset{X} \) egy \emph{szigma-algebra} 
		(\emph{\( \boldsymbol{\sigma} \)-algebra}), ha
		\begin{enumerate}[label={(\( \Sigma \)\arabic*)}, ref={\( \Sigma \)\arabic*.}]
			\item\label{ax:sigma-algebra-01}
			\( X \in \Omega \),
			
			\item\label{ax:sigma-algebra-02}
			\( A \in \Omega \quad \Longrightarrow \quad A^c \in \Omega \),
			\marginnote{
				Itt \( A^c = X \setminus A \) jelöli a komplementert.
			}
			
			\item\label{ax:sigma-algebra-03}
			\( A_n \in \Omega \ \ (n \in \N) 
			   \quad \Longrightarrow \quad 
			   \bigcup\limits_{n=0}^{\infty} A_n \in \Omega 
			\).
		\end{enumerate}
	\end{definition}
	
	\begin{example}
		Amennyiben \( A \in \powerset{X} \) egy tetszőleges halmaz, akkor az
		\[
			\{ \emptyset, X \}, \qquad \{ \emptyset, A, A^c, X \}, \qquad \powerset{X}
		\]
		halmazrendszerek nyilván mind \( \sigma \)-algebrák. 
%		Az elsőt \emph{triviális}, a másodikat pedig az \( A \) halmazt tartalmazó \emph{legszűkebb \( \sigma\)-algebrának} szokás nevezni.
	\end{example}
	
	\begin{statement}{Szigma-algebra tulajdonságok}{}
		Amennyiben \( \Omega \subseteq \powerset{X} \) egy \( \sigma \)-algebra,
		akkor igazak az alábbiak.
		\begin{enumerate}[label={(\( \Sigma \)\arabic*)}, ref={\( \Sigma \)\arabic*.}, start=4]
			\item\label{eq:sigma-algebra-04}
			\( \emptyset \in \Omega \).
			
			\item\label{eq:sigma-algebra-05}
			\( 
				A, B \in \Omega 
				\quad \Longrightarrow \quad 
				A \cup B,\ A \cap B,\ A \setminus B \in \Omega
			\).
			
			\item\label{eq:sigma-algebra-06}
			Tetszőleges \( A_0, \dots, A_n \in \Omega \), 
			illetve \( B_n \in \Omega \ (n \in \N) \) halmazok esetén
			\[
				\bigcup_{k=0}^{n}      A_k \in \Omega, \qquad
				\bigcap_{k=0}^{n}      A_k \in \Omega, \qquad
				\bigcap_{n=0}^{\infty} B_n \in \Omega.
			\]
		\end{enumerate}
	\end{statement}
	\begin{proof}\,
		\begin{enumerate}[label={(\( \Sigma \)\arabic*)}, ref={\( \Sigma \)\arabic*.}, start=4]
			\item 
			A \ref{ax:sigma-algebra-01} és \ref{ax:sigma-algebra-02} axióma alapján
			\( \emptyset = X \setminus X = X^c \in \Omega \).
			
			\item Az unióra való zártság bizonyításához alkalmazzuk a
			\marginnote{%
				Ugyanis
				\[
					A \cup B = 
					A \cup B \cup \emptyset \cup \emptyset \cup \cdots \in \Omega.
				\]
			}
			\ref{ax:sigma-algebra-03} axiómát az
			\[
				A_1 \coloneq A, \ A_2 \coloneq B, \ A_n \coloneq \emptyset
				\qquad (2 \leq n \in \N)
			\]
			halmazsorozaton.
			\marginnote{%
				\begin{theo*}[Metszet De Morgan-azonosság]%
					\phantomsection\label{th:metszet-de-morgan}\,\\[-9pt]
					Bármely \( \Gamma \neq \emptyset \) indexhalmaz esetén
					\[
					\mathopen{\raisebox{-0.9ex}{$ \biggl( $}} \,
					\bigcap_{\gamma \in \Gamma} A_\gamma
					\mathopen{\raisebox{-0.9ex}{$ \biggr) $}}^{\! \! c} =
					\bigcup_{\gamma \in \Gamma} A_\gamma^c
					\]
					teljesül, ahol minden \( A_\gamma \) egy halmaz.
				\end{theo*}
			}[-1.6\baselineskip]
			Ezt, valamint a \hyperref[th:metszet-de-morgan]{De Morgan-azonosságot} alkalmazva
			\[
				(A \cap B)^c = A^c \cup B^c \in \Omega
				\quad \xLongrightarrow{\ \text{\ref{ax:sigma-algebra-02}} } \quad
				A \cap B \in \Omega.
			\]
			Végül a különbség képzést a komplementerrel és metszettel kifejezve
			\[
				A \setminus B = A \cap B^c \in \Omega.
			\]
			
			\item 
			A véges metszetképzésre, illetve unióképzésre vonatkozó állítást a \ref{eq:sigma-algebra-05}
			alapesetekből kiindulva teljes indukcióval igazolhatjuk.
			
			Végül a metszet \hyperref[th:metszet-de-morgan]{De Morgan-azonosság} felhasználásával
			\[
				\Biggl( \, \bigcap_{n=0}^{\infty} B_n \Biggr)^{\! \! c} =
				\bigcup_{n=0}^{\infty} B_n^c \in \Omega
				\quad \xLongrightarrow{\ \text{\ref{ax:sigma-algebra-02}} } \quad
				\bigcap_{n=0}^{\infty} B_n \in \Omega.
			\]
		\end{enumerate}
	\end{proof}
	
	\noindent
	A szigma-algebra ugyan nagyon egyszerűnek tűnik, 
	ezzel együtt azonban esetenként egyszerre mégis ``túl sokat'' akaró előírás.
	Ezért -- enyhítve a	kívánalmakat -- 
	egyelőre kevesebb megszorításnak eleget tevő halmazrendszereket vezetünk be. 
	Az ilyen rendszereket illetően elsőként a gyűrű fogalmával ismerkedünk meg.	
	
	\begin{definition}{Halmazgyűrű}{}
		Azt mondjuk, hogy \( \ring{G} \subseteq \powerset{X} \) egy \emph{halmazgyűrű} 
		(röviden \emph{gyűrű}), ha
		%
		\begin{enumerate}[label={(\( \ring{G} \)\arabic*)}, ref={\( \ring{G} \)\arabic*.}]
			\item\label{ax:halmazgyűrű-01}
			\( \ring{G} \neq \emptyset \),
			
			\item\label{ax:halmazgyűrű-02}
			\( A, B \in \ring{G} \quad \Longrightarrow \quad A \cup B \in \ring{G} \),
			
			\item\label{ax:halmazgyűrű-03}
			\( A, B \in \ring{G} \quad \Longrightarrow \quad A \setminus B \in \ring{G} \).
		\end{enumerate}
	\end{definition}
	
	\begin{note}
		Nyilván minden szigma-algebra egyben gyűrű, de ez fordítva nem igaz.
	\end{note}
	
	\begin{statement}{Halmazgyűrű tulajdonságok}{}
		Amennyiben \( \ring{G} \subseteq \powerset{X} \) egy halmazgyűrű, akkor igazak az alábbiak.
		\begin{enumerate}[label={(\( \ring{G} \)\arabic*)}, ref={\( \ring{G} \)\arabic*.}, start=4]
			\item\label{eq:halmazgyűrű-04}
			\( \emptyset \in \ring{G} \).
			
			\item\label{eq:halmazgyűrű-05}
			Tetszőleges \( A_0, \dots, A_n \in \ring{G} \) véges halmazsorozat esetén
			\[
				\bigcup_{k=0}^{n} A_k \in \ring{G} \qquad \text{és} \qquad
				\bigcap_{k=0}^{n} A_k \in \ring{G}.
			\]
		\end{enumerate}
	\end{statement}
	\begin{proof}\,
		\begin{enumerate}
			\item Mivel a \ref{ax:halmazgyűrű-01} axióma alapján \( \ring{G} \) nem üres, ezért
			\[
				A \in \ring{G}
				\quad \xLongrightarrow{\ \text{\ref{ax:halmazgyűrű-03}} } \quad
				\emptyset = A \setminus A \in \ring{G}.
			\]
			
			\item Elég azt igazolni, hogy \( \ring{G} \) zárt a metszetképzésre, ami valóban így van
			\[
				A, B \in \ring{G}
				\quad \xLongrightarrow{\ \text{\ref{ax:halmazgyűrű-03}} } \quad
				A \cap B = A \setminus (A \setminus B) \in \ring{G}.
			\]
			Innen teljes indukcióval könnyen belátható mindkét állítás.
		\end{enumerate}
	\end{proof}
	
	Tovább ``enyhítjük'' a szóban forgó halmazrendszerekkel szembeni elvárásainkat. 
	A későbbi alkalmazásokban látni fogjuk, hogy a gyakorlat számára 
	fontos speciális esetekben még a gyűrű axiómáinál is kevesebbel rendelkező
	halmazrendszerekből elegendő kiindulnunk. Ezek az úgynevezett félgyűrűk.
	
	\begin{definition}{Félgyűrű}{}
		Azt mondjuk, hogy \( \ring{H} \subseteq \powerset{X} \) egy \emph{félgyűrű}, ha
		%
		\begin{enumerate}[label={(\( \ring{H} \)\arabic*)}, ref={\( \ring{H} \)\arabic*.}]
			\item\label{ax:félgyűrű-01}
			\( \ring{H} \neq \emptyset \),
			
			\item\label{ax:félgyűrű-02}
			\( A, B \in \ring{H} \quad \Longrightarrow \quad A \cap B \in \ring{H} \),
			
			\item\label{ax:félgyűrű-03}
			\(
				A, B \in \ring{H}
				\quad \Longrightarrow \quad
				A \setminus B = \bigcup\limits_{k=0}^{n} H_k
			\),
			ahol \( H_0, \dots, H_n \in \ring{H} \) diszjunktak.
		\end{enumerate}
	\end{definition}
	
	\begin{example}
		Legyen \( X \coloneq \R \).
		Ekkor \( \ring{H} \coloneq \setc{ \emptyset, I \subseteq \R }{ I \text{ intervallum} } \)
		félgyűrű.
	\end{example}
	
	\begin{statement}{Félgyűrű tulajdonságok}{}
		Amennyiben \( \ring{H} \subseteq \powerset{X} \) egy félgyűrű, akkor igazak az alábbiak.
		%
		\begin{enumerate}[label={(\( \ring{H} \)\arabic*)}, ref={\( \ring{H} \)\arabic*.}, start=4]
			\item
			\( \emptyset \in \ring{G} \).
			
			\item
			Tetszőleges \( A_0, \dots, A_n \in \ring{G} \) véges halmazsorozat esetén
			\(
				\bigcap\limits_{k=0}^{n} A_k \in \ring{H}
			\).
		\end{enumerate}
	\end{statement}
	\begin{proof}\,
		\begin{enumerate}[label={(\( \ring{H} \)\arabic*)}, ref={\( \ring{H} \)\arabic*.}, start=4]
			\item
			Mivel a \ref{ax:félgyűrű-01} axióma alapján \( \ring{H} \) nem üres, ezért
			\[
				A \in \ring{H}
				\quad \xLongrightarrow{\ \text{\ref{ax:félgyűrű-02}} } \quad
				\emptyset = A \setminus A = \bigcup\limits_{k=0}^{n} H_k
			\]
			valamilyen \( H_0, \dots, H_n \in \ring{H} \) páronként diszjunkt halmazokkal, ezért
			\[
				H_0 = \cdots = H_n = \emptyset \in \ring{H}.
			\]
			
			\item
			Az állítás teljes indukcióval bizonyítható, ahol az alapeset \ref{ax:félgyűrű-02}
		\end{enumerate}
	\end{proof}
	
	\section{Generált halmazstruktúrák}
	
	Könnyen igazolható, hogy tetszőlegesen sok szigma-algebrának a metszete szintén szigma-algebra.
	Hasonló módon igaz, hogy tetszőlegesen sok halmazgyűrű metszete is halmazgyűrű marad.
	Ennél fogva van értelme a soron következő fogalmaknak.
	
	\begin{definition}{Generált szigma-algebra}{}
		Legyen \( Y \subseteq \powerset{X} \) egy halmazrendszer, és tekintsük az
		\[
		\Sigma \coloneq \Sigma_Y \coloneq 
		\setc[\big]{ \Omega \subseteq \powerset{X} }
		{ \Omega \text{ szigma-algebra}, Y \subseteq \Omega }
		\]
		rendszert. Ekkor az \( Y \) halmazrendszer által \emph{generált szigma-algebra}
		\[
		\Omega(Y) \coloneq \bigcap_{\Omega \in \Sigma} \Omega.
		\]
	\end{definition}
	
	\noindent 
	Belátható, hogy \( \Omega(Y) \) a legszűkebb olyan szigma-algebra,
	ami tartalmazza az \( Y \) halmazrendszer minden elemét.
	Vagyis bármely \( \Omega \in \Sigma_Y \) esetén \( \Omega(Y) \subseteq \Omega \).
	
	\begin{definition}{Generált halmazgyűrű}{}
		Legyen \( Y \subseteq \powerset{X} \) egy halmazrendszer, és tekintsük a
		\[
			G \coloneq G_Y \coloneq 
			\setc[\big]{ \ring{G} \subseteq \powerset{X} }
			{ \ring{G} \text{ gyűrű}, Y \subseteq \ring{G} }
		\]
		rendszert. Ekkor az \( Y \) halmazrendszer által \emph{generált halmazgyűrű}
		\[
			\ring{G}(Y) \coloneq \bigcap_{\ring{G} \in G} \ring{G}.
		\]
	\end{definition}
	
	\noindent 
	Könnyen belátható, hogy \( \mathcal{G}(Y) \) valóban gyűrű. 
	Továbbá ez a legszűkebb olyan gyűrű, ami tartalmazza az \( Y \) halmazrendszer minden elemét.
	
	\newpage
	\noindent
	A korábbi halmazstruktúrákkal ellentétben félgyűrűk metszet már nem feltétlenül lesz félgyűrű.
	Ennél fogva nem értelmezhetőjük az előbbiekhez hasonlóan a generált félgyűrű fogalmát.
	
	\begin{lemma}{Félgyűrűvel generált halmazgyűrű szerkezete}{}
		Legyen \( \ring{H} \subseteq \powerset{X} \) egy félgyűrű. Ekkor fennáll az
		\[
			A \in \mathcal{G}(\ring{H})
			\quad \iff \quad
			A = \bigcup_{k=0}^n H_k
		\]
		ekvivalencia, 
		ahol \( n \in \N \) és \( H_0, \dots, H_n \in \ring{H} \) páronként diszjunkt halmazok.
	\end{lemma}
	\begin{proof}
		Tekintsük a következő halmazrendszert
		\[
			\ring{G} \coloneq 
			\setc[\Bigg]{ \bigcup_{k=0}^n H_k }
			            { H_0, \dots, H_n \in \ring{H} \text{ páronként diszjunktak} \ (n \in \N)}.
		\]
		Azt kell megmutatnunk, hogy \( \ring{G} = \mathcal{G}(\ring{H}) \).
		
		\vspace{6pt}
		\hrule
		\vspace{6pt}
		
		Először belátjuk, hogy \( \ring{G} \) gyűrű.
		Legyenek \( A, B \in \ring{G} \) tetszőleges halmazok,
		\marginnote{%
			Tehát az itt szereplő
			\[
				H_0, \dots, H_n \in \ring{H}, \quad
				\widetilde{H}_0, \dots, \widetilde{H}_m \in \ring{H}
			\]
			halmazok (a saját csoportjukon belül) páronként diszjunktak.
		}
		\[
			A = \bigcup_{k=0}^n H_k 
			\quad \text{és} \quad
			B = \bigcup_{\ell=0}^m \widetilde{H}_\ell
			\qquad (m, n \in \N).
		\]
		Ekkor \( \ring{G} \) zárt a metszetképzésre, ugyanis
		\[
			A \cap B = 
			\bigcup_{k=0}^n \bigcup_{\ell=0}^m \bigl( H_k \cap \widetilde{H}_\ell \bigr)
			\quad \Longrightarrow \quad
			A \cap B \in \ring{G}.
		\]
		Hiszen az itt szereplő metszethalmazok \( \ring{H} \)-beliek és páronként diszjunktak.
		Ennek segítségével belátjuk, hogy \( \ring{G} \) zárt a különbség képzésre, mert
		\[
			A \setminus B = 
			\bigcup_{k=0}^n \bigcap_{\ell=0}^m \bigl( H_k \setminus \widetilde{H}_\ell \bigr)
			\eqcolon \bigcup_{k=0}^n X_k.
		\]
		Ekkor a félgyűrűkre vonatkozó axióma, valamint \( \ring{G} \) miatt az \( X_k \) halmazok mind páronként diszjunktak. Következésképpen \( A \setminus B \in \ring{G} \).
		Továbbá
		\[
			A \cup B = (A \setminus B) \cup B \in \ring{G}
		\]
		is teljesül. Tehát \( \ring{G} \) valóban gyűrű.
		
		\vspace{6pt}
		\hrule
		\vspace{6pt}
		
		Nyilvánvaló, hogy \( \ring{H} \subseteq \ring{G} \).
		Mivel a \( \mathcal{G}(\ring{H}) \) gyűrű zárt az unió képzésre, ezért
		\[
			\ring{G} \subseteq \mathcal{G}(\ring{H}).
		\]
		
		Továbbá a \( \mathcal{G}(\ring{H}) \) generált gyűrű a legszűkebb \( \ring{H} \)-t 
		tartalmazó gyűrű, ezért 
		\[
			\mathcal{G}(\ring{H}) \subseteq \ring{G}.
		\]
		Összefoglalva \( \ring{G} = \mathcal{G}(\ring{H}) \).
	\end{proof}
	
\end{document}