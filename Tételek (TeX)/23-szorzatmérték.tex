\documentclass[
%parspace, % Add vertical space between paragraphs
%noindent, % No indentation of first lines in each paragraph
%nohyp,	   % No hyphenation of words
%twoside,  % Double sided format
%draft,    % Quicker draft compilation without rendering images
%final,    % Set final to hide todos
]{elteikthesis}[2024/04/26]


% The minted package is also supported for source highlighting
% See elteikthesis_minted.tex for example
%\usepackage[newfloat]{minted}
\usepackage{enumitem}

% Document's metadata
\title{Mérték, integrál, valószínűség} % title
\subtitle{23. Vizsgatétel}


% The document
\begin{document}
	
	% Set document language
	\documentlang{hungarian}
	
	\section{Szorzatmérték}
	
	Legyen adott az \( (X_i, \Omega_i, \mu_i) \ (i=1,2) \) mértéktér, majd tekintsük az
	\[
		X \coloneq X_1 \times X_2
	\]
	kétdimenziós teret.
%	a két alaphalmaz Descartes-szorzata.
	A cél egy olyan \( \Omega \subseteq \powerset{X} \) szigma-algebra és 
	\( \func{\mu}{\Omega}{[0, +\infty]} \) mérték megkonstruálása,
	amely rendelkezik a
	\[
		\mu(U \times V) = \mu_1(U) \cdot \mu_2(V) 
		\qquad (U \in \Omega_1, \ V \in \Omega_2)
	\]
	alábbi művelettartási tulajdonsággal.
	Kézenfekvőnek tűnik, hogy legyen
	\[
		\Omega \coloneq
		\Omega_1 \otimes \Omega_2 \coloneq
		\Omega \bigl( \setc{ U \times V }{ U \in \Omega_1, \ V \in \Omega_2 } \bigr).
	\]
%	valamint
%	\[
%		\Omega \coloneq
%		\Omega_1 \otimes \Omega_2 \coloneq
%		\Omega \bigl( \setc{ U \times V }{ U \in \Omega_1, \ V \in \Omega_2 } \bigr).
%	\]
%	az \( \Omega_1, \Omega_2 \) halmazok Descartes-szorzatai által meghatározott 
%	legszűkebb szigma-algebra.
%	\[
%		\func{f}{X}{[0, +\infty]}
%	\]
%	kétváltozós függvényeket vizsgáljuk az integrálhatóság és az integrál 
%	kiszámítása szempontjából. 
%	Az itt említett ``integrál'' (szigma-véges mértékek	esetén) majd a \( \mu_1, \mu_2 \) mértékekből származó \( \mu_1 \otimes \mu_2 \) szorzatmérték szerint lesz értendő.
	
	\begin{statement}{}{}
		Tetszőleges \( A \in \Omega \) halmaz 
		és \( x \in \Omega_1, y \in \Omega_2 \) elemek esetén
		\begin{enumerate}[label=\alph*)]
			\item \( A_x \coloneq \setc[\big]{ z \in X_2 }{ (x, z) \in A } \in \Omega_2. \)
			\item \( A^y \coloneq \setc[\big]{ v \in X_1 }{ (v, y) \in A } \in \Omega_1. \)
		\end{enumerate}
	\end{statement}
	
	\noindent
	Értelmezzünk most minden \( A \in \Omega \) halmaz mellett két újabb leképezést az
	\begin{alignat*}{3}
		\func{f_A}{X_1}{[0, +\infty]},& \quad 
		&{} f_A(x) &\coloneq \mu_2(A_x) \qquad &&(x \in X_1) \\[3pt]
		\func{f^A}{X_2}{[0, +\infty]},& \quad 
		&{} f^A(y) &\coloneq \mu_1(A^y) \qquad &&(y \in X_2)
	\end{alignat*}
	módon.
	
	\begin{statement}{}{}
		Legyenek \( (X_i, \Omega_i, \mu_i) \ (i=1,2) \) szigma-véges mértékterek.\\[6pt]
		Ekkor tetszőleges \( A \in \Omega \) halmaz esetén 
		\( f_A \in \StepFuncPlus(\mu_1) \) és \( f^A \in \StepFuncPlus(\mu_2) \),
		\[
			\int f_A \dd{\mu_1} = 
			\int f^A \dd{\mu_2}.
		\]
	\end{statement}
	
	\begin{definition}{Szorzatmérték}{}
		Legyenek \( (X_i, \Omega_i, \mu_i) \ (i=1,2) \) szigma-véges mértékterek.\\[6pt]
		Ekkor a \( \mu_1, \mu_2 \) által meghatározott \emph{szorzatmérték}
		\[
			(\mu_1 \otimes \mu_2) (A) \coloneq
			\int f_A \dd{\mu_1} = 
			\int f^A \dd{\mu_2}
			\qquad (A \in \Omega).
		\]
	\end{definition}
	
	\begin{notes}
		\item A \( \mu \coloneq (\mu_1 \otimes \mu_2) \) szorzatmérték szintén szigma-véges mérték.
		\item Az előbbi szorzatmértékre teljesül az alábbi ``művelettartási'' tulajdonság
		\[
			\mu(U \times V) = \mu_1(U) \cdot \mu_2(V)
			\qquad (U \in \Omega_1, \ V \in \Omega_2).
		\]
	\end{notes}
	
	\newpage
	\section{Fubini-tétel}
	
	\noindent
	Legyen a korábbi \( \func{f}{X}{\overline{\R}} \) kétváltozós függvény 
	és \( x \in X_1, y \in X_2 \) esetén
	%
	\begin{alignat*}{3}
		\func{f_x}{X_2}{\overline{\R}},& \quad 
		&{} f_x(z) &\coloneq f(x, z) \qquad &&(z \in X_2) \\[3pt]
		\func{f^y}{X_1}{\overline{\R}},& \quad 
		&{} f^y(v) &\coloneq f(v, y) \qquad &&(v \in X_1).
	\end{alignat*}
	%
	parciális függvényeket.
	
	\begin{statement}{}{}
		Ha az \( f \) mérhető, akkor az \( f_x, f^y \) parciális függvények is mérhetőek.
	\end{statement}
	
	\vspace{6pt}
	\noindent
	Legyen az előbbi parciális függvények esetén
	\begin{alignat*}{3}
		\func{\phi_f}{X_1}{\overline{\R}},& \quad 
		&{} \phi_f(x) &\coloneq \int f_x \dd{\mu_2} \qquad &&(x \in X_1), \\[3pt]
		\func{\phi^f}{X_2}{\overline{\R}},& \quad 
		&{} \phi^f(y) &\coloneq \int f^y \dd{\mu_1} \qquad &&(y \in X_2).
	\end{alignat*}
	
	\begin{theorem}{Tonelli-tétel}{tonelli}
		Ekkor
		\begin{enumerate}[label=\alph*)]
			\item \( \phi_f \in \StepFuncPlus(\mu_1), \ \phi^f \in \StepFuncPlus(\mu_2) \).
			\item
		\end{enumerate}
		\[
			\int f \dd{\mu} = 
			\int \phi_f \dd{\mu_1} =
			\int \phi^f \dd{\mu_2}.
		\]
	\end{theorem}
	
	\vspace{6pt}
	\noindent
	Amennyiben alkalmazzuk a hagyományos
	\[
		\int f \dd{\mu} = 
		\int f(x, y) \dd{x} \dd{y}
	\]
	jelölést, akkor a \hyperref[th:tonelli]{Tonelli-tétel} értelmében
	\[
		\int \biggl( \int f(x, y) \dd{x} \biggr) \dd{y} =
		\int \biggl( \int f(x, y) \dd{y} \biggr) \dd{x}.
	\]
	
	\begin{theorem}{Fubini-tétel}{}
		Legyen \( \func{f}{X}{\overline{\R}} \) egy Lebesgue-integrálható függvény, 
		azaz \( f \in \Lebesgue(\mu) \).\\[6pt]
		Ekkor az alábbiak igazak.
		\begin{enumerate}[label*=\alph*)]
			\item \( \mu_1 \)-m.m. \( x \in X_1 \) elemre \( f_x \in \Lebesgue(\mu_2) \).
			\item \( \mu_2 \)-m.m. \( y \in X_2 \) elemre \( f^y \in \Lebesgue(\mu_1) \).
			\item Fennállnak a \hyperref[th:tonelli]{Tonelli-tételben} mondott állítások.
		\end{enumerate}
	\end{theorem}
	
\end{document} 