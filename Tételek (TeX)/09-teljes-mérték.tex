\documentclass[
%parspace, % Add vertical space between paragraphs
%noindent, % No indentation of first lines in each paragraph
%nohyp,	   % No hyphenation of words
%twoside,  % Double sided format
%draft,    % Quicker draft compilation without rendering images
%final,    % Set final to hide todos
]{elteikthesis}[2024/04/26]


% The minted package is also supported for source highlighting
% See elteikthesis_minted.tex for example
%\usepackage[newfloat]{minted}
\usepackage{enumitem}

% Document's metadata
\title{Mérték, integrál, valószínűség} % title
\subtitle{9. Vizsgatétel}


% The document
\begin{document}
	
	% Set document language
	\documentlang{hungarian}
	
	\section{Emlékeztető}
		
	\begin{theorem}{Kvázimérték kiterjesztése mértékké}{}
		Legyen \( X \) egy halmaz, 
		\( \ring{G} \subseteq \powerset{X} \) gyűrű, 
		\( \func{ \widetilde{\mu} }{ \ring{G} }{ [0, +\infty] } \) kvázimérték.\\[6pt]
		Ekkor van olyan \( \Omega \subseteq \powerset{X} \) \( \sigma \)-algebra és 
		\( \func{ \mu }{ \Omega }{ [0, +\infty] } \) mérték, hogy 
		\[
			\ring{G} \subseteq \Omega
			\qquad \text{és} \qquad
			\widetilde{\mu} = \restr{\mu}{\ring{G}}.
		\]
	\end{theorem}
	
	\begin{definition}{Külső mérték, \( \boldsymbol{\mu^*} \)-mérhető}{}
		Legyen \( X \) egy halmaz, 
		valamint \( \func{\mu^*}{ \powerset{X} }{ \overline{\R} } \) olyan halmazfüggvény, ami
		\begin{enumerate}
			\item 
			\textbf{nemnegatív}, azaz \( \mu^* \geq 0 \);
			
			\item 
			\textbf{eltűnik \( \boldsymbol{\emptyset} \)-ban}, azaz
			\( \mu^*(\emptyset) = 0 \);
			
			\item 
			\textbf{monoton}, 
			azaz minden \( B \subseteq A \) esetén \( \mu^*(B) \leq \mu^*(A) \);
			
			\item
			\textbf{\( \boldsymbol{\sigma}\)-szubadditív}, 
			azaz minden \( A_n \ (n \in \N ) \) halmazsorozat esetén
			\[
			\mu^* \Biggl( \, \bigcup\limits_{n=0}^\infty A_n \Biggr) \leq 
			\sum\limits_{n=0}^{\infty} \mu^*(A_n).
			\]
		\end{enumerate}
		Ekkor azt mondjuk, hogy a \( \mu^* \) halmazfüggvény egy \emph{külső mérték}.\\[6pt]
		Továbbá egy \( A \in \powerset{X} \) halmazt \emph{\( \boldsymbol{\mu^*} \)-mérhetőnek}
		\marginnote{
			Ekvivalens, hogy minden \( B \in \powerset{X} \)-re
			\[
				\mu^*(B) \geq \mu^*(B \cap A) + \mu^*(B \setminus A).
			\]
		}
		nevezünk, amennyiben
		\[
			\forall B \in \powerset{X} \ \colon \quad
			\mu^*(B) = \mu^*(B \cap A) + \mu^*(B \setminus A).
		\]
	\end{definition}
	
	\begin{theorem}{Caratheodory-tétel}{caratheodory}
		Legyen \( X \) egy halmaz, 
		\( \func{\mu^*}{ \powerset{X} }{ \overline{\R} } \) külső mérték, valamint	
		\[
			\Omega \coloneq \setc[\big]{ A \in \powerset{X} }{ A \ \mu^* \text{-mérhető} }.
		\]
		Ekkor \( \Omega \) szigma-algebra és \( \mu \coloneq \restr{\mu^*}{\Omega} \) mérték.
	\end{theorem}
	
	\section{A kiterjesztés egyértelműsége}
	
	Ha egy \( X \) halmaz és \( \ring{G} \subseteq \powerset{X} \) gyűrű mellett
	\( \func{\widetilde{\mu}}{\ring{G}}{[0, +\infty]} \) kvázimérték, és a
	\[
		\func{\mu}{\Omega}{[0, +\infty]}
	\]
	mérték kiterjesztése az \( \widetilde{\mu} \)-nek, akkor az minden esetben kijelenthető, hogy
	\[
		\ring{G} \subseteq \Omega( \ring{G} ) \subseteq \Omega.
	\]
	Felmerül a kérdés, hogy vajon a \( \ring{G} \) gyűrű által generált \( \Omega( \ring{G} ) \) szigma-algebrára hányféleképpen terjeszthetjük ki \( \widetilde{\mu} \)-t?
	Teljesül az egyértelmű kiterjeszthetőség?
	
	\newpage
	\begin{example}
		Legyen \( X \neq \emptyset \) tetszőleges halmaz,
		és tekintsük a \( \ring{G} = \{ \emptyset \} \) triviális gyűrűt,
		\[
			\func{\widetilde{\mu}}{\ring{G}}{[0, +\infty]}, \qquad
			\widetilde{\mu}(\emptyset) = 0, \qquad
			\Omega( \ring{G} ) = \{ \emptyset, X \}.
		\]
		Ekkor \( \widetilde{ \mu } \) kvázimérték, valamint legyen
		\[
		%	\func{\mu_1, \mu_2}{\Omega(G)}{[0, +\infty]}, \qquad
			\mu_1( \emptyset ) = \mu_2( \emptyset ) = 0, \qquad
			\mu_1( X ) = 0, \qquad
			\mu_2( X ) = +\infty.
		\]
		Világos, hogy \( \mu_1 \) és \( \mu_2 \) mérték a triviális szigma-algebrán, ugyanakkor
		\[
			\mu_1 \neq \mu_2, \qquad 
			\restr{ \mu_1 }{ \ring{G} } = \restr{ \mu_2 }{ \ring{G} }.
		\]
	\end{example}
	
	\begin{definition}{Szigma-véges halmazfüggvény}{}
		Legyen \( X \) adott halmaz, 
		valamint \( \funcin{\varphi}{\powerset{X}}{[0, +\infty]} \) halmazfüggvény.\\[6pt]
		Azt mondjuk, hogy \( \varphi \) \emph{szigma-véges} 
		(röviden \emph{\( \boldsymbol{\sigma} \)-véges}), amennyiben
		\[
			X = \bigcup_{n=0}^{\infty} A_n 
			\quad \text{és} \quad
			\varphi(A_n) < +\infty \qquad (n \in \N)
		\]
		igaz, valamilyen \( A_n \in \dom{\varphi} \) páronként diszjunkt halmazokból álló sorozatra.
	\end{definition}
	
	\begin{theorem}{Szigma-véges kvázimérték kiterjesztése}{}
		Legyen \( \ring{G} \subseteq \powerset{X} \) gyűrű, 
		valamint \( \func{ \widetilde{\mu} }{ \ring{G} }{ [0, +\infty] } \) egy kvázimérték.\\[6pt]
		Amennyiben \( \widetilde{\mu} \) szigma-véges, akkor egyértelműen létezik olyan
		\[
			\func{ \mu }{ \Omega(\ring{G}) }{ [0, +\infty] }
		\]
		mérték, ami kiterjesztése \( \widetilde{\mu} \)-nak.
	%	vagyis \( \widetilde{\mu} = \restr{ \mu }{ \ring{G} } \).
	\end{theorem}
	
	\section{Teljes mérték}
	
	\begin{definition}{Teljes mérték}{}
		Legyen \( (X, \Omega, \mu) \) egy mértéktér.
		Azt mondjuk, hogy \( \mu \) \emph{teljes}, ha minden
		\[
			A \in \Omega, \ \mu(A) = 0
		\]
		nullamértékű halmazra esetén a \( B \subset A \) halmaz is mérhető, azaz \( B \in \Omega \).
	\end{definition}
	
	\begin{stat*}
		A \hyperref[th:caratheodory]{Caratheodory-tételben} szereplő 
		\( \mu = \restr{\mu^*}{\Omega} \) mérték teljes.
	\end{stat*}
	\begin{proof*}
		Legyen \( A \in \Omega, \ \mu(A) = 0 \) és 
		vegyünk egy tetszőleges \( B \subset A \) halmazt.\\[3pt]
		Azt kell megmutatnunk, hogy \( B \) egy \( \mu^* \)-mérhető halmaz.
		Ehhez vegyük észre, hogy
		\[
			0 \leq \mu^*(B) \leq \mu^*(A) = \mu(A) = 0
			\quad \Longrightarrow \quad
			\mu^*(B) = 0.
		\]
		Ha \( Z \subseteq X \) egy tetszőleges halmaz, akkor
		\[
			0 \leq 
			\mu^*(Z \cap B) \leq 
			\mu^*(B) =
			0
			\quad \Longrightarrow \quad
			\mu^*(Z \cap B) = 0.
		\]
		Innen rögtön adódik, hogy
		\[
			\mu^*(Z \cap B) + \mu^*(Z \setminus B) \leq
			\mu^*(Z).
		\]
		Ez pedig azzal ekvivalens, hogy \( B \) valóban \( \mu^* \)-mérhető.
		Tehát \( B \in \Omega \). \qed
	\end{proof*}
	
	\newpage
	\section{Lebesgue--mérték}
	
	Legyen \( p \in \posN \) egy rögzített kitevő. 
	Ekkor az \( \x, \y \in \R^p, \ \x < \y \) vektorok esetén az
	\[
	[\x, \y) \coloneq 
	\setc[\big]{ \vb{z} \in \R^p }{ \x \leq \vb{z} < \y }
	\]
	halmazt az \( \x, \y \) végpontú, balról zárt és jobbról nyílt (\( p \)-dimenziós) intervallumnak nevezzük. Könnyen belátható ilyenkor, hogy az
	\[
	\mathbf{I}^p \coloneq 
	\setc[\Big]{ \emptyset,\, [\x, \y) }{ \x, \y \in \R^p \,\text{ és }\, \x < \y }
	\]
	halmazrendszer egy félgyűrű. Tekintsük az \( \mathbf{I}^p \) által generált gyűrűt, vagyis az
	\[
	\mathcal{I}^p \coloneq 
	\mathcal{G}( \mathbf{I}^p ) =
	\setc[\Bigg]{ \bigcup_{k=0}^n I_k }
	{ I_0, \dots, I_n \in \mathbf{I}^p \text{ páronként diszjunktak } (n \in \N) }
	\]
	halmazt.
	Ezek segítségével definiáljuk az \( \mathbf{I}^p \)-n az alábbi additív halmazfüggvényt.
	\[
		m_p( \emptyset ) \coloneq 0, \qquad
		m_p \bigl( [\x, \y) \bigr) \coloneq \prod_{i=1}^{p} (y_i - x_i) \qquad
		( \x, \y \in \R^p, \ \x < \y)
	\]
	
	\begin{theorem}{}{}
		Egyetlen olyan
		\( \func{ \widetilde{\mu}_p }{\mathcal{I}^p}{[0, +\infty)} \) kvázimérték létezik, amire
		\( m_p = \restr{ \widetilde{\mu}_p }{ \mathbf{I}^p } \).
	\end{theorem}
	
	
	\vspace{3pt}
	\noindent
	Ezt az egyértelműen létező \( \widetilde{\mu}_p \) függvényt 
	\emph{Lebesgue-kvázimértéknek} hívjuk. 
	Ha
	\[
		A \in \mathcal{I}^p
		\quad \iff \quad
		A = \bigcup_{k=0}^{n} I_k
	\]
	valamilyen \( I_0, \dots, I_n \in \mathbf{I}^p \) páronként diszjunkt intervallumok esetén, akkor
	\[
		\widetilde{\mu}_p(A) = \sum_{k=0}^{n} m_p(I_k).
	\]
	
	\begin{definition}{Lebesgue-mérték}{}
		Tekintsük a soron következő külső mértéket:
		\[
			\mu_p^*( A ) \coloneq
			\inf \setc[\Bigg]{ \sum_{n=0}^{\infty} \widetilde{\mu}_p(A_n) }
			                 { \func{(A_n)}{\N}{\mathcal{I}^p}, \ A \subseteq \bigcup_{n=0}^{\infty} A_n }
			\quad (A \subseteq \R^p).
		\]
		Legyen az úgynevezett \emph{Lebesgue-mérhető} halmazok szigma-algebrája
		\[
			\widehat{\Omega}_p \coloneq
			\setc[\Big]{ A \subseteq \R^p }{ \text{az } A \ \mu_p^* \text{-mérhető} }.
		\]
		Ekkor a \( \widehat{ \mu }_p \coloneq \restr{ \mu_p^* }{ \widehat{\Omega}_p } \) 
		függvényt az \( \R^p \)-beli \emph{Lebesgue-mértéknek} hívjuk.
	\end{definition}
	
	\begin{note}
		A \( \widehat{ \mu }_p \) mérték szigma-véges és teljes.
	\end{note}
	
	\newpage
	\section{Lebesgue--Stieltjes-mérték}
	
	Legyen \( \func{\varphi}{\R}{\R} \) egy tetszőleges monoton növő, balról folytonos függvény,
	\[
		m_\varphi( \emptyset ) = 0, \quad
		m_\varphi \bigr( [a, b) \bigr) \coloneq \varphi(b) - \varphi(a) \qquad 
		(a, b \in \R, \ a < b).
	\]
	
	\begin{theorem}{}{}
		Egyetlen olyan
		\( \func{ \widetilde{\mu}_\varphi }{\mathcal{I}}{[0, +\infty)} \) kvázimérték létezik, amire
		\( m_\varphi = \restr{ \widetilde{\mu}_\varphi }{ \mathbf{I} } \).
	\end{theorem}
	
	\vspace{3pt}
	\noindent
	Ezt az egyértelműen létező \( \widetilde{\mu}_p \) függvényt 
	\emph{Lebesgue-kvázimértéknek} hívjuk. 
	Ha
	\[
		A \in \mathcal{I}
		\quad \iff \quad
		A = \bigcup_{k=0}^{n} I_k
	\]
	valamilyen \( I_0, \dots, I_n \in \mathbf{I} \) páronként diszjunkt intervallumok esetén, akkor
	\[
		\widetilde{\mu}_\varphi(A) = \sum_{k=0}^{n} m_\varphi(I_k).
	\]
	
	\vspace{3pt}
	\noindent
	Alkalmazzuk a \hyperref[th:caratheodory]{Caratheodory-tételt}
	az \( \widehat{\mu}_\varphi \) Stieltjes-féle kvázimértékre.
	
	\begin{definition}{Lebesgue--Stieltjes-mérték}{}
		Tekintsük a soron következő külső mértéket:
		\[
			\mu_\varphi^*( A ) \coloneq
			\inf \setc[\Bigg]{ \sum_{n=0}^{\infty} \widetilde{\mu}_\varphi(A_n) }
			{ \func{(A_n)}{\N}{\mathcal{I}}, \ A \subseteq \bigcup_{n=0}^{\infty} A_n }
			\quad (A \subseteq \R).
		\]
		Legyen a \emph{Lebesgue--Stieltjes-mérhető} halmazok szigma-algebrája
		\[
		\widehat{\Omega}_\varphi \coloneq
		\setc[\Big]{ A \subseteq \R }{ \text{az } A \ \mu_\varphi^* \text{-mérhető} }.
		\]
		Ekkor a \( \widehat{ \mu }_\varphi \coloneq \restr{ \mu_\varphi^* }{ \widehat{\Omega}_\varphi } \) 
		függvény az \( \R \)-beli \emph{Lebesgue--Stieltjes-mérték}.
	\end{definition}
	
	\begin{note}
		A \( \widehat{ \mu }_\varphi \) mérték szigma-véges és teljes.
	\end{note}
	
\end{document}