\documentclass[
%parspace, % Add vertical space between paragraphs
%noindent, % No indentation of first lines in each paragraph
%nohyp,	   % No hyphenation of words
%twoside,  % Double sided format
%draft,    % Quicker draft compilation without rendering images
%final,    % Set final to hide todos
]{elteikthesis}[2024/04/26]


% The minted package is also supported for source highlighting
% See elteikthesis_minted.tex for example
%\usepackage[newfloat]{minted}
\usepackage{enumitem}

% Document's metadata
\title{Mérték, integrál, valószínűség} % title
\subtitle{11. Vizsgatétel}

% The document
\begin{document}
	
	% Set document language
	\documentlang{hungarian}
	
	\section{Emlékeztető}
	
	Legyen \( p \in \posN \) egy rögzített kitevő. 
	Ekkor az \( \x, \y \in \R^p, \ \x < \y \) vektorok esetén az
	\[
		[\x, \y) \coloneq 
		\setc[\big]{ \vb{z} \in \R^p }{ \x \leq \vb{z} < \y }
	\]
	halmazt az \( \x, \y \) végpontú, balról zárt és jobbról nyílt (\( p \)-dimenziós) intervallumnak nevezzük. Könnyen belátható ilyenkor, hogy az
	\[
		\mathbf{I}^p \coloneq 
		\setc[\Big]{ \emptyset,\, [\x, \y) }{ \x, \y \in \R^p \,\text{ és }\, \x < \y }
	\]
	halmazrendszer egy félgyűrű. Tekintsük az \( \mathbf{I}^p \) által generált gyűrűt, vagyis az
	\[
		\mathcal{I}^p \coloneq 
		\mathcal{G}( \mathbf{I}^p ) =
		\setc[\Bigg]{ \bigcup_{k=0}^n I_k }
		{ I_0, \dots, I_n \in \mathbf{I}^p \text{ páronként diszjunktak } (n \in \N) }
	\]
	halmazt.
	
	\begin{definition}{Borel--halmaz}{}
		Az \( \R^p \)-beli \emph{Borel--halmazok} rendszere az alábbi szigma-algebra:
		\[
		\Omega_p \coloneq \Omega( \mathcal{I}^p ) = \Omega( \mathbf{I}^p ).
		\]
	\end{definition}
	
	\section{Invariancia}
	
	Definiáljuk egy tetszőleges \( A \in \Omega_p \) Borel-halmaz és \( \vb{a} \in \R^p \) vektor esetén az
	\[
		\vb{a} + A  \coloneq \setc{ \vb{a} + \x }{ \x \in \vb{a} }, \qquad
		-A \coloneq \setc{ -\x }{ \x \in \vb{a} }
	\]
	halmazokat. Ekkor igazak a soron következő invariancia tulajdonságok.
	
	\begin{statement}{Eltolás és tükrözés invariancia}{eltolás-tükrözés-invariancia}
		Legyen \( \mu_p \) az \( \R^p \)-beli Lebesgue--Borel-mérték.
		Ekkor az alábbiak igazak.
		\begin{enumerate}[label=\alph*)]
			\item
			\emph{Eltolás invariancia:} 
			bármely \( A \in \Omega_p \) halmazra és \( \vb{a} \in \R^p \) vektorra
			\[
				\vb{a} + A \in \Omega_p
				\qquad \text{és} \qquad
				\mu_p( \vb{a} + A ) = \mu_p(A).
			\]
			
			\item
			\emph{Tükrözés invariancia:} 
			tetszőleges \( A \in \Omega_p \) halmaz esetén
			\[
				-A \in \Omega_p
				\qquad \text{és} \qquad
				\mu_p( -A ) = \mu_p(A).
			\]
		\end{enumerate}
	\end{statement}
	
	\begin{note}
		Ugyanez fennáll a \( \widehat{\mu}_p \) Lebesgue-mérték esetén is.
	\end{note}
	
	\newpage
	\section{Példa nem Borel-mérhető halmazra}
	
	\begin{statement}{}{}
		Létezik olyan \( A \subseteq \R \) halmaz, ami nem Borel-mérhető, 
		azaz \( A \notin \Omega_1. \)
	\end{statement}
	\begin{proof}
		Vezessük be az alábbi relációt
		\[
			x \sim y
			\quad \ratio\Longleftrightarrow \quad
			x - y \in \Q.
		\]
		Ekkor fennállnak az alábbi állítások.
		\begin{enumerate}
			\item A \( \sim \) ekvivalenciareláció.
			\item Bármely \( A \) ekvivalenciaosztály esetén \( A \cap [0, 1) \neq \emptyset \).
		\end{enumerate}
		Ezt felhasználva legyen \( K \subseteq [0, 1) \) az a halmaz, 
		ami minden ekvivalenciaosztályból pontosan egy elemet tartalmaz és más eleme nincsen.
		Ekkor
		\[
			\R = \bigcup_{r \in \Q} (r + K)
		\]
		és ez páronként diszjunkt felbontás. Ekkor \( K \notin \Omega_1 \).
		
		\vspace{9pt}
		\hrule
		\vspace{9pt}
		
		Végül indirekt tegyük fel, hogy \( K \in \Omega_1 \).
		Ekkor az \hyperref[st:eltolás-tükrözés-invariancia]{eltolás invariancia} miatt
		\[
			\mu_1( \R ) = 
			+\infty =
			\sum_{r \in \Q} \mu_1(r + K) = 
			\sum_{r \in \Q} \mu_1(K).
		\]
		Innen egyszerűen adódik, hogy \( \mu_1(K) > 0 \). Ugyanakkor legyen
		\[
			B \coloneq \Q \cap [0, 1).
		\]
		Ekkor \( \mu_1 \) monotonitása miatt
		\[
			\bigcup_{r \in B} (r + K) \subseteq [0, 2)
			\quad \Longrightarrow \quad
			\mu_1(B) \leq 2.
		\]
		Viszont az \hyperref[st:eltolás-tükrözés-invariancia]{eltolás invariancia} alapján
		ellentmondásra jutunk:
		\[
			\mu_1(B) =
			\sum_{r \in B} \mu_1(r + K) =
			\sum_{r \in B} \mu_1(K) =
			+\infty.
		\]
	\end{proof}
	
\end{document}