\documentclass[
%parspace, % Add vertical space between paragraphs
%noindent, % No indentation of first lines in each paragraph
%nohyp,	   % No hyphenation of words
%twoside,  % Double sided format
%draft,    % Quicker draft compilation without rendering images
%final,    % Set final to hide todos
]{elteikthesis}[2024/04/26]


% The minted package is also supported for source highlighting
% See elteikthesis_minted.tex for example
%\usepackage[newfloat]{minted}
\usepackage{enumitem}

% Document's metadata
\title{Mérték, integrál, valószínűség} % title
\subtitle{\circled{14} Vizsgatétel}


% The document
\begin{document}
	
	% Set document language
	\documentlang{hungarian}
	
	\section{Emlékeztető}
	
	Emlékezzünk arra, hogy egy \( A \subseteq \R \) halmazt \emph{Borel-halmaznak} nevezünk, ha
	\[
		A \in \Omega_1 \coloneq \Omega( \mathcal{I} ) = \Omega( \mathbf{I} ).
	\]
	Az itt szereplő \( \mathbf{I} \) halmazrendszer az üres halmazt, 
	valamint az \( \R \) balról zárt, jobbról nyílt intervallumait tartalmazza,
	\( \mathcal{I} \) pedig az \( \mathbf{I} \) félgyűrű által generált gyűrű, azaz
	\[
		\mathbf{I} \coloneq 
		\setc[\Big]{ \emptyset, [a, b) \subseteq \R }{ a, b \in \R, \ a < b }, \qquad
		\mathcal{I} \coloneq
		\mathcal{G}( \mathbf{I} ).
	\]
	\begin{definition}{Mérhető függvény}{}	
		Legyen \( (X, \Omega) \) mérhető tér, valamint \( \func{f}{X}{\R} \) egy függvény.\\[6pt]
		Azt mondjuk, hogy az \( f \) függvény \emph{mérhető} (vagy \emph{Borel--mérhető}), ha
		\[
			f^{-1}[A] \coloneq
			\setc[\big]{ x \in X }{ f(x) \in A } \in \Omega
			\qquad \bigl( A \in \Omega_1 \bigr).
		\]
	\end{definition}
	
	\begin{definition}{Lépcsősfüggvény}{}
		Legyen \( (X, \Omega) \) mérhető tér, valamint \( \func{f}{X}{\R} \) egy függvény.\\[6pt]
		Azt mondjuk, hogy \( f \) egy \emph{lépcsősfüggvény}, 
		ha mérhető és \( \rng{f} \) véges.
	\end{definition}
	
	\vspace{6pt}
	\noindent
	Egy \( f \) leképezés pontosan akkor lépcsősfüggvény, ha kifejezhető az
	\[
		f = \sum_{y \in \rng{f}} y {\cdot} \chi_{\{ f = y \}}
	\]
	úgynevezett \emph{kanonikus alakban}. Továbbá bevezettük az alábbi osztályokat
	\[
		\StepFunc \coloneq \setc[\big]{ \func{f}{X}{\R} }{ f \text{ lépcsős}}, \qquad
		\StepFuncPos \coloneq \setc[\big]{ f \in \StepFunc }{ f \geq 0 }. 
	\]
	
	\begin{definition}{Nemnegatív lépcsősfüggvény integrálja}{}
		Egy \( f \in \StepFuncPos \) függvény (\( \mu \) mérték szerinti) \emph{integrálja} alatt az
		\[
		\int f \dd{\mu} \coloneq
		\sum_{y \in \rng{f}} y {\cdot} \mu \bigl( \{ f = y \} \bigr)
		\]
		nemnegatív számot (vagy a \( +\infty \)-t) értjük.
	\end{definition}
	
	\begin{theorem}{Az integrál alaptulajdonságai}{}
		Tekintsük az \( f, g \in \StepFuncPos \) függvényeket és az \( \alpha \geq 0 \) számot. 
		\marginnote{
			\begin{stat*}
				Az tételben foglalt jelölésekkel
				\[
					f + g \quad \text{és} \quad \alpha \cdot f
				\]
				szintén \( \StepFuncPos \)-beli függvény.
			\end{stat*}
		}[-1.8\baselineskip]
		Ekkor
		\begin{enumerate}
			\item \( \displaystyle \int (\alpha \cdot f) \dd{\mu} = \alpha \cdot \int f \dd{\mu} \);
			\item \( \displaystyle \int (f + g) \dd{\mu} = \int f \dd{\mu} + \int g \dd{\mu} \);
			\item \( \displaystyle f \leq g \quad \Longrightarrow \quad \int f \dd{\mu} \leq \int g \dd{\mu} \);
		\end{enumerate}
	\end{theorem}
	
	\newpage
	\section{Az integrál kiterjesztése}
	
	\begin{theorem}{}{kiterjesztés-függvénysorozattal}
		Legyen adott egy \( \StepFuncPos \)-beli, monoton növekedő függvénysorozat:
		\[
			f_n \in \StepFuncPos, \quad f_n \leq f_{n + 1} \qquad (n \in \N).
		\]
		Ha valamilyen \( g \in \StepFuncPos \) függvény
		esetén \( g \leq \sup\limits_{n \in \N} f_n \), akkor
		\[
			\int g \dd{\mu} \leq \sup_{n} \int f_n \dd{\mu}.
		\]
	\end{theorem}
	\begin{proof}
		Amennyiben \( 0 \leq c < 1 \) egy rögzített konstans, akkor az
		\marginnote{
			Ezen halmazsorozatra az igaz, hogy
			\[
				A_n \in \Omega, \quad (A_n) \nearrow X.
			\]
			Az utóbbi jelölés a következőt jelenti:
			\[
				A_n \subseteq A_{n + 1}, \quad X = \bigcup_{n = 0}^{\infty} A_n.
			\]
		}
		\[
			A_n \coloneq \bigl\{ f_n \geq c \cdot g \bigr\} \quad (n \in \N)
		\]
		nívóhalmazok \( \Omega \)-ban vannak, és monoton növekvő módon tartanak \( X \)-hez.
		
		\begin{enumerate}
			\item
			Mivel az \( f_n, g \ (n \in \N) \) függvények mind mérhetőek, ezért \( A_n \in \Omega \). \checkmark
			
			\item 
			Mivel az \( (f_n) \) sorozat monoton nő, ezért az \( (A_n) \) monoton bővül. \checkmark
			
			\item Ha \( x \in X \) tetszőleges, akkor
			\[
				c \cdot g(x) \leq g(x) \leq \sup_{n \in \N} f_n(x).
			\]
			Ebből kifolyólag, valamint az \( (f_n) \) sorozat monoton növekedés miatt
			\[
				\exists n \in \N \, \colon \quad
				c \cdot g(x) \leq f_n(x)
				\quad \Longrightarrow \quad
				x \in A_n.
			\]
			Tehát az \( (A_n) \) halmazsorozat valóban \( X \)-hez tart. \checkmark
		\end{enumerate}
		
		Mivel \( \mu \) mérték, valamint bármely \( Z \in \Omega \) esetén 
		az \( (A_n \cap Z) \nearrow Z \), ezért
		\begin{equation}\label{eq:majoráló-lépcsősfüggvény-sorozat}
			\mu(A_n \cap Z) \longrightarrow \mu(Z) \quad (n \to \infty). \tag{\( * \)}
		\end{equation}
		
		\vspace{6pt}
		\hrule
		\vspace{9pt}
		
		Vegyük észre, hogy a korábbiak értelmében fennáll az alábbi becslés:
		\marginnote{
			Véve \( g \) karakterisztikus alapját
			\begin{align*}
				g \cdot \chi_{A_n}
				&= \sum_{y \in \rng{g}} y {\cdot} \chi_{\{ g = y \}} {\cdot} \chi_{A_n} \\
				&= \sum_{y \in \rng{g}} y {\cdot} \chi_{\{ g = y \} \cap A_n}
			\end{align*}
			adódik. Továbbá elmondható, hogy
			\[
				\int \chi_B \dd{\mu} = \mu( B ) \quad (B \in \Omega).
			\]
		}
		\[
			f_n \geq c \cdot g \cdot \chi_{A_n} \qquad (n \in \N).
		\]
		Felhasználva a \( g \) karakterisztikus alapját, 
		valamint az integrál additivitását
		\[
			\sup_{n} \int f_n \dd{\mu} \geq
			\int f_n \dd{\mu} \geq 
			c \cdot \int g {\cdot} \chi_{A_n} \dd{\mu} = 
			c \cdot \sum_{y \in \rng{g}} y {\cdot} \mu\bigl( \{ y = g \} \cap A_n \bigr).
		\]
		Végül \eqref{eq:majoráló-lépcsősfüggvény-sorozat} alapján elmondható, 
		hogy az \( n \to \infty \) határátmenet után
		\[
			\sup_{n} \int f_n \dd{\mu} \geq 
			c \cdot \sum_{y \in \rng{g}} y {\cdot} \mu\bigl( \{ y = g \} \bigr) =
			c \cdot \int g \dd{\mu}.
		\]
	\end{proof}
	\newpage
	\begin{note}
		Mivel a fenti tételben szereplő \( (f_n) \) függvénysorozat monoton nő, így
		\[
			\sup_{n \in \N} f_n = \lim_{n \to \infty} f_n.
		\]
		Továbbá az integrál monotonitás miatt elmondható, 
		hogy az \( \bigl( \int f_n \dd{\mu} \bigr) \) sorozat is monoton nő,
		ezért létezik a soron következő határérték
		\[
			\sup_{n} \int f_n \dd{\mu} = 
			\lim_{n \to \infty} \int f_n \dd{\mu}.
		\]
		Tehát valójában az integrálsorozat határértékére adtunk egy alsó becslést. \qed 
	\end{note}
	
	\begin{theo*}
		Az előző tétel alapján elmondható, hogy tetszőleges
		\[
			f_n, g_n \in \StepFuncPos, \quad 
			f_n \leq f_{n + 1}, \quad 
			g_n \leq g_{n + 1} \qquad (n \in \N)
		\]
		függvénysorozatot, amelyek a határértéke azonos, azaz
		\[
			\sup_{n \in \N} f_n = 
			\lim_{n \to \infty} f_n = 
			\lim_{n \to \infty} g_n = 
			\sup_{n \in \N} g_n.
		\]
		Ekkor a hozzájuk tartozó integrálsorozatok határértékei is megegyeznek, vagyis
		\[
		%	\sup_{n} \int f_n \dd{\mu} = 
			\lim_{n \to \infty} \int f_n \dd{\mu} = 
			\lim_{n \to \infty} \int g_n \dd{\mu}.
		%	\sup_{n} \int g_n \dd{\mu}.
		\]
	\end{theo*}
	\begin{proof*}
		Ugyanis egy rögzített \( m \in \N \) index esetén
		\[
			g_m \leq \sup_{n \in \N} g_n = \sup_{n \in \N} f_n.
		\]
		Ezért \aref{th:kiterjesztés-függvénysorozattal} tétel felhasználásával
		\[
			\int g_m \dd{\mu} \leq \sup_{n} \int f_n \dd{\mu}
			\qquad \Longrightarrow \qquad
			\sup_{m} \int g_m \dd{\mu} \leq \sup_{n} \int f_n \dd{\mu}.
		\]
		Ugyanezen oknál fogva (lásd \( g_m \longleftrightarrow f_n \) szerepcsere) adódik, hogy
		\[
			\sup_{m} \int g_m \dd{\mu} \geq \sup_{n} \int f_n \dd{\mu}
			\qquad \Longrightarrow \qquad
			\boxed{\sup_{m} \int g_m \dd{\mu} = \sup_{n} \int f_n \dd{\mu}}.
		\]
		\qed
	\end{proof*}
	
	\noindent
	Mindez azt jelenti, hogy egy monoton növekedő \( \StepFuncPos \)-beli függvényekhez tartozó 
	integrálsorozat határértéke független magától a függvénysorozat megválasztásától. 
	Egyedül az számít, hogy milyen határfüggvényhez konvergál a sorozat.
	Ezért érdemes kitüntetett szereppel felruházni az előbb határfüggvényeket.
	
	\begin{definition}{\( \StepFuncPlus \)-függvények integrálja}{}
		Legyen
		\[
			\StepFuncPlus \coloneq
			\setc[\big]{ \func{f}{X}{\overline{\R}} }
			     { f = \lim(f_n), \text{ ahol } (f_n) \nearrow, \ \StepFuncPos \text{-beli}}.
		\]
		Ha \( f \in \StepFuncPlus \) és egy \( \func{(f_n)}{\N}{\StepFuncPos} \) sorozattal,
		\( f = \lim(f_n) \), akkor legyen
		\[
			\int f \dd{\mu} \coloneq 
			\lim_{n \to \infty} \int f_n \dd{\mu}
		\]
		az \( f \) függvény \( \mu \) mérték szerinti \emph{integrálja}.
	\end{definition}
	\newpage
	
	\begin{notes}
		\item 
		Nyilvánvaló, hogy \( \StepFuncPos \subset \StepFuncPlus \),
		valamint az \( \StepFuncPos \)-beli függvények integrálja megegyezik az 
		\( \StepFuncPlus \)-beli függvények integráljával.
		
		Ugyanis, ha \( f \in \StepFuncPos \), akkor az \( f_n \coloneq f \ (n \in \N) \)
		függvénysorozat monoton nő, valamint a határfüggvénye
		\[
			\lim(f_n) = f \in \StepFuncPos
			\qquad \Longrightarrow \qquad
			\StepFuncPos \subset \StepFuncPlus.
		\]
		Következésképpen az integrálja
		\[
		%	\StepFuncPos \text{-beli} \longrightarrow
			\underset{
				\substack{
					\mathclap{\big\uparrow} \\
					\mathclap{\StepFuncPlus \text{-beli}}
				}
			}{\int f \dd{\mu}} = 
			\lim_{n \to \infty} \int f_n \dd{\mu} =
			\lim_{n \to \infty} \int f \dd{\mu} =
			\underset{
				\substack{
					\mathclap{\big\uparrow} \\
					\mathclap{\StepFuncPos \text{-beli}}
				}
			}{\int f \dd{\mu}}.
		%	\longleftarrow \StepFuncPlus \text{-beli}.
		\]
	\end{notes}
	
	\begin{theorem}{Az \( \StepFuncPlus \) szerkezete}{}
		Az \( \func{f}{X}{[0, +\infty]} \) függvény pontosan akkor eleme 
		\( \StepFuncPlus \)-nak, 
		ha mérhető:
		\[
			\StepFuncPlus =
			\setc[\big]{ \func{f}{X}{\overline{ \R }} }{ \text{az } f \geq 0 \text{ mérhető}}
		\]
	\end{theorem}
	
	\begin{theorem}{Az \( \StepFuncPlus \)-beli integrál alaptulajdonságai}{}
		Tekintsük az \( f, g \in \StepFuncPlus \) függvényeket és az \( \alpha \geq 0 \) számot. 
		\marginnote{
			\begin{stat*}
				Az tételben foglalt jelölésekkel
				\[
					f + g \quad \text{és} \quad \alpha \cdot f
				\]
				szintén \( \StepFuncPlus \)-beli függvény.
			\end{stat*}
		}[-1.7\baselineskip]
		Ekkor
		\begin{enumerate}
			\item \( \displaystyle \int (\alpha \cdot f) \dd{\mu} = \alpha \cdot \int f \dd{\mu} \);
			\item \( \displaystyle \int (f + g) \dd{\mu} = \int f \dd{\mu} + \int g \dd{\mu} \);
			\item \( \displaystyle f \leq g \quad \Longrightarrow \quad \int f \dd{\mu} \leq \int g \dd{\mu} \);
		\end{enumerate}
	\end{theorem}
	
\end{document} 