\documentclass[
%parspace, % Add vertical space between paragraphs
%noindent, % No indentation of first lines in each paragraph
%nohyp,	   % No hyphenation of words
%twoside,  % Double sided format
%draft,    % Quicker draft compilation without rendering images
%final,    % Set final to hide todos
]{elteikthesis}[2024/04/26]


% The minted package is also supported for source highlighting
% See elteikthesis_minted.tex for example
%\usepackage[newfloat]{minted}
\usepackage{enumitem}

% Document's metadata
\title{Mérték, integrál, valószínűség} % title
\subtitle{12. Vizsgatétel}


% The document
\begin{document}
	
	% Set document language
	\documentlang{hungarian}
	
	\section{Emlékeztető}
	
	Emlékezzünk arra, hogy egy \( A \subseteq \R \) halmazt \emph{Borel-halmaznak} nevezünk, ha
	\[
		A \in \Omega_1 \coloneq \Omega( \mathcal{I} ) = \Omega( \mathbf{I} ).
	\]
	Az itt szereplő \( \mathbf{I} \) halmazrendszer az üres halmazt, 
	valamint az \( \R \) balról zárt, jobbról nyílt intervallumait tartalmazza,
	\( \mathcal{I} \) pedig az \( \mathbf{I} \) félgyűrű által generált gyűrű, azaz
	\[
		\mathbf{I} \coloneq 
		\setc[\Big]{ \emptyset, [a, b) \subseteq \R }{ a, b \in \R, \ a < b }, \qquad
		\mathcal{I} \coloneq
		\mathcal{G}( \mathbf{I} ).
	\]
	\begin{definition}{Borel--mérhető függvény}{}	
		Legyen \( (X, \Omega) \) mérhető tér, valamint \( \func{f}{X}{\R} \) egy függvény.\\[3pt]
		Azt mondjuk, hogy az \( f \) függvény \emph{mérhető} (vagy \emph{Borel--mérhető}), ha
		\[
		f^{-1}[A] \coloneq
		\setc[\big]{ x \in X }{ f(x) \in A } \in \Omega
		\qquad \bigl( A \in \Omega_1 \bigr).
		\]
	\end{definition}
	
	\begin{definition}{Lépcsősfüggvény}{}
		Legyen \( (X, \Omega) \) egy mérhető tér, valamint \( \func{f}{X}{\R} \) egy függvény.\\[3pt]
		Azt mondjuk, hogy \( f \) egy \emph{lépcsősfüggvény}, 
		ha mérhető és \( \rng{f} \) véges.
	\end{definition}
	
	Egy \( f \) leképezés pontosan akkor lépcsősfüggvény, ha kifejezhető az
	\[
		f = \sum_{y \in \rng{f}} y {\cdot} \chi_{\{ f = y \}}
	\]
	úgynevezett \emph{kanonikus alakban}. Továbbá bevezettük az alábbi osztályokat
	\[
		\StepFunc \coloneq \setc[\big]{ \func{f}{X}{\R} }{ f \text{ lépcsős}}, \qquad
		\StepFuncPlus \coloneq \setc[\big]{ f \in \StepFunc }{ f \geq 0 }. 
	\]
	
	\begin{definition}{Nemnegatív lépcsősfüggvény integrálja}{}
		Egy \( f \in \StepFuncPlus \) függvény (\( \mu \) mérték szerinti) \emph{integrálja} alatt az
		\[
		\int f \dd{\mu} \coloneq
		\sum_{y \in \rng{f}} y {\cdot} \mu \bigl( \{ f = y \} \bigr)
		\]
		nemnegatív számot (vagy a \( +\infty \)-t) értjük.
	\end{definition}
	
	\begin{theorem}{Az integrál alaptulajdonságai}{}
		Tekintsük az \( f, g \in \StepFuncPlus \) függvényeket és az \( \alpha \geq 0 \) számot. 
		\marginnote{
			\begin{stat*}
				Az tételben foglalt jelölésekkel
				\[
					f + g \quad \text{és} \quad \alpha \cdot f
				\]
				szintén \( \StepFuncPlus \)-beli függvény.
			\end{stat*}
		}
		Ekkor
		\begin{enumerate}
			\item \( \displaystyle \int (\alpha \cdot f) \dd{\mu} = \alpha \cdot \int f \dd{\mu} \);
			\item \( \displaystyle \int (f + g) \dd{\mu} = \int f \dd{\mu} + \int g \dd{\mu} \);
			\item \( \displaystyle f \leq g \quad \Longrightarrow \quad \int f \dd{\mu} \leq \int g \dd{\mu} \);
		\end{enumerate}
	\end{theorem}
	
	\newpage
	\section{Az integrál kiterjesztése}
	
	\begin{theorem}{}{}
		Legyen adott egy \( \StepFuncPlus \)-beli, monoton növekedő függvénysorozat:
		\[
			f_n \in \StepFuncPlus, \quad f_n \leq f_{n + 1} \qquad (n \in \N).
		\]
		Ha valamilyen \( g \in \StepFuncPlus \) függvény
		esetén \( g \leq \sup\limits_{n \in \N} f_n \). Ekkor
		\[
			\int g \dd{\mu} \leq \sup_{n} \int f_n \dd{\mu}.
		\]
	\end{theorem}
	\begin{proof}
		Amennyiben \( 0 \leq c < 1 \) egy rögzített konstans, akkor az
		\marginnote{
			Ezen halmazsorozatra az igaz, hogy
			\[
				A_n \in \Omega, \quad (A_n) \nearrow X.
			\]
		}
		\[
			A_n \coloneq \bigl\{ f_n \geq c \cdot g \bigr\} \quad (n \in \N)
		\]
		nívóhalmazok \( \Omega \)-ban vannak, és monoton növekvő módon tartanak \( X \)-hez.
		
		\begin{enumerate}
			\item Mivel az \( (f_n) \) sorozat monoton nő, ezért az \( (A_n) \) monoton bővül. \checkmark
			\item Ha \( x \in X \) tetszőleges, akkor
			\marginnote{
				Itt lehet, hogy \( g(x) = 0 \) vagy \( g(x) > 0 \).
			}
			\[
				c \cdot g(x) \leq g(x) \leq \sup_{n \in \N} f_n(x).
			\]
			Ebből kifolyólag, valamint az \( (f_n) \) sorozat monoton növekedés miatt
			\[
				\exists n \in \N \, \colon \quad
				c \cdot g(x) \leq f_n(x)
				\quad \Longrightarrow \quad
				x \in A_n.
			\]
			Tehát az \( (A_n) \) halmazsorozat valóban \( X \)-hez tart. \checkmark
		\end{enumerate}
			
%		Ugyanis, ha veszünk egy tetszőleges \( x \in X \) pontot, akkor két eset lehetséges.
%		\begin{enumerate}
%			\item Ha \( g(x) = 0 \), akkor
%			\[
%				0 = c \cdot g(x) \leq f_n(x)
%				\quad \Longrightarrow \quad
%				x \in A_n \qquad (n \in \N).
%			\]
%			
%			\item Ha \( g(x) > 0 \), akkor van olyan \( N \in \N \) küszöbindex, hogy
%			\marginnote{
%				Ugyanis az \( (f_n) \) sorozat monoton nő és
%				\[
%					c \cdot g(x) < g(x) \leq \sup_{n \in \N} f_n(x).
%				\]
%				Mivel az \( (f_n) \) sorozat monoton nő, ezért 
%			}
%			\[
%				c \cdot g(x) \leq f_n(x)
%				\quad \Longrightarrow \quad
%				x \in A_n \qquad (N < n \in \N).
%			\]
%		\end{enumerate}
		
		Mivel \( \mu \) mérték, valamint bármely \( Z \in \Omega \) esetén 
		az \( (A_n \cap Z) \nearrow Z \), ezért
		\[
			\mu(A_n \cap Z) \longrightarrow \mu(Z) \quad (n \to \infty).
		\]
		
		\noindent\rule{\linewidth}{0.4pt}\\
		
		Tekintsük az soron következő \( \StepFuncPlus \)-beli függvényeket
		\[
			f_n \geq c \cdot g \cdot \chi_{A_n} \qquad (n \in \N).
		\]
		Ekkor
		\marginnote{
			Ugyanis
			\begin{align*}
				\int g {\cdot} \chi_{A_n} \dd{\mu}
				&= \int \sum_{y \in \rng{g}} y {\cdot} \chi_{\{ g = y \}} {\cdot} \chi_{A_n} \dd{\mu} \\
			%	&= \int \sum_{y \in \rng{g}} y {\cdot} \chi_{\{ g = y \} \cap A_n} \dd{\mu} \\
				&= \sum_{y \in \rng{g}} \int y {\cdot} \chi_{\{ g = y \} \cap A_n} \dd{\mu} \\
				&= \sum_{y \in \rng{g}} y {\cdot} \mu \bigl( \{ g = y \} \cap A_n \bigr) \\
			\end{align*}
		}
		\[
			\alpha \coloneq 
			\sup_{n} \int f_n \dd{\mu} \geq
			\int f_n \dd{\mu} \geq 
			c \cdot \int g {\cdot} \chi_{A_n} \dd{\mu} = 
			c \cdot \sum_{y \in \rng{g}} y {\cdot} \mu\bigl( \{ y = g \} \cap A_n \bigr).
		\]
		Ekkor az \( n \to \infty \) határátmenet után
		\[
			\alpha \geq 
			c \cdot \sum_{y \in \rng{g}} y {\cdot} \mu\bigl( \{ y = g \} \bigr) =
			c \cdot \int g \dd{\mu}.
			\qquad \Longrightarrow \qquad
			\alpha \geq \int g \dd{\mu}.
		\]
	\end{proof}
	
\end{document} 