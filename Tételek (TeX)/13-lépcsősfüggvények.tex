\documentclass[
%parspace, % Add vertical space between paragraphs
%noindent, % No indentation of first lines in each paragraph
%nohyp,	   % No hyphenation of words
%twoside,  % Double sided format
%draft,    % Quicker draft compilation without rendering images
%final,    % Set final to hide todos
]{elteikthesis}[2024/04/26]


% The minted package is also supported for source highlighting
% See elteikthesis_minted.tex for example
%\usepackage[newfloat]{minted}
\usepackage{enumitem}

% Document's metadata
\title{Mérték, integrál, valószínűség} % title
\subtitle{12. Vizsgatétel}


% The document
\begin{document}
	
	% Set document language
	\documentlang{hungarian}
	
	\section{Emlékeztető}
	
	Emlékezzünk arra, hogy egy \( A \subseteq \R \) halmazt \emph{Borel-halmaznak} nevezünk, ha
	\[
		A \in \Omega_1 \coloneq \Omega( \mathcal{I} ) = \Omega( \mathbf{I} ).
	\]
	Az itt szereplő \( \mathbf{I} \) halmazrendszer az üres halmazt, 
	valamint az \( \R \) balról zárt, jobbról nyílt intervallumait tartalmazza,
	\( \mathcal{I} \) pedig az \( \mathbf{I} \) félgyűrű által generált gyűrű, azaz
	\[
		\mathbf{I} \coloneq 
		\setc[\Big]{ \emptyset, [a, b) \subseteq \R }{ a, b \in \R, \ a < b }, \qquad
		\mathcal{I} \coloneq
		\mathcal{G}( \mathbf{I} ).
	\]
	\begin{definition}{Borel--mérhető halmaz}{}
		Egy \( A \subseteq \overline{\R} \) halmaz 
		kibővített értelemben \emph{Borel--mérhető}, ha
		\[
			A \cap \R \in \Omega_1.
		\]
		Legyen az ilyen tulajdonságú halmazoknak a rendszere \( \overline{\Omega}_1 \).
	\end{definition}
	
	\begin{note}
		Tehát egy \( A \subseteq \overline{ \R } \) halmaz pontosan akkor Borel--mérhető, ha
		\[
			A = B \cup C
		\]
		módon bontható fel, ahol \( B \in \Omega_1 \) és 
		\( C \in \bigl\{ \emptyset, \{ -\infty \}, \{ +\infty \}, \{ -\infty, +\infty \} \bigr\} \).
	\end{note}
	
	\begin{definition}{Borel--mérhető függvény}{}
		%		Legyen \( X \neq \emptyset \) egy adott halmaz, 
		%		\( \Omega \subseteq \powerset{X} \) szigma-algebra.
		%		
		Legyen \( (X, \Omega) \) mérhető tér, 
		valamint \( \func{f}{X}{\overline{\R}} \) egy függvény.\\[3pt]
		Azt mondjuk, hogy az \( f \) függvény \emph{mérhető} (vagy \emph{Borel--mérhető}), ha
		\[
		f^{-1}[A] \coloneq
		\setc[\big]{ x \in X }{ f(x) \in A } \in \Omega
		\qquad \bigl( A \in \overline{\Omega}_1 \bigr).
		\]
	\end{definition}
	\begin{note}
		Szóban, az \( f \) függvény pontosan akkor mérhető, 
		ha minden kibővített értelemben Borel-mérhető \( A \)
		halmaz \( f^{-1}[A] \) ősképe az \( \Omega \)-ban van.
	\end{note}
	
	\newpage
	\section{Lépcsősfüggvények}
	
	\begin{definition}{Lépcsősfüggvény}{}
		Legyen \( (X, \Omega, \mu) \) egy mértéktér, 
		valamint \( \func{f}{X}{\R} \) egy függvény.\\[3pt]
		Azt mondjuk, hogy \( f \) egy \emph{lépcsősfüggvény}, 
		ha mérhető és \( \rng{f} \) véges.
	\end{definition}
	
	Egy \( f \) leképezés pontosan akkor lépcsősfüggvény, ha
	\[
		f = \sum_{i=0}^{n} \alpha_i {\cdot} \chi_{A_i}
	\]
	módon írható fel, 
	valamilyen \( \alpha_0, \dots, \alpha_n \in \R \) 
	és \( A_0, \dots, A_n \in \Omega \) esetén.
	Speciálisan az
	\[
		f = \sum_{y \in \rng{f}} y {\cdot} \chi_{\{ f = y \}}
	\]
	alakot \emph{kanonikus előállításnak} nevezzük. Továbbá legyen
	\[
		\StepFunc \coloneq \setc[\big]{ \func{f}{X}{\R} }{ f \text{ lépcsős}}, \qquad
		\StepFuncPlus \coloneq \setc[\big]{ f \in \StepFunc }{ f \geq 0 }. 
	\]
	
	\begin{statement}{Lépcsősfüggvények alaptulajdonságai}{lépcsősfüggvény-műveletek}
		Legyenek \( f,g \in \StepFunc \) lépcsősfüggvények, valamint \( \alpha \in \R \). Ekkor az
		\[
			f + g, \ \alpha \cdot g, \ f \cdot g, \ \max\{ f, g \}, \ \min\{ f, g \}
		\]
		függvények valamennyien az \( \StepFunc \) osztályban vannak.
	\end{statement}
	
	\noindent 
	A továbbiakban nemnegatív lépcsősfüggvényeket fogunk tekinteni.
	Nyilvánvaló módon \aref{st:lépcsősfüggvény-műveletek} állítás ekkor is igaz marad, 
	midőn az \( \alpha \geq 0 \).
	
	\begin{definition}{Nemnegatív lépcsősfüggvény integrálja}{}
		Egy \( f \in \StepFuncPlus \) függvény (\( \mu \) mérték szerinti) \emph{integrálja} alatt az
		\[
			\int f \dd{\mu} \coloneq
			\sum_{y \in \rng{f}} y {\cdot} \mu \bigl( \{ f = y \} \bigr)
		\]
		nemnegatív számot (vagy a \( +\infty \)-t) értjük.
	\end{definition}
	
	\begin{theorem}{Az integrál alaptulajdonságai}{}
		Tekintsük az \( f, g \in \StepFuncPlus \) függvényeket és az \( \alpha \geq 0 \) számot. Ekkor
		\begin{enumerate}
			\item \( \displaystyle \int (\alpha \cdot f) \dd{\mu} = \alpha \cdot \int f \dd{\mu} \);
			\item \( \displaystyle \int (f + g) \dd{\mu} = \int f \dd{\mu} + \int g \dd{\mu} \);
			\item \( \displaystyle f \leq g \quad \Longrightarrow \quad \int f \dd{\mu} \leq \int g \dd{\mu} \);
		\end{enumerate}
	\end{theorem}

	
\end{document}