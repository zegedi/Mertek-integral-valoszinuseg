\documentclass[
%parspace, % Add vertical space between paragraphs
%noindent, % No indentation of first lines in each paragraph
%nohyp,	   % No hyphenation of words
%twoside,  % Double sided format
%draft,    % Quicker draft compilation without rendering images
%final,    % Set final to hide todos
]{elteikthesis}[2024/04/26]


% The minted package is also supported for source highlighting
% See elteikthesis_minted.tex for example
%\usepackage[newfloat]{minted}
\usepackage{enumitem}

% Document's metadata
\title{Mérték, integrál, valószínűség} % title
\subtitle{12. Vizsgatétel}


% The document
\begin{document}
	
	% Set document language
	\documentlang{hungarian}
	
	\section{Borel--mérhető leképezések}
	
	Emlékezzünk arra, hogy egy \( A \subseteq \R \) halmazt \emph{Borel-halmaznak} nevezünk, ha
	\[
		A \in \Omega_1 \coloneq \Omega( \mathcal{I} ) = \Omega( \mathbf{I} ).
	\]
	Az itt szereplő \( \mathbf{I} \) halmazrendszer az üres halmazt, 
	valamint az \( \R \) balról zárt, jobbról nyílt intervallumait tartalmazza, tehát
	\[
		\mathbf{I} \coloneq 
		\setc[\Big]{ \emptyset, [a, b) \subseteq \R }{ a, b \in \R, \ a < b }.
	\]
	Továbbá \( \mathcal{I} \) pedig az \( \mathbf{I} \) félgyűrű által generált gyűrű, vagyis
	\[
		\mathcal{I} \coloneq
		\mathcal{G}( \mathbf{I} ) = 
		\setc[\Bigg]{ \bigcup_{k=0}^{n} I_k }
		            { I_0, \dots, I_n \in \mathbf{I} \text{ páronként diszjunktak } (n \in \N) }.
	\]
	Vezessük be a \emph{kibővített valós számok} halmazát
	\[
		\overline{ \R } \coloneq \R \cup \{ -\infty, +\infty \}.
	\]
	\begin{definition}{Borel--mérhető halmaz}{}
		Egy \( A \subseteq \overline{\R} \) halmaz 
		kibővített értelemben \emph{Borel--mérhető}, ha
		\[
			A \cap \R \in \Omega_1.
		\]
		Legyen az ilyen tulajdonságú halmazoknak a rendszere \( \overline{\Omega}_1 \).
%		\[
%			\overline{\Omega}_1 \coloneq
%			\setc[\big]{ A \subseteq \overline{\R} }{ A \text{ kibővített Borel--halmaz}}
%		\]
	\end{definition}
	
	\begin{note}
		Világos, hogy minden \( A \subseteq \R \) Borel-mérhető halmaz egyben kibővített 
		értelemben is Borel-mérhető. 
		Továbbá valóban az említett fogalom kibővítéséről beszélhetünk, ugyanis az
		\[
			\{ -\infty \}, \{ +\infty \}, \{ -\infty, +\infty \} \subseteq \overline{ \R }
		\]
		halmazok minden .
		Egy \( A \subseteq \overline{ \R } \) halmaz pontosan akkor Borel--mérhető, ha
		\[
			A = B \cup C
		\]
		módon bontható fel, ahol \( B \in \Omega_1 \) és 
		\( C \in \bigl\{ \emptyset, \{ -\infty \}, \{ +\infty \}, \{ -\infty, +\infty \} \bigr\} \).
	\end{note}
	
	\begin{definition}{Borel--mérhető függvény}{}
		Legyen \( (X, \Omega) \) mérhető tér, 
		valamint \( \func{f}{X}{\overline{\R}} \) egy függvény.\\[6pt]
		Azt mondjuk, hogy az \( f \) függvény \emph{mérhető} (vagy \emph{Borel--mérhető}), ha
		\[
			f^{-1}[A] \coloneq
			\setc[\big]{ x \in X }{ f(x) \in A } \in \Omega
			\qquad \bigl( A \in \overline{\Omega}_1 \bigr).
		\]
	\end{definition}
%	\begin{note}
%		Szóban, az \( f \) függvény pontosan akkor mérhető, 
%		ha minden kibővített értelemben Borel-mérhető \( A \)
%		halmaz \( f^{-1}[A] \) ősképe az \( \Omega \)-ban van.
%	\end{note}

	\begin{example}
		Legyen \( (X, \Omega) \) mérhető tér, \( A \subseteq X \) egy halmaz. Ekkor
		\[
			\func{\chi_A}{X}{\R}, \qquad
			\chi_A(x) \coloneq 
			\left\{
			\begin{aligned}
				1, & \quad \text{ha } x \in A, \\
				0, & \quad \text{ha } x \notin A, \\
			\end{aligned}
			\right.
		\]
		az \( A \) halmaz \emph{karakterisztikus függvénye}.
		Ekkor \( \chi_A \) mérhető \( \iff \) \( A \in \Omega \). \qed
	\end{example}
	
	\begin{theorem}{Mérhető függvények tulajdonságai}{}
		Legyen \( (X, \Omega) \) egy mérhető tér, 
		valamint \( \func{f, f_n, g}{X}{\overline{\R}} \ (n \in \N) \).
		\begin{enumerate}
			\item 
			Ha \( * \in \{ \geq, >, \leq, < \} \),
			akkor \( f \) mérhető \( \iff \) 
			\( \forall \alpha \in \R \colon \{ f * \alpha \} \in \Omega \).
			
			\item
			Ha \( f, g \) mérhető és \( * \in \{ \geq, >, \leq, <, =, \neq \} \), 
			akkor \( \{ f * g \} \in \Omega \).
			
			\item
			Ha \( f, g \) mérhető, akkor \( (f \cdot g) \) és \( \abs{f} \) is mérhető függvény.
			\marginnote{
				Innentől: \( 0 \cdot (\pm \infty) \coloneq (\pm \infty) \cdot 0 \coloneq 0 \).
			}
			
			\item
			Ha \( f, g \) mérhető és létezik az \( (f \pm g) \) függvény, akkor az is mérhető.
			\marginnote{
				Például, ha \( f,g \) véges, akkor ez teljesül.
			}
			
			\item
			Ha \( (f_n) \) mérhető függvényeknek a sorozata, akkor a
			\marginnote{
				Az itt szereplő függvények:
				\begin{align*}
					\limsup (f_n) 
					&\coloneq \lim_{n \to \infty} \biggl( \, \sup_{k \geq n} f_k \biggr), \\
					\liminf (f_n) 
					&\coloneq \lim_{n \to \infty} \biggl( \, \inf_{k \geq n} f_k \biggr).
				\end{align*}
			}
			\[
				\sup_{n \in \N}(f_n), \quad
				\inf_{n \in \N}(f_n), \quad
				\limsup(f_n), \quad
				\liminf(f_n)
			\]
			függvények is mérhetőek.
			
			\item
			Ha \( (f_n) \) mérhető függvényeknek a sorozata pontonként konvergál az
			\[
				f(x) \coloneq \lim_{n \to \infty} f_n(x) \qquad (x \in X)
			\]
			határfüggvényhez, akkor az \( f \) is mérhető.
		\end{enumerate}
	\end{theorem}
	
	\newpage
	
	\begin{theorem}{Jegorov--tétel}{}
		Legyen \( (X, \Omega, \mu) \) egy mértéktér, ahol \( \mu \) véges mérték,
		\( \func{f_n}{X}{\R} \ (n \in \N) \).\\[6pt]
		Ha \( (f_n) \) mérhető függvényeknek a sorozata pontonként konvergál az
		\[
			f(x) \coloneq \lim_{n \to \infty} f_n(x) \in \R \qquad (x \in X)
		\]
		határfüggvényhez, 
		akkor tetszőleges \( \varepsilon > 0 \) számhoz van olyan 
		\( X_\varepsilon \in \Omega \), hogy
		\begin{enumerate}[label=\alph*)]
			\item az \( (f_n) \) sorozat az \( X_\varepsilon \) halmazon 
			egyenletesen konvergál az \( f \)-hez;
			\item \( \mu( X \setminus X_\varepsilon ) < \varepsilon \).
		\end{enumerate}
	\end{theorem}
	\begin{proof}
		Tekintsük egy \( 1 \leq k \in \N \) index esetén az
		\[
			X_{n,k} \coloneq \bigcup_{i=n}^{\infty} \Bigl\{ \abs\big{f_i - f} \geq 1 / k \Bigr\}
			\quad \Longrightarrow \quad
			X_{n,k} \in \Omega \qquad (n \in \N)
		\]
		halmazsorozatot.
		Mivel az \( (f_n) \) függvénysorozat \( f \)-hez tart, 
		ezért az említett halmazok monoton szűkülő módon tartanak az üres halmazhoz, azaz
		\marginnote{
			Ugyanis, indirekt tegyük fel, hogy
			\[
				\exists x \in X \,\colon \quad
				x \in \bigcap_{n=0}^{\infty} X_{n, k}.
			\]
			Ez csak akkor lehetséges, ha
			\[
				\abs\big{ f_i(x) - f(x) } \geq \frac{1}{k}
			\]
			végtelen sok \( i \in \N \) indexre igaz, tehát
			\[
				\abs\big{ f_i(x) - f(x) } \centernot{\longrightarrow} 0.
			\]
		}
		\[
			X_{n+1, k} \subseteq X_{n, k} \quad (n \in \N)
			\qquad \text{és} \qquad
			\emptyset = \bigcap_{n=0}^{\infty} X_{n, k}.
		\]
		Mivel feltettük, hogy \( \mu \) véges mérték, 
		ezért \( \mu(X_{n, k}) \longrightarrow 0 \ (n \to \infty) \). Vagyis
		\[
			\exists	n_k \in \N \colon \quad
			\mu(X_{n, k}) < \frac{\varepsilon}{2^{k + 1}}.
		\]
		Ez alapján tekintsük a következő halmazat:
		\[
			X_\varepsilon \coloneq
			X \setminus \Biggl( \, \bigcup_{k=1}^{\infty} X_{n_k, k} \Biggr).
		\]
		Mivel \( \Omega \) szigma-algebra, ezért \( X_\varepsilon \in \Omega \) .
		Továbbá minden \( x \in X_\varepsilon \) helyen
		\[
			\abs\big{ f_i(x) - f(x) } < \frac{1}{k} \qquad (n_k \leq i \in \N).
		\]
		Tehát az \( (f_n) \) sorozat egyenletesen konvergens \( X_\varepsilon \)-on.
		Végül
		\marginnote{
			Hiszen az itt szereplő mértani sorösszeg
			\[
				\sum_{k = 1}^{\infty} \frac{1}{2^k} = \frac{1/2}{1 - 1/2} = 1.
			\]
		}
		\[
			\mu( X \setminus X_\varepsilon ) \leq 
		%	\mu \Biggl( \, \bigcup_{k=1}^{\infty} \mu(X_{n_k}, k) \Biggr) \leq
			\sum_{k = 1}^{\infty} \mu(X_{n_k}, k) <
			\varepsilon \cdot \sum_{k = 1}^{\infty} \frac{1}{2^k} =
			\varepsilon.
		\]
	\end{proof}
	
\end{document}