\documentclass[
%parspace, % Add vertical space between paragraphs
%noindent, % No indentation of first lines in each paragraph
%nohyp,	   % No hyphenation of words
%twoside,  % Double sided format
%draft,    % Quicker draft compilation without rendering images
%final,    % Set final to hide todos
]{elteikthesis}[2024/04/26]


% The minted package is also supported for source highlighting
% See elteikthesis_minted.tex for example
%\usepackage[newfloat]{minted}
\usepackage{enumitem}

\makeatletter
\def\biggg{\bBigg@{3}}
\def\bigggm{\mathrel\biggg}
\def\bigggl{\mathopen\biggg}
\def\bigggr{\mathclose\biggg}
\def\Biggg{\bBigg@{3.5}}
\def\Bigggm{\mathrel\Biggg}
\def\Bigggl{\mathopen\Biggg}
\def\Bigggr{\mathclose\Biggg}
\makeatother

% Document's metadata
\title{Mérték, integrál, valószínűség} % title
\subtitle{\circled{6} Vizsgatétel}
\date{2024} % year of defense

% Author's metadata
\author{Egedi Viktor}
\degree{programtervező informatikus BSc}

% Superivsor(s)' metadata
\supervisor{Dr. Simon Péter} % internal supervisor's name
\affiliation{egyetemi tanár} % internal supervisor's affiliation

% University's metadata
\university{Eötvös Loránd Tudományegyetem} % university's name
\faculty{Informatikai Kar} % faculty's name
\department{Numerikus Analízis Tanszék} % department's name
\city{Budapest} % city
\logo{elte_cimer_szines} % logo

% The document
\begin{document}
	
	% Set document language
	\documentlang{hungarian}
	
	\section{Emlékeztető}
	
	Az (absztrakt) halmazok mérését (a mértéküknek az értelmezését) egy
	\marginnote{
		Ekkor
		\[
			\powerset{X} \coloneq \setc[\big]{A \text{ halmaz}}{A \subseteq X}
		\]
		az \( X \) úgynevezett hatványhalmaza.
	}
	\[
		\funcin{\varphi}{\powerset{X}}{[0, +\infty]}
	\]
	függvény segítségével végezzük, ahol \( X \) egy adott alaphalmaz. Ezen
	\begin{enumerate}
		\item \( \varphi \) \emph{(véges) additív}, ha
		\[
			\varphi \Biggl(\, \bigcup_{k=0}^n A_k \Biggr) = \sum_{k=0}^n \varphi(A_k)
		\]
		minden olyan \( A_0, \dots, A_n \in \dom{\varphi} \) páronként diszjunkt elemű halmazrendszerre fennáll, amelynek az egyesítésére
		\( A_0 \cup \cdots \cup A_n \in \dom{\varphi} \) teljesül;
		
		\item \( \varphi \) \emph{szigma-additív} 
		(röviden \emph{\( \boldsymbol{\sigma} \)-additív}), ha
		\[
			\varphi \Biggl(\, \bigcup_{n=0}^\infty A_n \Biggr) = 
			\sum_{n=0}^\infty \varphi(A_n)
		\]
		minden olyan \( (A_n) \) páronként diszjunkt tagokból álló 
		\( \dom{\varphi} \)-beli sorozatra igaz, amelynek az egyesítésére 
		\( \bigcup\limits_{n \in \N} \! A_n \in \dom{\varphi} \) teljesül.
	\end{enumerate}
	
	\begin{definition}{Mérték, kvázimérték, előmérték}{}
		Azt mondjuk, hogy a \( \funcin{\mu}{\powerset{X}}{[0, +\infty]} \) halmazfüggvény egy
		\begin{enumerate}
			\item \emph{mérték}, ha \( \dom{\mu} \) szigma-algebra, \( \mu(\emptyset) = 0 \), és a \( \mu \) szigma-additív;
			
			\item \emph{kvázimérték}, ha \( \dom{\mu} \) halmazgyűrű, 
			\( \mu(\emptyset) = 0 \), és a \( \mu \) szigma-additív;
			
			\item \emph{előmérték}, ha \( \dom{\mu} \) halmazgyűrű, 
			\( \mu(\emptyset) = 0 \), és a \( \mu \) additív.
		\end{enumerate}
	\end{definition}
	
	\newpage
	
	\begin{theorem}{Az előmérték tulajdonságai}{}
		Legyen \( \mu \) előmérték a \( \ring{G} \subseteq \powerset{X} \) gyűrűn,
		\marginnote{Tehát \( \func{\mu}{\ring{G}}{[0, +\infty]} \) egy előmérték.}
		továbbá \( A, B, A_n \in \ring{G} \ (n \in \N) \).
		
		\begin{enumerate}
			\item\label{th:előmérték-monoton}
			\( \mu \) monoton, azaz \( B \subseteq A \) esetén \( \mu(B) \leq \mu(A) \).
			
			\item\label{th:előmérték-szita-formula}
			\( \mu(A \cup B) + \mu(A \cap B) = \mu(A) + \mu(B) \).
			
			\item\label{th:előmérték-különbség-mértéke}
			Ha \( \mu(B) \) véges és \( B \subseteq A \),
			akkor \( \mu(A \setminus B) = \mu(A) - \mu(B) \).
			
			\item\label{th:előmérték-véges-szubadditív}
			Minden \( n \in \N \) indexre 
			\( \mu \biggl(\, \bigcup\limits_{k=0}^n \! A_k \biggr) \leq \sum\limits_{k=0}^n \mu( A_k ) \).
			
			\item\label{th:előmérték-szubadditívitás-kiterjesztése}
			Ha az \( (A_n) \) tagjai páronként diszjunktak
			és \( \bigcup\limits_{n=0}^{\infty} \! A_n \in \ring{G} \), akkor
			\[
			\mu \Biggl(\, \bigcup\limits_{n=0}^{\infty} \! A_n \Biggr) \leq 
			\sum\limits_{n=0}^{\infty} \mu( A_n ).
			\]
		\end{enumerate}
	\end{theorem}
	\begin{proof}\,
		\begin{enumerate}
			\item Mivel \( A = B \cup (A \setminus B) \) diszjunkt felbontás és \( \mu \) nemnegatív, ezért
			\[
			\mu(A) = \mu(B) + \mu(A \setminus B) \geq \mu(B). 
			\tag{\( * \)}\label{eq:előmérték-monoton}
			\]
			
			\item Ha \( \mu(B) \) véges, akkor \eqref{eq:előmérték-monoton} átrendezésével adódik a belátandó állítás.
			
			\item Két esetet különböztetünk meg.
			\begin{enumerate}
				\item Amennyiben \( \mu(A \cap B) = +\infty \), 
				akkor az \( A \cap B \subseteq A, B \subseteq A \cup B \) tartalmazások,
				valamint a \hyperref[th:előmérték-monoton]{\( \mu \) monotonitása} alapján
				\[
				\mu(A \cap B) = \mu(A \cup B) = \mu(A) = \mu(B) = +\infty.\ \checkmark
				\]
				
				\item Ha most \( \mu(A \cap B) \) véges, 
				akkor az \( A \cup B = A \cup \bigl( B \setminus (A \cap B) \bigr) \) diszjunkt felbontás és \( \mu \) additivitása miatt
				\[
				\mu(A \cup B) = 
				\mu(A) + \mu \bigl( B \setminus (A \cap B) \bigr) \overset{\ref{th:előmérték-különbség-mértéke}}{=} 
				\mu(A) + \mu(B) - \mu( A \cap B).\ \checkmark
				\]
			\end{enumerate}
			
			\item Az állítás teljes indukcióval igazolható, 
			felhasználva \ref{th:előmérték-szita-formula}-t.
			
			\item Mivel \( A_0, \dots, A_n \ (n \in \N) \) páronként diszjunktak 
			és \( \mu \) additív, ezért
			\[
			\sum_{n=0}^{\infty} \mu( A_n ) =
			\lim_{n \to \infty} \sum_{k=0}^{n} \mu( A_k ) =
			\lim_{n \to \infty} \mu \Biggl( \, \bigcup_{k=0}^n \! A_k \Biggr).
			\]
			Ugyanakkor az
			\( \bigcup\limits_{k=0}^n \! A_k \subseteq \bigcup\limits_{k=0}^{\infty} \! A_k \)
			tartalmazás és \hyperref[th:előmérték-monoton]{\( \mu \) monotonitása} miatt
			\[
			\sum_{n=0}^{\infty} \mu( A_n ) =
			\lim_{n \to \infty} \mu \Biggl( \, \bigcup_{k=0}^n \! A_k \Biggr) \leq
			\lim_{n \to \infty} \mu \Biggl( \, \bigcup_{k=0}^{\infty} \! A_k \Biggr) =
			\mu \Biggl( \, \bigcup_{n=0}^{\infty} \! A_n \Biggr).
			\]
		\end{enumerate}
	\end{proof}
	
	\begin{theorem}{Az előmérték és a kvázimérték kapcsolata}{előmérték-kvázimérték-kapcsolata}
		Legyen \( \mu \) egy előmérték a \( \ring{G} \subseteq \powerset{X} \) gyűrűn,
		\marginnote{Vagyis \( \func{\mu}{\ring{G}}{[0, +\infty]} \) alakú függvény.}
		és vegyük az alábbi állításokat.
		%
		\begin{enumerate}[label=\alph*)]
			\item\label{eq:kvázimérték-jellemzése-01} A \( \mu \) kvázimérték.
			
			\item\label{eq:kvázimérték-jellemzése-02}
			\marginnote{
				\Az{\ref{eq:kvázimérték-jellemzése-02}} állítást röviden úgy mondjuk, hogy
				\begin{center}
					\textit{\textbf{``a \( \boldsymbol{\mu} \) alulról félig folytonos''}.}
				\end{center}
			}
			Minden \( \ring{G} \)-beli \( A_n \subseteq A_{n + 1} \ (n \in \N) \) 
			monoton bővülő halmazsorozatra
			\[
				A \coloneq \bigcup_{n=0}^{\infty} A_n \in \ring{G}
				\qquad \Longrightarrow \qquad
				\mu(A) = \lim_{n \to \infty} \mu(A_n).
			\]
			
			\item\label{eq:kvázimérték-jellemzése-03}
			\marginnote{
				\Az{\ref{eq:kvázimérték-jellemzése-03}} állítást röviden úgy mondjuk, hogy
				\begin{center}
					\textit{\textbf{``a \( \boldsymbol{\mu} \) felülről félig folytonos''}.}
				\end{center}
			}
			Minden \( \ring{G} \)-beli \( B_{n + 1} \subseteq B_n \ (n \in \N) \)
			halmazsorozatra, ha \( \mu(B_n) < +\infty \)
			\[
				B \coloneq \bigcap_{n=0}^{\infty} B_n \in \ring{G}
				\qquad \Longrightarrow \qquad
				\mu(B) = \lim_{n \to \infty} \mu(B_n).
			\]
			
			\item\phantomsection\label{eq:kvázimérték-jellemzése-04} 
			Minden \( \ring{G} \)-beli \( C_{n + 1} \subseteq C_n \ (n \in \N) \)
			halmazsorozatra, ha \( \mu(C_n) < +\infty \)
			\[
			\emptyset = \bigcap_{n=0}^{\infty} C_n \in \ring{G}
			\qquad \Longrightarrow \qquad
			\mu(\emptyset) = \lim_{n \to \infty} \mu(C_n) = 0.
			\]
		\end{enumerate}
%		\begin{alignat*}{4}
%				A &\coloneq \bigcup_{n=0}^{\infty} A_n \in \ring{G}
%				\qquad &&\Longrightarrow \qquad
%				& \mu(A) &= \lim_{n \to \infty} \mu(A_n).
%			%
%				\intertext{%
%						\item\label{eq:kvázimérték-jellemzése-03} 
%						Minden \( \ring{G} \)-beli \( B_{n + 1} \subseteq B_n \ (n \in \N) \)
%						halmazsorozatra, ha \( \mu(B_n) < +\infty \)}
%			%
%				B &\coloneq \bigcap_{n=0}^{\infty} B_n \in \ring{G}
%				\qquad &&\Longrightarrow \qquad
%				& \mu(B) &= \lim_{n \to \infty} \mu(B_n).
%			%
%				\intertext{%
%						\item\phantomsection\label{eq:kvázimérték-jellemzése-04} 
%						Minden \( \ring{G} \)-beli \( C_{n + 1} \subseteq C_n \ (n \in \N) \)
%						halmazsorozatra, ha \( \mu(C_n) < +\infty \)}
%			%
%				\emptyset &= \bigcap_{n=0}^{\infty} C_n \in \ring{G}
%				\qquad &&\Longrightarrow \qquad
%				& \mu(\emptyset) &= \lim_{n \to \infty} \mu(C_n) = 0.
%			\end{alignat*}
		Ekkor
		\begin{enumerate}
			\item 
			\ref{eq:kvázimérték-jellemzése-01} \( \Longleftrightarrow \)
			\ref{eq:kvázimérték-jellemzése-02} \( \Longrightarrow \)
			\ref{eq:kvázimérték-jellemzése-03} \( \Longleftrightarrow \)
			\ref{eq:kvázimérték-jellemzése-04};
			\marginnote{
				\textcolor{red}{\textbf{Figyelem!}} Ilyenkor
				\ref{eq:kvázimérték-jellemzése-02} \( \centernot\Longleftarrow \)
				\ref{eq:kvázimérték-jellemzése-03}.
			}
			
			\item ha \( \mu \) véges, akkor még 
			\ref{eq:kvázimérték-jellemzése-02} \( \Longleftrightarrow \)
			\ref{eq:kvázimérték-jellemzése-03} is fennáll.
			\marginnote{
			%	Tehát \( \mu(Z) < +\infty \ (Z \in \ring{G}) \).
				Tehát minden \( Z \in \ring{G} \) esetén \( \mu(Z) \) véges.
			}
		\end{enumerate}
	\end{theorem}
	\begin{proof} \,\\[6pt]
		
		\fbox{\ref{eq:kvázimérték-jellemzése-01} \( \Longrightarrow \)
			  \ref{eq:kvázimérték-jellemzése-02}}
		%
		Tekintsük az \( A \) ``határhalmaznak'' az
		\marginnote{
			Tehát az előbbi halmazsorozat elemei
			\[
				D_0 \coloneq A_0, \ D_n \coloneq A_n \setminus A_{n - 1},
			\]
			ha \( n \in \posN \). Ekkor nyilvánvaló módon
			\[
				A_n = \bigcup_{k=0}^n D_k \quad (n \in \N)
				\tag{\( * \)}\label{eq:A_n-előállítása}
			\]
			is teljesül, hiszen az \( (A_n) \) halmazsorozat monoton bővül.
%			\begin{align*}
%				\mu(A)
%				&= \mu \Biggl( \, \bigcup_{n=0}^{\infty} D_n \Biggr)
%				 = \sum_{n=0}^{\infty} \mu( D_n ) \\
%				&= \lim_{n \to \infty} \sum_{k=0}^{n} \mu( D_k ) \\
%				&= \lim_{n \to \infty} \mu \Biggl( \, \bigcup_{k=0}^n D_k \Biggr) \\
%				&= \lim_{n \to \infty} \mu( A_n ) \\
%			\end{align*}
		}
		\[
			A = A_0 \cup (A_1 \setminus A_0) \cup (A_2 \setminus A_1) \cup \cdots
		\]
		páronként diszjunkt halmazokból álló felbontását. Ekkor
		%
		\begin{align*}
			\mu(A)
			&= \mu(A_0) + \mu(A_1 \setminus A_0) + \mu(A_2 \setminus A_1) + \cdots \\
			&= \lim_{n \to \infty} \Bigl( 
			\mu(A_0) + \mu(A_1 \setminus A_0) + \cdots + \mu(A_n \setminus A_{n-1})
			\Bigr)\\
			&= \lim_{n \to \infty} \mu \bigl( 
			A_0 \cup (A_1 \setminus A_0) \cup \cdots \cup (A_n \setminus A_{n-1})
			\bigr) \\
			&= \lim_{n \to \infty} \mu (A_n).
		\end{align*}
		Kihasználtuk, hogy \( \mu \) szigma- és véges additív, 
		valamint a \eqref{eq:A_n-előállítása} egyenlőséget.
		
		\noindent\rule{\linewidth}{0.4pt}\\
		
		\fbox{\ref{eq:kvázimérték-jellemzése-02} \( \Longrightarrow \)
			  \ref{eq:kvázimérték-jellemzése-01}}
		%
		Azt kell igazolni, hogy \( \mu \) szigma-additív. Legyen ehhez
		\[
			X_n \in \ring{G}, \quad
			A \coloneq \bigcup_{k=0}^{\infty} X_k \in \ring{G} \qquad \text{és} \qquad
			A_n \coloneq \bigcup_{k=0}^{n} X_n \qquad (n \in \N),
		\]
		ahol az \( (X_n) \) halmazsorozat tagjai páronként diszjunktak. 
		Ekkor az \( (A_n) \) sorozat tagjai monoton bővülő módon tartanak az \( A \)-hoz,
		ezért
		\marginnote{
			Kihasználjuk, hogy \( \mu \) addititása miatt
			\[
				\mu(A_n) = 
				\mu \Biggl( \, \bigcup_{k=0}^n X_n \Biggr) \! =
				\sum_{k=0}^{n} \mu( X_k ).
				\tag{\( ** \)}\label{eq:mu-an-additív}
			\]
		}
		\[
			\mu \Biggl( \, \bigcup_{n=0}^{\infty} X_n \Biggr) =
			\mu (A) \overset{\text{\ref{eq:kvázimérték-jellemzése-02}}}{=}
			\lim_{n \to \infty} \mu( A_n ) \overset{\eqref{eq:mu-an-additív}}{=}
			\lim_{n \to \infty} \sum_{k=0}^{n} \mu( X_k ) =
			\sum_{k=0}^{\infty} \mu( X_k ).
		\]
		
		\newpage
%		\noindent\rule{\linewidth}{0.4pt}\\

		A többi állítás bizonyítása során gyakran alkalmazzuk az alábbi észrevételt.
		
		\begin{lem*}
			Amennyiben \( Z \subseteq W \) és \( \mu(W) \) véges,
			\marginnote{Természetesen \( Z, W \in \ring{G} \) halmazok.}
			akkor \( \mu(Z) \) is véges, és ezért
			\[
				\mu(W \setminus Z) = \mu(W) - \mu(Z).
			\]
		\end{lem*}
		\begin{proof*}
			Mivel a \( \mu \) monoton, ezért \( \mu(Z) \leq \mu(W) \), 
			ahonnan \( \mu(Z) \) véges mivolta következik.
			Továbbá a második állítás minden előmértékre igaz.
		\end{proof*}
		
		\fbox{\ref{eq:kvázimérték-jellemzése-02} \( \Longrightarrow \)
			  \ref{eq:kvázimérték-jellemzése-03}}
		%
		Mivel a \( (B_n) \) sorozat monoton szűkül, ezért az 
		\[
			A_n \coloneq B_0 \setminus B_n \qquad (n \in \N)
		\]
		halmazsorozat monoton bővülve tart a \( B_0 \setminus B \in \ring{G} \) határhalmazhoz.
		\marginnote{
			Mivel \( B_n \subseteq B_0 \) és \( \mu(B_n) \) véges, ezért
			\[
				\mu(B_0 \setminus B_n) = \mu(B_0) - \mu(B_n).
			\]
		}%
		Ekkor
		\[
			\mu(B_0 \setminus B) \overset{\text{\ref{eq:kvázimérték-jellemzése-02}}}{=}
			\lim_{n \to \infty} \mu(B_0 \setminus B_n) =
		%	\lim_{n \to \infty} \bigl( \mu(B_0) - \mu(B_n) \bigr) =
			\mu(B_0) - \lim_{n \to \infty} \mu(B_n).
		\]
		Nyilván \( B \subseteq B_0 \). 
		Mivel \( \mu(B_0) \) véges, ezért a \( \mu \) monotonitás miatt \( \mu(B) \) is az.
		\[
			\mu(B_0 \setminus B) =
			\mu(B_0) - \lim_{n \to \infty} \mu(B_n) =
			\mu(B_0) - \mu(B).
		\]
		Innen \( \mu(B_0) \)-vel egyszerűsítve adódik a bizonyítandó állítás.
		
		\noindent\rule{\linewidth}{0.4pt}\\
		
		\fbox{\ref{eq:kvázimérték-jellemzése-03} \( \Longrightarrow \)
			  \ref{eq:kvázimérték-jellemzése-04}}
		%
		Az igazolandó \ref{eq:kvázimérték-jellemzése-04} állítás speciális esete \az{\ref{eq:kvázimérték-jellemzése-03}} kijelentésnek.
		
		\noindent\rule{\linewidth}{0.4pt}\\
		
		\fbox{\ref{eq:kvázimérték-jellemzése-04} \( \Longrightarrow \)
			  \ref{eq:kvázimérték-jellemzése-03}}
		%	  
		Ha a \( (B_n) \) sorozat monoton szűkülve tart a \( B \)-hez, akkor a
		\[
			C_n \coloneq B_n \setminus B \in \ring{G} \qquad (n \in \N)
		\]
		halmazsorozat monoton szűkülve tart az üres halmazhoz.
		\marginnote{
			Mivel \( B \subseteq B_n \) és \( \mu(B_n) \) véges, ezért
			\[
				\mu(B_n \setminus B) = \mu(B_n) - \mu(B).
			\]
		}
		Ekkor
		\[
			0 = 
			\mu( \emptyset ) =
			\lim_{n \to \infty} \mu( C_n ) =
			\lim_{n \to \infty} \mu( B_n \setminus B ) =
			\lim_{n \to \infty} \mu( B_n ) - \mu(B).
		\]
		Mivel itt \( \mu(B) \) véges, ezért átrendezés után kapjuk a bizonyítandó állítást.
		
		\noindent\rule{\linewidth}{0.4pt}\\
		
		A továbbiakban feltesszük, hogy \( \mu \) véges. Az alábbi állítást mutatjuk meg.
		
		\fbox{\ref{eq:kvázimérték-jellemzése-04} \( \Longrightarrow \)
			  \ref{eq:kvázimérték-jellemzése-02}}
		%
		Ugyanis
		
	\end{proof}
	
	\newpage
	\section{A Lebesgue-féle kvázimérték}
	
	A továbbiakban legyen \( p \in \posN \) egy rögzített kitevő, valamint az
	\[
		\x = (x_1, \dots, x_p), \ \y = (y_1, \dots, y_p) \in \R^p
	\]
	vektorok körében definiáljuk a komponensenkénti rendezést az alábbi módon:
	\[
		\begin{alignedat}{3}
			\x &\leq \y \quad &&\ratio \Longleftrightarrow \quad x_i &&\leq y_i \\[3pt]
			\x &<    \y \quad &&\ratio \Longleftrightarrow \quad x_i &&<    y_i
		\end{alignedat}
		\qquad (i = 1,\dots, p).
	\]
	Amennyiben \( \x < \y \), akkor az
	\[
		[\x, \y) \coloneq 
		\setc[\big]{ \vb{z} \in \R^p }{ \x \leq \vb{z} < \y } =
		[ x_1, y_1 ) \times \cdots \times [x_p, y_p )
	\]
	halmazt az \( \x, \y \) végpontú, balról zárt és jobbról nyílt (\( p \)-dimenziós) intervallumnak nevezzük. Könnyen belátható ilyenkor, hogy az
	\[
		\mathbf{I}^p \coloneq 
		\setc[\Big]{ \emptyset,\, [\x, \y) }{ \x, \y \in \R^p \,\text{ és }\, \x < \y }
	\]
	halmazrendszer egy félgyűrű. Tekintsük az \( \mathbf{I}^p \) által generált gyűrűt, vagyis az
	\[
		\mathcal{I}^p \coloneq 
		\mathcal{G}( \mathbf{I}^p ) =
		\setc[\Bigg]{ A \coloneq \bigcup_{k=0}^n I_k }
					{ I_0, \dots, I_n \in \mathbf{I}^p \text{ diszjunktak } (n \in \N) }
	\]
	halmazt.
	
	\begin{definition}{Lebesgue-féle kvázimérték}{}
		Legyen \( \func{ m_p }{ \mathbf{I}^p }{[0, +\infty)} \) az a véges halmazfüggvény, amire
		\marginnote{
%			Vagyis amennyiben adottak az
%			\[
%				I_0, \dots, I_n \in \mathbf{I}^p \quad (n \in \N)
%			\]
%			diszjunkt intervallumok, akkor az
%			\[
%			%	A \coloneq \bigcup_{k=0}^n I_k 
%				A \coloneq I_0 \cup \cdots \cup I_n \in \mathcal{I}^p
%			\]
%			halmaz ``kvázimértéke'' legyen
%			\[
%				\widetilde{\mu}_p(A) \coloneq \sum_{k=0}^{n} m_p(I_k)
%			\]
			Vagyis tetszőlegesen véve egy
			\[
				A \in \mathcal{I}^p, \ A = \bigcup_{k=0}^n I_k
			\]
			diszjunkt felbontású halmazt, akkor
			\[
				\widetilde{\mu}_p(A) = \sum_{k=0}^{n} m_p(I_k).
			\]
		}
		\[
			m_p( \emptyset ) \coloneq 0, \quad
			m_p \bigl( [\x, \y) \bigr) \coloneq \prod_{k=1}^{p} (y_i - x_i) 
			\qquad \bigl( [\x, \y) \in \mathbf{I}^p \bigr).
		\]
		Ekkor az \( m_p \) halmazfüggvény
		\[
			\func{ \widetilde{\mu}_p }{ \mathcal{I}^p }{[0, +\infty)}, \qquad
			\restr{ \widetilde{\mu}_p }{ \mathbf{I}^p } = m_p.
		\]
		kiterjesztését \emph{Lebesgue-féle kvázimértéknek} nevezzük.
	\end{definition}
	
	
	\newpage
	\begin{theorem}{}{}
		Az előbb definiált \( \func{ \widetilde{\mu}_p }{ \mathfrak{I}^p }{ [0, +\infty) } \)
		függvény egy kvázimérték.
	\end{theorem}
	\begin{proof}
		Mivel \( \widetilde{\mu}_p \) véges előmérték, 
		ezért \aref{th:előmérték-kvázimérték-kapcsolata} tétel alapján elegendő azt megmutatni,
		hogy minden \( \mathfrak{I}^p \)-beli \( A_{n+1} \subseteq A_n \ (n \in \N) \)
		sorozat esetén
		\[
			\bigcap_{n=0}^{\infty} A_n = \emptyset
			\qquad \Longrightarrow \qquad
			\lim_{n \to \infty} \widetilde{\mu}_p( A_n ) = 0.
		\]
		Indirekt tegyük fel, hogy van olyan, az előbbi feltételeknek eleget tevő \( (A_n) \) halmazsorozat, amelyre
		\[
			\delta \coloneq
			\lim_{n \to \infty} \widetilde{\mu}_p( A_n ) =
			\inf \setc[\big]{ \widetilde{\mu}_p( A_n ) }{ n \in \N } > 0.
		\]
		Ekkor megadható olyan \( (B_n) \) halmazsorozat, illetve \( (\delta_n) \) számsorozat, hogy
		\marginnote{%
			Tehát \( \func{(B_n)}{\N}{\mathcal{I}^p} \) és \( \func{(\delta_n)}{\N}{\R} \).
		}[-\baselineskip]
		\[
			\overline{B}_n \subseteq A_n
			\qquad \text{és} \qquad
			\widetilde{\mu}_p( A_n ) - \widetilde{\mu}_p( B_n ) < \delta_n
			\qquad (n \in \N).
		\]
		Továbbá legyen
		\[
			C_n \coloneq \bigcap_{k=0}^n B_k \qquad (n \in \N)
		\]
		Nyilvánvaló, hogy az így megadott halmazsorozatokra igaz a következő:
		\[
			\overline{C}_n \subseteq 
			\overline{B}_n \subseteq A_n \subseteq A_0 \qquad (n \in \N).
		\]
		Következésképpen a \( \overline{C}_n \) halmazok mindegyike korlátos és zárt, valamint
		\[
			\bigcap_{n=0}^{\infty} \overline{C}_n \subseteq \bigcap_{n=0}^{\infty} A_n = \emptyset
			\qquad \Longrightarrow \qquad
			\bigcap_{n=0}^{\infty} \overline{C}_n = \emptyset.
		\]
		Megmutatjuk, hogy egyetlen \( C_n \) halmaz sem üres.
		Mivel ilyenkor a \( \overline{C}_n \) halmazok sem üresek, 
		valamint ezek korlátos és zárt, egymásba skatulyázott intervallumok, 
		ezért a \hyperref[th:cantor-tétel]{Cantor-tétel} alkalmazásával a
		\marginnote{
			Két \( \R^p \)-beli \( \x \leq \y \) vektor esetén az
			\[
				[\x, \y] \coloneq \setc[\big]{ \vb{z} \in \R^p }{ \x \leq \vb{z} \leq \y }
			\]
			halmazt zárt intervallumnak nevezzük.
			\begin{theo*}[Cantor-tétel]\phantomsection\label{th:cantor-tétel}
				Amennyiben
				\[
					[\x_{n + 1}, \y_{n + 1}] \subseteq [\x_n, \y_n]
				\]
				fennáll minden \( n \in \N \) indexre, akkor
				\[
					\bigcap_{n=0}^{\infty} [\x_n, \y_n] \neq \emptyset.
				\]
			\end{theo*}
		}[-7.3\baselineskip]
		\[
			\bigcap_{n=0}^{\infty} \overline{C}_n \neq \emptyset
		\]
		ellentmondás áll elő.
		Tehát teljes indukcióval adódik, hogy \( C_n \neq \emptyset \ (n \in \N) \).
		Ugyanis az \( n = 0 \) alapesetben
		\[
			\widetilde{\mu}_p( C_0 ) = 
			\widetilde{\mu}_p( B_0 ) \geq 
			\widetilde{\mu}_p( A_0 ) \neq 0
			\qquad \Longrightarrow \qquad
			C_0 \neq \emptyset. \ \checkmark
		\]
		Továbbá, ha valamely rögzített \( n \in \N \) indexig teljesül az állítás, akkor
		%
		\begin{align*}
			\widetilde{\mu}_p \bigl( C_{n + 1} \bigr)
			&= \widetilde{\mu}_p \bigl( C_n \cap B_{n + 1} \bigr) \\[3pt]
			&= \widetilde{\mu}_p \bigl( C_n \bigr) + 
			   \widetilde{\mu}_p \bigl( B_{n + 1} \bigr) - 
			   \widetilde{\mu}_p \bigl( C_n \cup B_{n + 1} \bigr) \\[3pt]
			&\geq \widetilde{\mu}_p \bigl( C_n \bigr) +
			      \widetilde{\mu}_p \bigl( A_{n + 1} \bigr) - \delta_{n + 1} -
			      \widetilde{\mu}_p \bigl( A_n \bigr)
		\end{align*}
		%
		
	\end{proof}
	
\end{document}