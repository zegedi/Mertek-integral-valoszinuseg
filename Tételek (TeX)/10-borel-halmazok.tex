\documentclass[
%parspace, % Add vertical space between paragraphs
%noindent, % No indentation of first lines in each paragraph
%nohyp,	   % No hyphenation of words
%twoside,  % Double sided format
%draft,    % Quicker draft compilation without rendering images
%final,    % Set final to hide todos
]{elteikthesis}[2024/04/26]


% The minted package is also supported for source highlighting
% See elteikthesis_minted.tex for example
%\usepackage[newfloat]{minted}
\usepackage{enumitem}

% Document's metadata
\title{Mérték, integrál, valószínűség} % title
\subtitle{10. Vizsgatétel}


% The document
\begin{document}
	
	% Set document language
	\documentlang{hungarian}
	
	\section{Borel--halmazok}
	
	\begin{definition}{Borel--halmaz, Lebesgue--halmaz}{}
		Az \( \R^p \)-beli \emph{Borel--halmazok} rendszere az alábbi szigma-algebra:
		\[
			\Omega_p \coloneq \Omega( \mathcal{I}^p ) = \Omega( \mathbf{I}^p ).
		\]
		Az \( \R^p \)-beli \emph{Lebesgue--halmazok} rendszere az alábbi szigma-algebra:
		\[
			\Omega_p \coloneq \Omega( \mathcal{I}^p ) = \Omega( \mathbf{I}^p ).
		\]
	\end{definition}
	
	Vezessük be a soron következő \( \R^p \)-beli halmazrendszereket:
	\[
		\mathcal{T}_p \coloneq \setc[\big]{ A \subseteq \R^p }{ A \text{ nyílt} }, \quad
		\mathcal{C}_p \coloneq \setc[\big]{ B \subseteq \R^p }{ B \text{ zárt} }, \quad
		\mathcal{K}_p \coloneq \setc[\big]{ K \subseteq \R^p }{ K \text{ kompakt} }.
	\]
	\begin{alignat*}{2}
		\mathcal{T}_p &\coloneq \Bigl\{ A &&\subseteq \R^p \ \colon A \text{ nyílt} \Bigr\}, \\
		\mathcal{C}_p &\coloneq \Bigl\{ B &&\subseteq \R^p \ \colon B \text{ zárt} \Bigr\}, \\
		\mathcal{K}_p &\coloneq \Bigl\{ K &&\subseteq \R^p \ \colon K \text{ kompakt} \Bigr\}.
	\end{alignat*}
	
	\begin{statement}{}{}
		A \( p \)-dimenziós Borel--halmazok rendszerére az alábbi egyenlőségek igazak:
		\[
			\Omega_p = 
			\Omega( \mathcal{T}_p ) = 
			\Omega( \mathcal{C}_p ) = 
			\Omega( \mathcal{K}_p ).
		\]
	\end{statement}
	\begin{proof}
		Mivel minden \( \R^p \)-beli kompakt halmaz korlátos és zárt, ezért
		\marginnote{
			\( A \subseteq \R^p \) kompakt \( \iff \) \( A \) korlátos és zárt.
		}
		\[
			\mathcal{K}_p \subset \mathcal{C}_p \subseteq \Omega( \mathcal{C}_p )
			\qquad \Longrightarrow \qquad
			\Omega( \mathcal{K}_p ) \subseteq \Omega( \mathcal{C}_p ).
		\]
		Ha \( B \in \mathcal{C}_p \) zárt halmaz, 
		akkor alk \( (B_n) \) kompakt halmazokból képzett
		\marginnote{
			Hiszen \( A_n \subset A \ (n \in \N) \) korlátos és zárt.
		}
		\[
			B_n \coloneq \setc[\big]{ \x \in B }{ \norm{\x} \leq n } \quad (n \in \N)
			B = \bigcup_{n=0}^{\infty} B_n
			\qquad \Longrightarrow \qquad
			B \in \Omega( \mathcal{K}_p )
		\]
		Mivel minden szigma-algebra zárt a megszámlálható unióra, ezért
		\[
			B \in \Omega( \mathcal{K}_p )
			\quad \Longrightarrow \quad
			\mathcal{C}_p \subseteq \Omega( \mathcal{K}_p )
			\quad \Longrightarrow \quad
			\Omega( \mathcal{C}_p ) \subseteq \Omega( \mathcal{K}_p ).
		\]
		
		Ha \( A \in \mathcal{T}_p \) nyílt halmaz, akkor a komplementere zárt.
		\marginnote{
			Itt az \( \R^p \)-re vonatkozó komplementer
			\[
				A^c = \R^p \setminus A.
			\]
		}
		Tehát
		\[
			A^c \in \mathcal{C}_p \subseteq \Omega( \mathcal{C}_p ).
		\]
		Mivel minden szigma-algebra zárt a komplementer képzésre, ezért
		\[
			A \in \Omega( \mathcal{C}_p )
			\quad \Longrightarrow \quad
			\mathcal{T}_p \subseteq \Omega( \mathcal{C}_p )
			\quad \Longrightarrow \quad
			\Omega( \mathcal{T}_p ) \subseteq \Omega( \mathcal{C}_p ).
		\]
		Fordítva hasonló gondolatmenettel
		\marginnote{
			Ha \( B \in \mathcal{C}_p \) zárt, akkor \( B^c \in \mathcal{T}_p \) nyílt.
		}
		látható be, hogy
		\[
			\Omega( \mathcal{C}_p ) \subseteq \Omega( \mathcal{T}_p )
			\quad \Longrightarrow \quad
			\smash{\uwave{\Omega( \mathcal{C}_p ) = \Omega( \mathcal{T}_p )}}.
		\]
	\end{proof}
	
\end{document}
