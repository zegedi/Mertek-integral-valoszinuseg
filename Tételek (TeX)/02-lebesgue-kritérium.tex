\documentclass[
%parspace, % Add vertical space between paragraphs
%noindent, % No indentation of first lines in each paragraph
%nohyp,	   % No hyphenation of words
%twoside,  % Double sided format
%draft,    % Quicker draft compilation without rendering images
%final,    % Set final to hide todos
]{elteikthesis}[2024/04/26]


% The minted package is also supported for source highlighting
% See elteikthesis_minted.tex for example
%\usepackage[newfloat]{minted}
\usepackage{enumitem}

\makeatletter
\def\biggg{\bBigg@{3}}
\def\bigggm{\mathrel\biggg}
\def\bigggl{\mathopen\biggg}
\def\bigggr{\mathclose\biggg}
\def\Biggg{\bBigg@{3.5}}
\def\Bigggm{\mathrel\Biggg}
\def\Bigggl{\mathopen\Biggg}
\def\Bigggr{\mathclose\Biggg}
\makeatother

% Document's metadata
\title{Mérték, integrál, valószínűség} % title
\date{2024} % year of defense

% Author's metadata
\author{Egedi Viktor}
\degree{programtervező informatikus BSc}

% Superivsor(s)' metadata
\supervisor{Dr. Simon Péter} % internal supervisor's name
\affiliation{egyetemi tanár} % internal supervisor's affiliation

% University's metadata
\university{Eötvös Loránd Tudományegyetem} % university's name
\faculty{Informatikai Kar} % faculty's name
\department{Numerikus Analízis Tanszék} % department's name
\city{Budapest} % city
\logo{elte_cimer_szines} % logo

% The document
\begin{document}
	
	% Set document language
	\documentlang{hungarian}
	%\documentlang{english}
	
	\begin{theorem}{Lebesgue-kritérium}{}
		Legyen \( \func{f}{[a, b]}{\R} \) korlátos függvény, valamint
		\[
			\mathcal{A}_f \coloneq \setc[\Big]{ x \in [a, b] }{ f \notin \ContAt{x} }.
		\]
		Ekkor \( f \in \Riem{[a, b]} \) azzal ekvivalens, 
		hogy az \( \mathcal{A}_f \) halmaz nullamértékű.
	\end{theorem}
	\begin{proof}\,\\[6pt]
		\Ifstep
		Tegyük fel, hogy \( f \in \Riem{[a,b]} \). Ekkor az 1. lemma alapján
		\[
			\mathcal{A}_f = 
			\setc[\Big]{ x \in [a, b] }{ o_x(f) > 0 } =
			\bigcup_{n = 1}^{\infty} \setc[\bigg]{ x \in [a, b] }{ o_x(f) > \frac{1}{n} }
			\eqcolon \bigcup_{n = 1}^{\infty} A_n.
		\]
		Elegendő lenne azt belátni, hogy \( A_n \) nullamértékű%
		%
		\sidenote[][0pt]{%
			Megszámlálhatóan sok nullamértékű halmaz uniója szintén nullamértékű.
		}.
		%
		Sőt azt igazoljuk, hogy
		\[
			A_\delta \coloneq \setc[\Big]{ x \in [a, b] }{ o_x(f) > \delta }
			\qquad (\delta > 0)
		\]
		nullamértékű. 
		A továbbiak szempontjából legyen \( \delta > 0 \) egy tetszőlegesen rögzített érték.
		Mivel az \( f \) Riemann-integrálható, ezért bármely \( \varepsilon > 0 \)-hoz
		\[
			\exists \tau \subset [a, b] \text{ felosztás} \colon \quad
			\omega(f, \tau) < \varepsilon.
		\]
		Legyen \( \mathcal{I} \) azon \( \tau \) felosztás szerinti osztásintervallumoknak a halmaza, amik a belsejükben tartalmaznak \( A_\delta \)-beli pontot, azaz
		\[
			\mathcal{I} \coloneq
			\setc[\Big]{ J \in \mathcal{F}(\tau) }{ \inter(J) \cap A_\delta \neq \emptyset }.
		\]
		Világos, hogy ekkor \( \tau \cup \mathcal{I} \) lefedi az \( A_\delta \) halmazat%
		%
		\sidenote[][0pt]{%
			Mivel adott \( J \in \mathcal{I} \) intervallumhoz 
			van olyan \( z \in A_\delta \) pont, hogy
			\[
				z \in \inter(J)
				\quad \text{ és } \quad
				o_z(f) > \delta,
			\]
			ezért elmondható az alábbi becslés:
			\[
				O_J(f) = 
				\sup \mathcal{O}(f, J) \geq 
				o_z(f) > 
				\delta.
			\]
		%	\begin{align*}
		%		O_J(f) 
		%		&=    \sup \Bigl\{ \abs\big{ f(u) - f(v)} \,\colon\ u, v \in J \Bigr\} \\
		%		&\geq \inf \Bigl\{ \abs\big{ f(u) - f(v)} \,\colon\ u, v \in J \Bigr\} \\
		%		&\geq  o_x(f) > \delta.
		%	\end{align*}
		}.
		Továbbá
		\[
			\varepsilon	>
			\omega(f, \tau) =
			\sum_{I \in \mathcal{F}(\tau)} O_I(f) {\cdot} \abs{I} \geq
			\sum_{J \in \mathcal{I}} O_J(f) {\cdot} \abs{J} \geq
			\sum_{J \in \mathcal{I}} \delta {\cdot} \abs{J}.
		\]
		Következésképpen az \( \mathcal{I} \)-beli intervallumok hosszösszege így becsülhető:
		\[
		%	\sum_{J \in \mathcal{I}} \delta {\cdot} \abs{J} < \varepsilon
		%	\quad \iff \quad
			\sum_{J \in \mathcal{I}} \abs{J} < \frac{\varepsilon}{\delta}.
		\]
		Mivel \( \tau \) véges halmaz, ezért nullamértékű. 
		Következésképpen minden \( z \in \tau \) osztóponthoz hozzárendelhető 
		egy olyan \( J_z \subset \R \) intervallum, amellyel
		\[
			\tau \subset \bigcup_{z \in \tau} J_z
			\quad \text{ és } \quad
			\sum_{z \in \tau} \abs{J_z} < \varepsilon.
		\]
		Összességében elmondható, hogy
		\[
			A_\delta \,\subseteq\,
			\mathcal{I} \cup
		%	\mathopen{\raisebox{-0.9ex}{$ \biggl( $}}\,
			\bigcup_{z \in \tau} J_z
		%	\mathopen{\raisebox{-0.9ex}{$ \biggr) $}} 
			\quad \text{ és } \quad
		%	\sum_{z \in \tau} \abs{J_z}
			\sum_{z \in \tau} \abs{J_z} + \sum_{J \in \mathcal{I}} \abs{J} <
			\varepsilon + \frac{\varepsilon}{\delta} =
			\varepsilon \biggl( 1 + \frac{1}{\delta} \biggr).
		\]
		Ez pedig pontosan azt jelenti, hogy az \( A_\delta \) halmaz nullamértékű.
		
		\newpage
		
		\Backifstep
		Most legyen az \( \mathcal{A}_f \) halmaz nullamértékű. 
	%	Ekkor tetszőleges \( \delta > 0 \) esetén \( A_\delta \) is nullamértékű.
		Ekkor tetszőleges \( \varepsilon > 0 \)-hoz van olyan 
		\( I_n \subset \R \) korlátos és zárt intervallumsorozat \( (n \in \N) \), amellyel%
		%
		\sidenote[][0pt]{%
			Emlékeztetés gyanánt, ha \( x \in [a, b] \):
			\[
				x \in \mathcal{A}_f
				\quad \iff \quad
				f \notin \ContAt{x}.
			\]
		}
		\[
			\mathcal{A}_f \subseteq \bigcup_{n=0}^{\infty} \inter( I_n )
			\quad \text{ és } \quad
			\sum_{n = 0}^{\infty} \abs{ I_n } < \varepsilon.
		\]
		Ha pedig \( x \in [a, b] \) folytonossági pontja \( f \)-nek, 
		akkor az 1. lemma alapján
		\[
			f \in \ContAt{x} \quad \iff \quad o_x(f) = 0.
		\]
		Így a lokális oszcilláció jelentése miatt van olyan \( J_x \subset \R \) intervallum, hogy
		\[\label{eq:oszcilláció-felső-becslése}
			x \in \inter(J_x), \quad
		%	\qquad \text{ és } \quad
			O_{J_x} (f) = 
			\sup \Bigl\{ \abs\big{ f(u) - f(v) } \,\colon\, u, v \in J_x \cap [a, b] \Bigr\}
			< \varepsilon.
			\tag{\( * \)}
		\]
		Ezek alapján könnyen megadhatunk egy lefedését az \( [a, b] \) intervallumra:
		%
		\marginnote{
			\[
				\mathcal{A}^c_f = [a, b] \setminus \mathcal{A}_f
			\]
		}[2\baselineskip]
		\[
			[a, b] \subset 
			\Biggl(\, \bigcup_{n=0}^{\infty} \inter I_n  \Biggr) \cup
			\bigggl(\, \bigcup_{x \in \mathcal{A}_f^c} \inter J_x \bigggr).
		\]
		Ugyanakkor a Borel-lemma alapján az előbbi nyílt lefedésből kiválasztatunk
		olyan véges \( A \subset \N \) és \( B \subset \mathcal{A}_f^c \) halmazokat, 
		amelyekkel szintén lefedhetjük az \( [a, b] \) intervallumot:
		\[
			[a, b] \subset 
			\mathopen{\raisebox{-0.9ex}{$ \biggl( $}}\,
			\bigcup_{n \in A} \inter I_n
			\mathopen{\raisebox{-0.9ex}{$ \biggr) $}} 
			\ \cup \
			\mathopen{\raisebox{-0.9ex}{$ \biggl( $}}\,
			\bigcup_{x \in B} \inter J_x
			\mathopen{\raisebox{-0.9ex}{$ \biggr) $}}.
		\]
		Most vezessük be azt a \( \tau \subset [a, b] \) felosztást, 
		ami az \( I_n, J_x \ (n \in A,\, x \in B) \) intervallumok végpontjait 
		és az \( a, b \) számokat tartalmazza. Ekkor az
		\begin{alignat*}{6}
			U &\coloneq \Bigl\{ 
				&& & I &\in \mathcal{F}(\tau) \ \Big\vert \ && & I &\subseteq I_n &&\quad (n \in A) 
			\Bigr\} \\[3pt]
			%		
			V &\coloneq 
			\Bigl\{ 
				&& & J &\in \mathcal{F}(\tau) \ \Big\vert \ && & J &\subseteq J_x &&\quad (x \in B) 
			\Bigr\}
		\end{alignat*}
		osztásintervallumoknak (a nem feltétlenül diszjunkt) szétosztását tekintve
		\[
			\omega(f, \tau) = 
			\sum_{I \in \mathcal{F}(\tau)} O_I (f) {\cdot} \abs{I} \leq
			\sum_{I \in U} O_I (f) {\cdot} \abs{I} + \sum_{J \in V} O_J (f) {\cdot} \abs{J}.
		\]
		Mivel feltettük, hogy az \( f \) korlátos, ezért egy alkalmas \( C \geq 0 \) számmal%
		%
		\sidenote[][0pt]{%
			Minden \( I \subseteq [a, b] \) intervallum esetén
			\begin{align*}
				O_I(f) 
				&=    \sup \Bigl\{ \abs\big{f(x) - f(y)} \,\colon\, x, y \in I \Bigr\} \\
				&\leq \sup \Bigl\{ \abs\big{f(x)} + \abs\big{f(y)} \,\colon\, x, y \in I \Bigr\} \\
			%	&\leq \sup \Bigl\{ 2C \,\colon\, x, y \in I \Bigr\} \\
				&\leq 2C.
			\end{align*}
		}
		\[
			\abs\big{f(x)} \leq C \quad \bigl( x \in [a, b] \bigr)
			\qquad \Longrightarrow \qquad
			\smash{\uwave{O_I(f) \leq 2C}}.
		\]
		Ennek és a \eqref{eq:oszcilláció-felső-becslése}-os becslésnek a felhasználásával kapjuk, hogy
		\begin{align*}
			\omega(f, \tau) &\leq 
		%	\sum_{I \in U} O_I (f) {\cdot} \abs{I} + 
		%	\sum_{J \in V} O_J (f) {\cdot} \abs{J} \leq
			\sum_{I \in U} 2C          {\cdot} \abs{I} + 
			\sum_{J \in V} \varepsilon {\cdot} \abs{J}   \leq
			2C          \cdot \sum_{n=0}^{\infty} \abs{I_n} + 
			\varepsilon \cdot \sum_{\mathclap{J \in \mathcal{F}(\tau)}} \abs{J} \\[3pt] &<
			2C \varepsilon + \varepsilon (b - a) =
			\smash{\uwave{\varepsilon ( 2C + b - a )}}.
		\end{align*}
		Következésképpen \( f \in \Riem{[a, b]} \).
	\end{proof}
\end{document}