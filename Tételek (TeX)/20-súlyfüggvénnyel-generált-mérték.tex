\documentclass[
%parspace, % Add vertical space between paragraphs
%noindent, % No indentation of first lines in each paragraph
%nohyp,	   % No hyphenation of words
%twoside,  % Double sided format
%draft,    % Quicker draft compilation without rendering images
%final,    % Set final to hide todos
]{elteikthesis}[2024/04/26]


% The minted package is also supported for source highlighting
% See elteikthesis_minted.tex for example
%\usepackage[newfloat]{minted}
\usepackage{enumitem}

% Document's metadata
\title{Mérték, integrál, valószínűség} % title
\subtitle{20. Vizsgatétel}


% The document
\begin{document}
	
	% Set document language
	\documentlang{hungarian}
	
	\section{Súlyfüggvények}
	
	\begin{definition}{Súlyfüggvény}{}
		Legyen \( (X, \Omega, \mu) \) mértéktér, 
		\( f \in \StepFuncPlus \) egy adott függvény. Ekkor a
		\[
			\func{\mu_f}{\Omega}{[0, +\infty]}, \qquad 
			\mu_f(A) \coloneq 
			\int_A f \dd{\mu} \coloneq 
			\int f {\cdot} \chi_A \dd{\mu}
		\]
		leképezést \emph{súlyfüggvénynek} nevezzük.
	%	az \( f \) függvény \emph{integrálja} az \( A \) halmazon.
	\end{definition}
	
	\begin{notes}
		\item
		Speciálisan az \( A = X \) esetben
		\[
			\mu_f(X) = 
			\int_X f \dd{\mu} = 
			\int f {\cdot} \chi_X \dd{\mu} = 
			\int f \dd{\mu}.
		\]
		
		\item Amennyiben az \( A \in \Omega \) halmaz nullamértékű, akkor
		\[
			f {\cdot} \chi_A = 0 \ \mu \text{-m.m.}
			\quad \Longrightarrow \quad
			\mu_f(A) = 
		%	\int_A f \dd{\mu} =
			\int f {\cdot} \chi_A \dd{\mu} =
			0.
		\]
	\end{notes}
	
	\begin{statement}{}{}
		Legyen \( (X, \Omega, \mu) \) mértéktér, \( f \in \StepFuncPlus \).
		Ekkor a \( \mu_f \) súlyfüggvény mérték.
	\end{statement}
	\begin{proof}
		A \( \mu_f \) függvényről az alábbiak mondhatóak el.
		\begin{enumerate}
			\item
			\emph{Nemnegatív}, ugyanis az integrál monoton 
			és \( f {\cdot} \chi_A \in \StepFuncPos \ (A \in \Omega) \).
			
			\item
			\emph{Eltűnik \( \emptyset \)-ban}, hiszen
			\[
				\mu_f( \emptyset ) = 
				\int f {\cdot} \chi_\emptyset \dd{\mu} = 
				\int \chi_\emptyset \dd{\mu} = 
				\mu( \emptyset ) =
				0.
			\]
			
			\item
			\emph{Szigma-additív}, mert bármely \( A_n \in \Omega \ (n \in \N) \)
			páronként diszjunkt halmazsorozat esetén
			\[
				A \coloneq \bigcup_{n=0}^{\infty} A_n
				\qquad \Longrightarrow \qquad
				f {\cdot} \chi_A = \sum_{n=0}^{\infty} f {\cdot} \chi_{A_n}.
			\]
			Innen a Beppo Levi-tétel sorokra vonatkozó alakjából
			\[
				\mu_f(A) = 
			%	\int f {\cdot} \chi_{A_n} \dd{\mu} =
				\int \sum_{n=0}^{\infty} f {\cdot} \chi_{A_n} \dd{\mu} =
				\sum_{n=0}^{\infty} \int f {\cdot} \chi_{A_n} \dd{\mu} =
				\sum_{n=0}^{\infty} \mu_f(A_n).
			\]
		\end{enumerate}
		Továbbá \( \mu_f \) az \( \Omega \) szigma-algebrán van értelmezve, 
		tehát \( \mu_f \) valóban mérték.
	\end{proof}
	
\end{document} 