\documentclass[
%parspace, % Add vertical space between paragraphs
%noindent, % No indentation of first lines in each paragraph
%nohyp,	   % No hyphenation of words
%twoside,  % Double sided format
%draft,    % Quicker draft compilation without rendering images
%final,    % Set final to hide todos
]{elteikthesis}[2024/04/26]


% The minted package is also supported for source highlighting
% See elteikthesis_minted.tex for example
%\usepackage[newfloat]{minted}
\usepackage{enumitem}

\makeatletter
\def\biggg{\bBigg@{3}}
\def\bigggm{\mathrel\biggg}
\def\bigggl{\mathopen\biggg}
\def\bigggr{\mathclose\biggg}
\def\Biggg{\bBigg@{3.5}}
\def\Bigggm{\mathrel\Biggg}
\def\Bigggl{\mathopen\Biggg}
\def\Bigggr{\mathclose\Biggg}
\makeatother

% Document's metadata
\title{Mérték, integrál, valószínűség} % title
\subtitle{\circled{8} Vizsgatétel}
%\date{2024} % year of defense
%
%% Author's metadata
%\author{Egedi Viktor}
%\degree{programtervező informatikus BSc}
%
%% Superivsor(s)' metadata
%\supervisor{Dr. Simon Péter} % internal supervisor's name
%\affiliation{egyetemi tanár} % internal supervisor's affiliation
%
%% University's metadata
%\university{Eötvös Loránd Tudományegyetem} % university's name
%\faculty{Informatikai Kar} % faculty's name
%\department{Numerikus Analízis Tanszék} % department's name
%\city{Budapest} % city
%\logo{elte_cimer_szines} % logo

% The document
\begin{document}
	
	% Set document language
	\documentlang{hungarian}
	
	\section{Kiterjesztési tételek}
	
	\begin{theorem}{Kvázimérték kiterjesztése mértékké}{}
		Legyen \( X \) egy halmaz, 
		\( \ring{G} \subseteq \powerset{X} \) gyűrű, 
		\( \func{ \widetilde{\mu} }{ \ring{G} }{ [0, +\infty] } \) kvázimérték.\\
		Ekkor van olyan \( \Omega \subseteq \powerset{X} \) szigma-algebra és 
		\( \func{ \mu }{ \Omega }{ [0, +\infty] } \) mérték, 
		hogy \( \widetilde{\mu} = \restr{\mu}{\ring{G}} \).
	\end{theorem}
	\begin{proof}
		Legyen tetszőleges \( A \in \powerset{X} \) halmaz esetén
		\[
			\Sigma_A \coloneq 
			\setc[\Bigg]{ \func{(\sigma_n)}{ \N }{ \ring{G} }}{ A \subseteq \bigcup_{n=0}^\infty \sigma_n }
		\]
		valamint az \( \inf \emptyset \coloneq +\infty \) megállapodás mellett
		\[
			\func{ \mu^* }{ \powerset{X} }{ \overline{\R} }, \quad
			\mu^*(A) \coloneq
			\inf \setc[\Bigg]{ \sum_{n=0}^{\infty} \widetilde{\mu}( \sigma_n ) }
                { (\sigma_n) \in \Sigma_A }
		\]
		\begin{tcolorbox}[colback=green!10, colframe=green!80]
			\begin{lem*}
				Az így definiált \( \mu^* \) halmazfüggvényre a következők igazak.
				\begin{enumerate}
					\item 
					\textbf{Nemnegatív}, azaz \( \mu^* \geq 0 \).
					
					\item 
					\textbf{Eltűnik \( \boldsymbol{\emptyset} \)-ban}, azaz
					\( \mu^*(\emptyset) = 0 \).
					
					\item 
					\textbf{Monoton}, 
					azaz minden \( B \subseteq A \) esetén \( \mu^*(B) \leq \mu^*(A) \).
					
					\item\label{eq:mu-csillag-szubadditív}
					\textbf{Szubadditív}, 
					azaz minden \( A_n \ (n \in \N ) \) halmazsorozat esetén
					\[
						\mu^* \Biggl( \, \bigcup\limits_{n=0}^\infty A_n \Biggr) \leq 
						\sum\limits_{n=0}^{\infty} \mu^*(A_n).
					\]
				\end{enumerate}
			\end{lem*}
		\end{tcolorbox}
		\begin{proof*}\,
			\begin{enumerate}
				\item 
				Nyilvánvalóan igaz, hiszen \( \widetilde{\mu} \) nemnegatív.
				
				\item 
				Mivel \( \mu^*(\emptyset) \geq 0 \), és a konstans üres halmazból képzett \( (\emptyset) \in \Sigma_\emptyset \), ezért
				\[
					\mu^*(\emptyset) \leq \sum_{n=0}^{\infty} \widetilde{\mu}( \emptyset ) = 0
					\qquad \Longrightarrow \qquad
					\mu^*(\emptyset) = 0.
				\]
				
				\item
				A feltétel miatt \( \Sigma_A \subseteq \Sigma_B \), 
				plusz az infimum tulajdonságaiból adódik.
				
				\item Két esetet különböztetünk meg.
				\begin{enumerate}
					\item 
					Ha valamilyen \( n \in \N \) indexre \(  \mu^*(A_n) = +\infty \), akkor igaz.
					
					\item 
					Ha minden \( n \in \N \) index esetén \( \mu^*(A_n) \) véges, 
					akkor az infimum tulajdonság szerint
					\[
						\forall \varepsilon_n > 0 \text{-hoz}, \
						\exists ( \sigma_{nk} ) \in \Sigma_{A_n} \ \colon \quad
						\sum_{n=0}^{\infty} \widetilde{ \mu }( \sigma_{nk} ) < \mu^*(A_n) + \varepsilon_n.
					\]
					Ugyanakkor
					\[
						\bigcup_{n=0}^{\infty} A_n \subseteq 
						\bigcup_{n=0}^{\infty} \bigcup_{k=0}^{\infty} \sigma_{nk}
						(\sigma_{nk}) \in \Sigma_{\cup A_n }
					\]
					\[
						\mu^* \Biggl( \bigcup_{n=0}^{\infty} A_n \Biggr) \leq
						\sum_{n=0}^{\infty} \sum_{k=0}^{\infty} \mu^*( \sigma_{nk} ) < 
						\sum_{n=0}^{\infty} \mu^*( A_n ) + 
						\sum_{n=0}^{\infty} \varepsilon_n < 
						\sum_{n=0}^{\infty} \mu^*( A_n ) + \varepsilon.
					\]
				\end{enumerate}
			\end{enumerate}
		\end{proof*}
		
		\begin{tcolorbox}[colback=yellow!30, colframe=yellow!80]
			\begin{def*}
				Egy \( A \in \powerset{X} \) halmaz \emph{\( \boldsymbol{\mu^*} \)-mérhető}, amennyiben
				\[
					\forall B \in \powerset{X} \ \colon \quad
					\mu^*(B) = \mu^*(B \cap A) + \mu^*(B \setminus A).
				\]
			\end{def*}
		\end{tcolorbox}
		
		\begin{tcolorbox}[colback=green!10, colframe=green!80]
			\begin{lem*}\phantomsection\label{lem:mu-csillag-merheto}
				Egy \( A \in \powerset{X} \) halmaz pontosan akkor \( \mu^* \)-mérhető, ha
				\[
					\forall B \in \powerset{X} \ \colon \quad
					\mu^*(B) \geq \mu^*(B \cap A) + \mu^*(B \setminus A).
				\]
			\end{lem*}
		\end{tcolorbox}
		\begin{proof*}
			Ugyanis \hyperref[eq:mu-csillag-szubadditív]{\( \mu^* \) szubadditív} tulajdonsága miatt a
			\[
				\mu^*( B ) =
				\mu^* \bigl( (B \cap A) \cup (B \setminus A) \bigr) \leq
				\mu^*( B \cap A ) + \mu^*( B \setminus A )
			\]
			fordított irányú egyenlőtlenség minden \( B \in \powerset{X} \) halmazra fennáll.
			\null\hfill \( \blacksquare \)
		\end{proof*}
		
		\noindent\rule{\linewidth}{0.4pt}\\
		
		Vezessük be a következő halmazrendszert
		\[
			\Omega \coloneq \setc[\big]{ A \in \powerset{X} }{ A \ \mu^* \text{-mérhető}}.
		\]
		Ekkor \( \ring{G} \subseteq \Omega \), 
		valamint \( \mu^*(G) = \widetilde{\mu}(G) \) minden \( G \in \ring{G} \) esetén.
		
		Ehhez azt kell belátni, hogy minden \( G \in \ring{G}, \ B \in \powerset{X} \) halmazra
		\[
			\mu^*(B) \geq
			\mu^*(B \cap G) + \mu^*(B \setminus G).
		\]
		Ha \( \mu^*(B) = +\infty \), akkor az állítás teljesül, 
		különben \( \Sigma_B \neq \emptyset \) miatt
		\[
			\exists (\sigma_n) \in \Sigma_B \, \colon \
			\sum_{n=0}^{\infty} \widetilde{\mu}( \sigma_n ) < \mu^*(B) + \varepsilon.
		\]
		Ugyanakkor \( \widetilde{\mu} \) additív, ezért
		\marginnote{
			Kihasználjuk, hogy
			\[
				\sigma_n = 
				(\sigma_n \cap G) \cup (\sigma_n \setminus G)
				\quad (n \in \N)
			\]
			diszjunkt felbontás.
		}
		\[
			\sum_{n=0}^{\infty} \widetilde{\mu}( \sigma_n ) =
%			\sum_{n=0}^{\infty} \widetilde{\mu} 
%			\bigr( (\sigma_n \cap G) \cup (\sigma_n \setminus G) \bigr) =
			\sum_{n=0}^{\infty} \widetilde{\mu}( \sigma_n \cap G ) +
			\sum_{n=0}^{\infty} \widetilde{\mu}( \sigma_n \setminus G ).
		\]
		Világos, hogy
		\[
			(\sigma_n \cap G) \in \Sigma_{B \cap G}
			\quad \text{és} \quad
			(\sigma_n \setminus G) \in \Sigma_{B \setminus G}.
		\]
		Innen \( \mu^* \) definíciója alapján
		\[
			\sum_{n=0}^{\infty} \widetilde{\mu}( \sigma_n \cap G ) \geq \mu^*( B \cap G ), \qquad
			\sum_{n=0}^{\infty} \widetilde{\mu}( \sigma_n \setminus G ) \geq \mu^*( B \setminus G ).
		\]
		Összefoglalva
		\[
			\mu^*(B) + \varepsilon >
			\mu^*( B \cap G ) + \mu^*( B \setminus G ).
		\]
		Tekintsük a ``kvázi-konstans'' halmazsorozatot
		\[
			(G, \emptyset, \emptyset, \dots, \emptyset, \dots) \in \Sigma_G
			\qquad \Longrightarrow \qquad
			\mu^*( G ) \leq \widetilde{\mu}( G ).
		\]
		Továbbá minden \( (\sigma)_n \in \Sigma_B \) halmazsorozat esetén
		\[
			G \subseteq \bigcup_{n=0}^{\infty} \sigma( A_n )
			\quad \Longrightarrow \quad
			\widetilde{\mu}( G ) \leq \sum_{n=0}^{\infty} \widetilde{\mu}( \sigma_n )
			\quad \Longrightarrow \quad
			\widetilde{\mu}( G ) \leq \mu^*( G ).
		\]
		Összefoglalva \( \mu^*( G ) = \widetilde{\mu}( G ) \).
		
		\noindent\rule{\linewidth}{0.4pt}\\
		
		Már csak azt kéne bebizonyítani, hogy egy \( \Omega \) szigma-algebra.
		\marginnote{
			Ugyanis minden \( B \in \powerset{X} \) esetén
			\begin{align*}
				\mu^*(B) 
				&= \mu^*(B \cap X) + \mu^*(B \setminus X) \\
				&= \mu^*(B) + \mu^*(\emptyset) \\
				&= \mu^*(B).
			\end{align*}
			Amennyiben \( A \in \Omega \), akkor
			\begin{align*}
				\mu^*(B) 
				&= \mu^*( B \cap A^c ) + \mu^*( B \setminus A^c) \\
				&= \mu^*( B \setminus A ) + \mu^*( B \cap A).
			\end{align*}
		}
		\begin{enumerate}
			\item Az \( X \in \Omega \) tartalmazás teljesül. \checkmark
			\item A komplementerképzésre való zártság is teljesül. \checkmark
			\item Azt kell igazolni, hogy \( \Omega \) zárt a megszámlálható unióra, vagyis
			\[
				A_n \in \Omega \ (n \in \N)
				\qquad \Longrightarrow \qquad
				\bigcup_{n=0}^{\infty} A_n \in \Omega.
			\]
		\end{enumerate}
		Először megmutatjuk, 
		hogy bármely \( A_0, A_1 \in \Omega \) esetén \( A_0 \cup A_1 \in \Omega \). Ugyanis
		\begin{align*}
			\mu^*(B) 
			&\geq \mu^*(B \cap A_0) + \mu^*(B \setminus A_0) \\
			&\geq \mu^* \bigl( (B \cap A_0) \cap A_1 \bigr) + 
				  \mu^* \bigl( (B \cap A_0) \setminus A_1 \bigr) +
				  \mu^* \bigl( (B \setminus A_0) \cap A_1 \bigr) \\
			&+    \mu^* \bigl( (B \setminus A_0) \setminus A_1 \bigr)
		\end{align*}
		Ugyanis vegyük észre, 
		hogy \( (B \setminus A_0) \setminus A_1 = B \cap (A_0 \cup A_1) \), valamint
		\begin{align*}
			B \cap (A_0 \cup A_1)
			&= \bigl( (B \cap A_0) \cap A_1 \bigr) \cup
			   \bigl( (B \cap A_0) \setminus A_1 \bigr) \cup
			   \bigl( (B \cap A_1) \setminus A_0 \bigr) \\
			&= \bigl( (B \cap A_0) \cap A_1 \bigr) \cup
			   \bigl( (B \cap A_0) \setminus A_1 \bigr) \cup
			   \bigl( (B \setminus A_0) \cap A_1 \bigr).
		\end{align*}
		Alkalmazva a \hyperref[eq:mu-csillag-szubadditív]{\( \mu^* \) szubadditív} 
		tulajdonságát kapjuk, hogy 
		\[
			\mu^*(B) \geq 
			\mu^* \bigl( B \cap (A_0 \cap A_1) \bigr) + 
			\mu^* \bigl( B \setminus (A_0 \cap A_1) \bigr)
		\]
		azaz \( \Omega \) valóban zárt a kételemű unióra. 
		Ugyanakkor (lásd \hyperref[lem:mu-csillag-merheto]{\( \mu^*\)-mérhető lemma})
		\[
			\mu^*(B) = 
			\mu^* \bigl( B \cap (A_0 \cap A_1) \bigr) + 
			\mu^* \bigl( B \setminus (A_0 \cap A_1) \bigr)
		\]
		Speciálisan \( A_0 \cap A_1 = \emptyset \), 
		valamint a \( B \leftrightarrow B \cap (A_0 \cup A_1) \) szerepcsere után
		\[
			\mu^* \bigl( B \cap (A_0 \cup A_1) \bigr) =
			\mu^*( B \cap A_0 ) + \mu^*( B \cap A_1 ).
		\]
		Innen teljes indukcióval adódik, hogy
		\[
			\bigcup_{k=0}^n A_k \in \Omega
		\]
		valamint ha az \( A_n \ (n \in \N) \) halmazok páronként diszjunktak, akkor
		\[
			\mu^* \Biggl( B \cap \bigcup_{k=0}^n A_k \Biggr) = 
			\sum_{k=0}^{n} \mu^*(B \cap A_k).
		\]
		Legyenek tehát az \( A_n \ (n \in \N) \) halmazok páronként diszjunktak. Ekkor
		\begin{align*}
			\mu^*(B)
			&\geq \mu^* \Biggl( B \cap      \bigcup_{k=0}^n A_k \Biggr) +
			      \mu^* \Biggl( B \setminus \bigcup_{k=0}^n A_k \Biggr) \\[3pt]
			&\geq \sum_{k=0}^{n} \mu^*(B \cap A_k) +
			      \mu^* \Biggl( B \setminus \bigcup_{k=0}^n A_k \Biggr) \\
		\end{align*}
		Véve az \( n \to \infty \) határátmenetet, 
		majd kihasználva \hyperref[eq:mu-csillag-szubadditív]{\( \mu^* \) szubadditivitását} 
		\[
			\mu^*(B) \geq
			\sum_{k=0}^{\infty} \mu^*(B \cap A_k) +
			\mu^* \Biggl( B \setminus \bigcup_{k=0}^{\infty} A_k \Biggr) \geq
			\mu^* \Biggl( B \cap \bigcup_{k=0}^{\infty} A_k \Biggr) +
			\mu^* \Biggl( B \setminus \bigcup_{k=0}^{\infty} A_k \Biggr)
		\]
		Innen rögtön következik, hogy \( \Omega \) zárt a megszámlálható unióra, vagyis
		\marginnote{
			Ez akkor is igaz, amikor az \( A_n \ (n \in \N) \) halmazok nem páronként diszjunktak.
		}
		\[
			\bigcup_{n=0}^{\infty} A_n \in \Omega.
		\]
		Tehát \( \Omega \) valóban szigma-algebra, továbbá speciális választásként
		\[
			B \coloneq \bigcup_{n=0}^{\infty} A_n
			\qquad \Longrightarrow \qquad
			\mu^* \Biggl( \,\bigcup_{n=0}^{\infty} A_n \Biggr) =
			\sum_{n=0}^{\infty} \mu^*( A_n ).
		\]
		Vagyis \( \restr{\mu^*}{\Omega} \) szigma-additív, 
		ezért minden eddigi alapján \( \restr{\mu^*}{\Omega} \) egy mérték.
	\end{proof}
	
	\begin{definition}{Külső mérték}{}
		Legyen \( X \) egy halmaz, 
		valamint \( \func{\mu^*}{ \powerset{X} }{ \overline{\R} } \) olyan halmazfüggvény, ami
			\begin{enumerate}
			\item 
			\textbf{nemnegatív}, azaz \( \mu^* \geq 0 \);
			
			\item 
			\textbf{eltűnik \( \boldsymbol{\emptyset} \)-ban}, azaz
			\( \mu^*(\emptyset) = 0 \);
			
			\item 
			\textbf{monoton}, 
			azaz minden \( B \subseteq A \) esetén \( \mu^*(B) \leq \mu^*(A) \);
			
			\item
			\textbf{szubadditív}, 
			azaz minden \( A_n \ (n \in \N ) \) halmazsorozat esetén
			\[
				\mu^* \Biggl( \, \bigcup\limits_{n=0}^\infty A_n \Biggr) \leq 
				\sum\limits_{n=0}^{\infty} \mu^*(A_n).
			\]
		\end{enumerate}
		Ekkor azt mondjuk, hogy \( \mu^* \) egy \emph{külső mérték}.
	\end{definition}
	
	\begin{theorem}{Caratheodory-tétel}{}
		Legyen \( X \) egy halmaz, 
		\( \func{\mu^*}{ \powerset{X} }{ \overline{\R} } \) külső mérték, valamint
		\[
			\Omega \coloneq \setc[\big]{ A \in \powerset{X} }{ A \ \mu^* \text{-mérhető} }.
		\]
		Ekkor \( \Omega \) szigma-algebra és \( \mu \coloneq \restr{\mu^*}{\Omega} \) mérték.
	\end{theorem}
	
\end{document}