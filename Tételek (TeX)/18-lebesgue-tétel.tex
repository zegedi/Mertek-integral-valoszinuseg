\documentclass[
%parspace, % Add vertical space between paragraphs
%noindent, % No indentation of first lines in each paragraph
%nohyp,	   % No hyphenation of words
%twoside,  % Double sided format
%draft,    % Quicker draft compilation without rendering images
%final,    % Set final to hide todos
]{elteikthesis}[2024/04/26]


% The minted package is also supported for source highlighting
% See elteikthesis_minted.tex for example
%\usepackage[newfloat]{minted}
\usepackage{enumitem}

% Document's metadata
\title{Mérték, integrál, valószínűség} % title
\subtitle{\circled{18} Vizsgatétel}


% The document
\begin{document}
	
	% Set document language
	\documentlang{hungarian}
	
	\section{Lebesgue-tétel}
	
	\,
	
	\begin{theorem}{Lebesgue-tétel}{lebesgue}
		Legyen \( (X, \Omega, \mu) \) mértéktér, \( p \in [1, +\infty] \), 
		valamint az \( f_n \in \Lebesgue[p] \ (n \in \N) \) egy olyan függvénysorozat, 
		amelyre a következők igazak:
		\begin{enumerate}[label=\roman*)]
			\item\label{eq:lebesgue-01}
			majdnem mindenhol létezik a \( \lim(f_n) \) pontonkénti limesz;
			
			\item\label{eq:lebesgue-02}
			alkalmas \( F \in \StepFuncPlus, \ \norm{F}_p < +\infty \) függvénnyel
			minden \( n \in \N \) indexre
			\[
			%	\abs{f_n} \leq F \qquad \mu \text{-m.m.}
				\abs\big{f_n(x)} \leq F(x) \qquad (\mu \text{-m.m. } x \in X).
			\]
		\end{enumerate}
		Ekkor
		\begin{enumerate}[label=\alph*)]
			\item\label{th:lebesgue-01}
			van olyan \( \func{f}{X}{\R} \) mérhető függvény, hogy
			\[
				f = \lim_{n \to \infty} f_n \quad \mu \text{-m.m.};
			\]
			
			\item\label{th:lebesgue-02}
			minden \ref{th:lebesgue-01}-beli \( f \) függvényre \( f \in \Lebesgue[p] \), 
			továbbá \( p \in [1, +\infty) \) esetén
			\[
				\lim\limits_{n \to \infty} \norm{f_n - f}_p = 0.
			\]
		\end{enumerate}
	\end{theorem}
	\begin{proof}
		\Aref{eq:lebesgue-01} és \ref{eq:lebesgue-02} feltétel figyelembe vételével
		van olyan \( A \in \Omega \), hogy
		\[
			\mu(A) = 0, \qquad
			\lim_{n \to \infty} f_n(x) \in \R, \qquad
			\abs\big{ f_n(x) } \leq F(x) < +\infty
		\]
		fennáll minden \( x \in X \setminus A \) helyen és \( n \in \N \) indexre. 
		\marginnote{
			Tehát az \( f \) függvény nem más, mint a
			\[
			%	f = \lim_{n \to \infty} \bigl( f_n \cdot \chi_{X \setminus A} \bigr)
				g_n \coloneq f_n \cdot \chi_{X \setminus A} \quad (n \in \N)
			\]
			mérhető függvényekből álló sorozat
			\[
				f = \lim(g_n)	
			\]
			határfüggvénye, ami szintén mérhető.
		}
		Legyen
		\[
			f(x) \coloneq 
		%	\lim_{n \to \infty} \Bigl( f_n(x) \cdot \chi_C(x) \Bigr) =
			\left\{
			\begin{alignedat}{2}
				&\lim_{n \to \infty} f_n(x), \quad && \text{ha } x \in X \setminus A, \\[3pt]
				&0,                          \quad && \text{ha } x \in A.
			\end{alignedat}
			\right.
		\]
		Ekkor az \( f = \lim(f_n) \) majdnem mindenhol, és az \( f \) mérhető és véges, mert
		\[
			\abs\big{f(x)} \leq F(x) < +\infty \qquad (x \in X).
		\]
		
		Ha az \( \func{f}{X}{\R} \) függvény eleget tesz \aref{th:lebesgue-01}-nak,
		akkor \aref{eq:lebesgue-02} feltétel alapján
		\[
			\abs\big{f(x)} \leq F(x) \qquad (\mu \text{-m.m. } x \in X).
		\]
		Ekkor az integrál monotonitása miatt
		\[
			\norm{f}_p \leq \norm{F}_p < +\infty
			\qquad \Longrightarrow \qquad
			f \in \Lebesgue[p].
		\]
		
		Azt kell már csak megmutatnunk, 
		hogy \( \lim\limits_{n \to \infty} \norm{f - f_n}_p = 0\). Legyen ehhez
		\[
			g_n \coloneq \abs{f - f_n}^p \qquad (n \in \N).
		\]
		Mivel \( g_n \) nemnegatív és mérhető minden \( n \in \N \) indexre, 
		ezért \( g_n \in \StepFuncPlus \), és
		\[
			g_n = 
			\abs{f - f_n}^p \leq
			\bigl( \abs{f} + \abs{f_n} \bigr)^p \leq
			2^p \cdot F^p
			\qquad \mu \text{-m.m.}
		\]
		\newpage
		Alkalmazva a Fatou-lemma második állítását
		\marginnote{
			\begin{theo*}[Fatou-lemma II.] Ha egy
				\[
					\func{(h_n)}{\N}{\StepFuncPlus}
				\]
				sorozathoz van olyan \( G \in \StepFuncPlus \), hogy
				\[
					\int G \dd{\mu} < +\infty, \quad
					h_n \leq G \ \mu \text{-m.m.},
				\]
				akkor
				\[
					\limsup_{n \to \infty} \int h_n \dd{\mu} \leq
					\int \limsup_{n \to \infty} h_n \dd{\mu}
				\]
				akkor
			\end{theo*}
		}[-2\baselineskip]
		\[
			\limsup_{n \to \infty} \int g_n \dd{\mu} \leq
			\int \limsup_{n \to \infty} g_n \dd{\mu} =
			\int \lim_{n \to \infty} g_n \dd{\mu} = 0.
		\]
		Tehát
		\[
			\limsup_{n \to \infty} \norm{f_n - f}_p = 
			\liminf_{n \to \infty} \norm{f_n - f}_p = 
			\lim_{n \to \infty} \norm{f_n - f}_p = 0.
		\]
	\end{proof}
	
	\begin{notes}
		\item Speciálisan \( p = 1 \) esetén a Lebesgue-tétel következménye, hogy
		\[
			\int f \dd{\mu} = \lim_{n \to \infty} \int f_n \dd{\mu}.
		\]
		Ugyanis figyelembe véve az
		\[
			\abs\bigg{ \int f \dd{\mu} - \int f_n \dd{\mu} } \leq
			\int \abs\big{ f - f_n } \dd{\mu} =
			\norm{ f - f_n }_1 \longrightarrow 0 
			\qquad (n \to \infty).
		\]
		
		\item
		Speciálisan tegyük fel, hogy a \( \mu \) véges és az \( (f_n) \) függvénysorozat egyenletesen korlátos majdnem minden pontban, vagyis tetszőleges \( n \in \N \) indexre
		\[
			\abs\big{f_n(x)} \leq C \qquad (\mu \text{-m.m. } x \in X).
		\]
		Ekkor teljesül a Lebesgue tétel második feltétele, ugyanis \( F \equiv C \) mellett
		\[
			\norm{F}_p = 
			\biggl( \int C^p \dd{\mu} \biggr)^{\! 1/p} =
			C \cdot \bigl( \mu( X ) \bigr)^{1/p} < 
			+\infty.
		\]
	\end{notes}
	
	\begin{theorem}{Kis Lebesgue-tétel}{}
		Legyen \( (X, \Omega, \mu) \) mértéktér, \( \mu \) véges, 
		valamint az \( f_n \in \Lebesgue \ (n \in \N) \) olyan függvénysorozat, 
		amelyre a következők igazak:
		\begin{enumerate}[label=\roman*)]
			\item
			majdnem mindenhol létezik a \( \lim(f_n) \) pontonkénti limesz;
			
			\item
			megadható olyan \( C \) konstans, hogy minden \( n \in \N \) indexre
			\[
			%	\abs{f_n} \leq F \qquad \mu \text{-m.m.}
				\abs\big{f_n(x)} \leq C \qquad (\mu \text{-m.m. } x \in X).
			\]
		\end{enumerate}
		Ekkor van olyan \( f \in \Lebesgue \) függvény, 
		hogy majdnem mindenhol \( f = \lim(f_n) \) és
		\[
		%	f = \lim_{n \to \infty} f_n \quad \mu \text{-m.m.}
		%	\qquad \text{és} \qquad
			\int f \dd{\mu} = \lim_{n \to \infty} \int f_n \dd{\mu}.
		\]
	\end{theorem}
	
	\newpage
	\section{Teljesség}
	
	\begin{theorem}{Az \( \Lebesgue[p] \) teljessége}{lp-teljes}
		Legyen \( (X, \Omega, \mu) \) mértéktér, 
		valamint \( 1 \leq p \leq +\infty \) esetén \( f_n \in \Lebesgue[p] \ (n \in \N) \).\\[6pt]
		Tegyük fel, hogy bármely \( \varepsilon > 0 \)-hoz van olyan \( N \in \N \) küszöbindex, hogy
		\[
			\norm{f_m - f_n}_p < \varepsilon \qquad (m,n > N).
		\]
		Ekkor van olyan \( f \in \Lebesgue[p] \) függvény,
		hogy \( \lim\limits_{n \to \infty} \norm{f - f_n}_p = 0 \).
	\end{theorem}
	\begin{proof}
		A Cauchy-kritérium miatt van olyan \( (n_k) \) indexsorozat, hogy
		\marginnote{
			\( \func{(n_k)}{\N}{\N} \) szigorúan monoton növő.
		}
		\begin{equation}\label{eq:lp-teljes-cauchy-becslés}
			\norm\big{ f_{n_{k + 1}} - f_{n_k} }_p < \frac{1}{2^k} 
			\qquad (k \in \N).
			\tag{\( * \)}
		\end{equation}
		Ekkor a Beppo Levi-tétel alapján létezik az \( \StepFuncPlus \)-beli
%		\marginnote{
%			\begin{theo*}[Beppo Levi-tétel]\phantomsection\label{th:beppo-levi}\,\\
%				Minden \( \func{(h_n)}{\N}{\StepFuncPlus} \) sorozatra
%				\[
%					\sum_{n=0}^{\infty} h_n \in \StepFuncPlus,
%				\]
%				valamint
%				\[
%					\int \sum_{n=0}^{\infty} h_n \dd{\mu} =
%					\sum_{n=0}^{\infty} \int h_n \dd{\mu}.
%				\]
%			\end{theo*}
%		}
		\[
			g_k \coloneq f_{n_{k + 1}} - f_{n_k} \quad (k \in \N), \qquad
			g \coloneq \sum_{k=0}^{\infty} \abs{ g_k }
		\]
		összegfüggvény, amelyre
		\marginnote{
			Lásd \hyperref[th:minkowski]{Minkowski-egyenlőtlenség} és \eqref{eq:lp-teljes-cauchy-becslés}.
		}
		\[
			\norm{g}_p \leq
		%	\norm{ \sum_{k=0}^{\infty} \abs{g_k} }_p \leq
			\sum_{k=0}^{\infty} \norm{ g_k }_p <
			\sum_{k=0}^{\infty} \frac{1}{2^k} =
		%	\frac{1}{1 - 1/2} =
			2.
		\]
		Következésképpen \( g \) majdnem mindenhol véges, 
		így a \( \sum( g_k ) \) teleszkopikus összegfüggvény \( \mu \)-m.m. abszolút konvergens.
		Ugyanakkor
		\[
			\sum_{k=0}^{\ell - 1} \bigl( f_{n_{k + 1}} - f_{n_k} \bigr) =
			f_{n_\ell} - f_{n_0}
			\qquad (1 \leq \ell \in \N),
		\]
		tehát az \( (f_{n_\ell}) \) részsorozat majdnem mindenhol konvergens. Innen
		\[
			\abs\big{ f_{n_\ell} } =
			\abs\Bigg{ \sum_{k=0}^{\ell - 1} g_k + f_{n_0} } \leq g + \abs{ f_{n_0} }
			\qquad (1 \leq \ell \in \N)
		\]
		majdnem mindenhol igaz,
		\marginnote{
			\begin{theo*}[Minkowski-egyenlőtlenség]\phantomsection\label{th:minkowski}
				\[
					\norm{ f + h }_p \leq \norm{ f }_p + \norm{ h }_p
				\]
				bármilyen \( f, h \in \Lebesgue[p] \) függvényre.
			\end{theo*}
		}[-1.4\baselineskip]
		továbbá a \hyperref[th:minkowski]{Minkowski-egyenlőtlenség} miatt
		\[
			\norm\big{ \abs{ f_{n_0} } + g }_p \leq 
			\norm{ f_{n_0} }_p + \norm{g}_p < +\infty.
		\]
		Teljesülnek a \hyperref[th:lebesgue]{Lebesgue-tétel} feltételei.
		Ekkor van olyan	\( f \in \Lebesgue[p] \), hogy
		\[
			f = \lim_{\ell \to \infty} f_{n_\ell} \quad \mu \text{-m.m.}
		\]
		Jól ismert, hogy amennyiben egy Cauchy-sorozatnak van konvergens részsorozata, 
		akkor maga a teljes sorozat is konvergens, továbbá a határértékeik azonosak.
		Ennél fogva az \( (f_n) \) függvénysorozat is konvergens és
		\[
			f = \lim_{n \to \infty} f_n \quad \mu \text{-m.m.}
		\]
		Eddig a pontig \( p \) értéke tetszőleges lehetett.
		
		
		\begin{enumerate}
			\item Ha \( p < +\infty \), 
			akkor szintén a \hyperref[th:lebesgue]{Lebesgue-tétel} alapján
			\[
				\lim_{n \to \infty} \norm{ f_n - f}_p = 0.
			\]
			
			\item Ha \( p = +\infty \),
			\marginnote{A \hyperref[th:lebesgue]{Lebesgue-tétel} nem alkalmazható!}
			akkor \eqref{eq:lp-teljes-cauchy-becslés} alapján tetszőleges \( m < k \) indexre
			\[
				\abs\big{ f_{n_m} - f_{n_k} } =
				\abs{ \sum_{i=m}^{k - 1} \bigl( f_{n_i} - f_{n_{i + 1}} \bigr) } \leq
				\sum_{i=m}^{k - 1} \norm\big{ f_{n_i} - f_{n_{i + 1}} }_\infty \leq
				\sum_{i=m}^{k - 1} \frac{1}{2^i}.
			\]
			Ekkor a határérték és a rendezés kapcsolata miatt
			\marginnote{
				Az itt szereplő mértani sorösszeg
				\[
					\sum_{i=m}^{\infty} \frac{1}{2^i} =
					\frac{1}{2^m} \cdot \frac{1}{1 - 1/2} = 
					\frac{2}{2^m}.
				\]
			}
			\[
				\abs\big{ f_{n_m} - f } =
				\lim_{k \to \infty} \abs\big{ f_{n_m} - f_{n_k} } \leq
				\sum_{i=m}^{\infty} \frac{1}{2^i} =
				2^{1-m}
				\quad (m \in \N).
			\]
			Tehát a végtelen-norma definíciója, majd a közrefogási elv-miatt
			\marginnote{
				Hiszen \( m \to \infty \) mellett
				\[
					0 \leq
					\abs\big{ f_{n_m} - f } \leq
					\norm\big{ f_{n_m} - f }_\infty \longrightarrow 0.
				\]
			}
			\[
				\lim_{m \to \infty} \norm{ f - f_{n_m} }_\infty =
				\lim_{n \to \infty} \norm{ f - f_n }_\infty =
				0.
			\]
		\end{enumerate}
		
	\end{proof}
	
\end{document} 