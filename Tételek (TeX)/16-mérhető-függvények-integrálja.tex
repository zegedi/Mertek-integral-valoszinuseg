\documentclass[
%parspace, % Add vertical space between paragraphs
%noindent, % No indentation of first lines in each paragraph
%nohyp,	   % No hyphenation of words
%twoside,  % Double sided format
%draft,    % Quicker draft compilation without rendering images
%final,    % Set final to hide todos
]{elteikthesis}[2024/04/26]


% The minted package is also supported for source highlighting
% See elteikthesis_minted.tex for example
%\usepackage[newfloat]{minted}
\usepackage{enumitem}

% Document's metadata
\title{Mérték, integrál, valószínűség} % title
\subtitle{16. Vizsgatétel}


% The document
\begin{document}
	
	% Set document language
	\documentlang{hungarian}
	
	\section{Emlékeztető}
	
	\section{Integrálható függvények}
	
	\begin{definition}{Pozitív rész, negatív rész}
		Legyen \( (X, \Omega, \mu) \) mértéktér,
		valamint  \( \func{f}{X}{\overline{ \R }} \) egy függvény.
		Ekkor
		\marginnote{
			Tehát a pozitív rész függvény röviden
			\[
				f^+ = f {\cdot} \chi_{\{ f > 0 \}}.
			\]
		}
		\[
			f^+ \coloneq f {\cdot} \chi_{\{ f > 0 \}}
		\]
%		\[
%			f^+(x) \coloneq
%			\left\{
%			\begin{aligned}
%				f(x), & \quad \text{ha } f(x) \geq 0, \\
%				0,    & \quad \text{ha } f(x) < 0
%			\end{aligned}
%			\right.
%		\]
		az \( f \) függvény \emph{pozitív része}, valamint
		\[
			f^- \coloneq -f {\cdot} \chi_{\{ f < 0 \}}
		\]
%		\[
%			f^-(x) \coloneq
%			\left\{
%			\begin{aligned}
%				-f(x), & \quad \text{ha } f(x) \leq 0, \\
%				 0,    & \quad \text{ha } f(x) > 0
%			\end{aligned}
%			\right.
%		\]
		az \( f \) függvény \emph{negatív része}.
	\end{definition}
	
	Világon, hogy a pozitív és negatív rész függvények nemnegatívok, valamint
	\[
		f = f^+ - f^-
		\qquad \text{és} \qquad
		\abs{f} = f^+ + f^-.
	\]
	Továbbá az is nyilvánvaló, hogyha az \( f \) mérhető függvény, 
	akkor \( f^+, f^- \in \StepFuncPlus \).
	Következésképpen léteznek a nemnegatív
	\[
		\int f^+ \dd{\mu}, \quad \int f^- \dd{\mu}
	\]
	integrálok, amelyek lehetnek akár \( +\infty \) is.
	
	\begin{definition}{Mérhető függvény integrálja}{}
		Legyen \( (X, \Omega, \mu) \) mértéktér, 
		valamint \( \func{f}{X}{\overline{ \R }} \) egy mérhető függvény.\\[3pt]
		Azt mondjuk, hogy létezik az \( f \) függvény \emph{integrálja}, amennyiben az
		\[
			\int f^+ \dd{\mu}, \ \int f^- \dd{\mu}
		\]
		integrálok közül legalább az egyik véges. Ekkor az
		\[
			\int f \dd{\mu} \coloneq
			\int f^+ \dd{\mu} - \int f^- \dd{\mu}
		\]
		érték az \( f \) függvény \( \mu \) mérték szerinti \emph{integrálja}.
	\end{definition}
	
	\noindent 
	Amennyiben az \( f \in \StepFuncPlus \), 
	akkor az \( \func{f}{X}{\overline{\R}} \) egy nemnegatív mérhető függvény.
	Ennél fogva a negatív része azonosan nulla, ahonnan \( \int f^- \dd{\mu} = 0 \) adódik.
	Tehát az \( f \) integrálja megegyezik az \( \StepFuncPlus \)-beli integráljával.
	
	A későbbiekben azok a függvények lesznek fontosak a számunkra, 
	amelyeknek van integrálja, és az véges. 
	Minden ilyen leképezést \emph{integrálható függvénynek} nevezünk. 
	Vezessük be ezzel kapcsolatban a következő jelölést:
	\[
		\Lebesgue \coloneq \Lebesgue(\mu) \coloneq
		\setc[\Bigg]{ \func{f}{X}{\overline{\R}} }{ f \text{ mérhető és } \int f \dd{\mu} \in \R}.
	\]
	\newpage
	
	\begin{theorem}{A Lebesgue-integrál alaptulajdonságai}{}
		Legyen \( (X,\Omega,\mu) \) mértéktér, 
		valamint \( f, g \in \Lebesgue \) és \( \alpha \in \R \).
		\begin{enumerate}
			\item 
			\( (\alpha \cdot f) \in \Lebesgue \) 
			\quad és \quad 
			\( \displaystyle \int (\alpha \cdot f) \dd{\mu} = \alpha \cdot \int f \dd{\mu} \).
		\end{enumerate}
	\end{theorem}
		
\end{document} 