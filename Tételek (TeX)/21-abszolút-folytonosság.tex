\documentclass[
%parspace, % Add vertical space between paragraphs
%noindent, % No indentation of first lines in each paragraph
%nohyp,	   % No hyphenation of words
%twoside,  % Double sided format
%draft,    % Quicker draft compilation without rendering images
%final,    % Set final to hide todos
]{elteikthesis}[2024/04/26]


% The minted package is also supported for source highlighting
% See elteikthesis_minted.tex for example
%\usepackage[newfloat]{minted}
\usepackage{enumitem}

% Document's metadata
\title{Mérték, integrál, valószínűség} % title
\subtitle{20. Vizsgatétel}


% The document
\begin{document}
	
	% Set document language
	\documentlang{hungarian}
	
	\section{Emlékeztető}
	
	\begin{definition}{Súlyfüggvény}{}
		Legyen \( (X, \Omega, \mu) \) mértéktér, 
		\( f \in \StepFuncPlus \) egy adott függvény. Ekkor a
		\[
			\func{\mu_f}{\Omega}{[0, +\infty]}, \qquad 
			\mu_f(A) \coloneq 
			\int_A f \dd{\mu} \coloneq 
			\int f {\cdot} \chi_A \dd{\mu}
		\]
		leképezést \emph{súlyfüggvénynek} nevezzük.
	\end{definition}
	
	\begin{note}
		Amennyiben az \( A \in \Omega \) halmaz nullamértékű, akkor
		\[
			f {\cdot} \chi_A = 0 \ \mu \text{-m.m.}
			\quad \Longrightarrow \quad
			\mu_f(A) = 
			\int f {\cdot} \chi_A \dd{\mu} =
			0.
		\]
	\end{note}
	
	\begin{statement}{}{}
		Legyen \( (X, \Omega, \mu) \) mértéktér, \( f \in \StepFuncPlus \).
		Ekkor a \( \mu_f \) súlyfüggvény mérték.
	\end{statement}
	
	\section{Abszolút folytonosság}
	
	\begin{definition}{}{}
		Legyen \( (X, \Omega) \) mérhető tér, 
		valamint \( \func{\mu, \nu}{\Omega}{[0, +\infty]} \) két mérték.\\[6pt]
		Azt mondjuk, 
		hogy \( \nu \) \emph{abszolút folytonos} \( \mu \)-re nézve (jelben \( \nu \ll \mu \)), ha
		\[
			\mu(A) = 0
			\quad \Longrightarrow \quad
			\nu(A) = 0
			\qquad (A \in \Omega).
		\]
	\end{definition}
	
	\begin{lemma}{}{}
		Legyen \( (X, \Omega) \) mérhető tér, 
		\( \func{\mu, \nu}{\Omega}{[0, +\infty]} \) két mérték, ahol \( \nu \) véges.\\[6pt]
		Ekkor \( \nu \ll \mu \) azzal ekvivalens, hogy
		bármely \( \varepsilon > 0 \)-hoz van olyan \( \delta > 0 \):
		\[
%			\forall \varepsilon > 0 \text{-hoz},\
%			\exists \delta > 0,
%			\forall A \in \Omega \colon \
%			\mu(A) < \delta \quad \longrightarrow \quad \nu(A) < \delta.
			\mu(A) < \delta
			\quad \Longrightarrow \quad
			\nu(A) < \varepsilon
			\qquad (A \in \Omega).
		\]
	\end{lemma}
	\begin{proof}\,\\
		\Ifstep
		Indirekt tegyük fel, hogy megadható olyan \( \varepsilon > 0 \) szám, amellyel
		\[
			X_n \in \Omega, \quad 
			\mu(X_n) < \frac{1}{2^n} 
			\quad \text{de} \quad
			\nu( X_n ) \geq \varepsilon \quad (n \in \N).
		\]
		Definiáljuk a soron következő halmazokat:
		\[
			A_n \coloneq \bigcup_{k=n}^{\infty} X_k 
			\qquad \text{és} \qquad
			A \coloneq \bigcap_{k=0}^{\infty} A_k
			\qquad (n \in \N)
		\]
		Ekkor az \( (A_n) \) halmazsorozat monoton szűkülve tart az \( A \)-hoz, ezért
		\[
			\mu( A ) \leq 
			\mu( A_n ) \leq 
			\sum_{k=n}^{\infty} \mu( X_n ) \leq
			\sum_{k=n}^{\infty} \frac{1}{2^k}
		%	= \frac{2}{2^n}
			\longrightarrow 0
			\qquad (n \to \infty).
		\]
		Tehát \( \mu(A) = 0 \), vagyis az abszolút folytonosság miatt \( \nu(A) = 0 \).
		Viszont a \( \nu \) véges mérték és az \( (A_n) \) sorozat monoton szűkülve tart az \( A \)-hoz. Ezért
		\[
			\nu(A) = 
			\lim_{n \to \infty} \nu(A_n) \geq 
			\lim_{n \to \infty} \nu(X_n) \geq 
			\varepsilon > 0.
		\]
		Ez viszont ellentmondás.
		
		\Backifstep Triviális, ha ugyanis egy \( A \in \Omega \) halmazra \( \mu(A) = 0 \), 
		akkor a feltevés szerint tetszőleges \( \varepsilon > 0 \)-hoz megadható olyan \( \delta > 0 \), amivel
		\[
			\mu(A) < \delta
			\quad \Longrightarrow \quad
			0 \leq \nu(A) < \varepsilon.
		\]
		Mivel itt az \( \varepsilon \) tetszőleges lehet, ezért \( \nu(A) = 0 \).
	\end{proof}
	
\end{document} 