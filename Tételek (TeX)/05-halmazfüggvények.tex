\documentclass[
%parspace, % Add vertical space between paragraphs
%noindent, % No indentation of first lines in each paragraph
%nohyp,	   % No hyphenation of words
%twoside,  % Double sided format
%draft,    % Quicker draft compilation without rendering images
%final,    % Set final to hide todos
]{elteikthesis}[2024/04/26]


% The minted package is also supported for source highlighting
% See elteikthesis_minted.tex for example
%\usepackage[newfloat]{minted}
\usepackage{enumitem}

% Document's metadata
\title{Mérték, integrál, valószínűség} % title
\subtitle{\circled{5} Vizsgatétel}


% The document
\begin{document}
	
	% Set document language
	\documentlang{hungarian}
	
	\section{Halmazfüggvények}
	
	Az (absztrakt) halmazok mérését (a mértéküknek az értelmezését) egy
	\[
		\funcin{\varphi}{\powerset{X}}{[0, +\infty]}
	\]
	függvény segítségével végezzük majd, ahol az \( X \) adott alaphalmaz.
	
	\begin{definition}{Véges- és szigma-additív halmazfüggvény}{szigma-additiv}
		Azt mondjuk, hogy a \( \funcin{\varphi}{\powerset{X}}{[0, +\infty]} \) halmazfüggvény
		\begin{enumerate}
			\item \emph{(véges) additív}, ha
			\[
				\varphi \Biggl(\, \bigcup_{k=0}^n A_k \Biggr) = \sum_{k=0}^n \varphi(A_k)
			\]
			minden olyan \( A_0, \dots, A_n \in \dom{\varphi} \) páronként diszjunkt halmazrendszerre fennáll, amelynek az egyesítésére
			\( A_0 \cup \cdots \cup A_n \in \dom{\varphi} \) teljesül;
			
			\item \emph{szigma-additív} (\( \sigma \)-additív), ha
			\[
				\varphi \Biggl(\, \bigcup_{n=0}^\infty A_n \Biggr) = 
				\sum_{n=0}^\infty \varphi(A_n)
			\]
			minden olyan \( A_n \in \dom{\varphi} \ (n \in \N) \) páronként diszjunkt halmazsorozatra fennáll, amelynek az egyesítésére 
			\( \bigcup\limits_{n \in \N} A_n \in \dom{\varphi} \) teljesül.
		\end{enumerate}
	\end{definition}
	
	\begin{stat*}
		Legyen \( \varphi \) egy additív halmazfüggvény, 
		amire \( \emptyset \in \dom{\varphi} \) fennáll.
		\begin{enumerate}
			\item Ha \( \varphi \) additív és \( \varphi(\emptyset) \) véges, 
			akkor \( \varphi(\emptyset) = 0 \).
			
			\item Ha \( \varphi \) szigma-additív és \( \varphi(\emptyset) \) véges, 
			akkor \( \varphi(\emptyset) = 0 \) és \( \varphi \) additív is.
		\end{enumerate}
	\end{stat*}
	\begin{proof*}\,
		\begin{enumerate}
			\item Mivel \( \emptyset = \emptyset \cup \emptyset = \emptyset \cap \emptyset \), ezért alkalmazhatjuk a \( \varphi \) additív tulajdonságát
			\[
				\varphi(\emptyset) = 
				\varphi(\emptyset) + \varphi(\emptyset)
				\qquad \Longrightarrow \qquad
				\varphi(\emptyset) = 0.
			\]
			
			\item Amennyiben \( \varphi \) szigma-additív, akkor a
			\[
				\varphi(\emptyset) =
				\varphi \Biggl(\, \bigcup_{n=0}^\infty \emptyset \Biggr) =
				\sum_{n=0}^{\infty} \varphi(\emptyset)
			\]
			sorösszeg pontosan akkor lesz véges, ha \( \varphi(\emptyset) = 0 \).
			Továbbá, amennyiben az
			\[
				A_0, \dots, A_n \in \dom{\varphi} \quad (n \in \N)
			\]
			páronként diszjunkt halmazrendszert, 
			akkor az \( A_k \coloneq \emptyset \ (k > n) \) választással
			\[
				\varphi \Biggl(\, \bigcup_{k=0}^n        A_k \Biggr) =
				\varphi \Biggl(\, \bigcup_{k=0}^{\infty} A_k \Biggr) =
				\sum_{n=0}^{\infty} \varphi(A_k) =
				\sum_{n=0}^n        \varphi(A_k).
			\]
			Ezt pedig azt jelenti, hogy \( \varphi \) véges additív. \qed
		\end{enumerate}
	\end{proof*}
	
	\begin{definition}{Mérték, kvázimérték, előmérték}{}
		Azt mondjuk, hogy a \( \funcin{\mu}{\powerset{X}}{[0, +\infty]} \) halmazfüggvény egy
		\begin{enumerate}
			\item \emph{mérték}, ha \( \dom{\mu} \) szigma-algebra, \( \mu(\emptyset) = 0 \), és a \( \mu \) szigma-additív;
			
			\item \emph{kvázimérték}, ha \( \dom{\mu} \) halmazgyűrű, 
			\( \mu(\emptyset) = 0 \), és a \( \mu \) szigma-additív;
			
			\item \emph{előmérték}, ha \( \dom{\mu} \) halmazgyűrű, 
			\( \mu(\emptyset) = 0 \), és a \( \mu \) additív.
		\end{enumerate}
	\end{definition}
		
	\begin{theorem}{Az előmérték tulajdonságai}{előmérték-tulajdonságok}
		Legyen \( \mu \) előmérték a \( \ring{G} \subseteq \powerset{X} \) gyűrűn,
		\marginnote{Tehát \( \func{\mu}{\ring{G}}{[0, +\infty]} \) egy előmérték.}
		továbbá \( A, B, A_n \in \ring{G} \ (n \in \N) \).
		
		\begin{enumerate}[label=\arabic*.]
			\item\label{th:előmérték-monoton}
			Ha \( B \subseteq A \), akkor \( \mu(B) \leq \mu(A) \).
			\marginnote{\ref{th:előmérték-monoton} tulajdonság: \emph{\( \boldsymbol{\mu} \) monoton növő}.}
			
			\item\label{th:előmérték-különbség-mértéke}
			Ha \( B \subseteq A \) és  \( \mu(B) \) véges,
			akkor \( \mu(A \setminus B) = \mu(A) - \mu(B) \).
			\marginnote{\ref{th:előmérték-különbség-mértéke} tulajdonság: \emph{\( \boldsymbol{\mu} \) szubtraktív}.}
			
			\item\label{th:előmérték-szita-formula}
			\( \mu(A \cup B) + \mu(A \cap B) = \mu(A) + \mu(B) \).
			\marginnote{\ref{th:előmérték-szita-formula} tulajdonság: \emph{szita-formula}.}
			
			\item\label{th:előmérték-véges-szubadditív}
			Minden \( n \in \N \) indexre 
			\( \mu \biggl(\, \bigcup\limits_{k=0}^n \! A_k \biggr) \leq \sum\limits_{k=0}^n \mu( A_k ) \).
			\marginnote{\ref{th:előmérték-véges-szubadditív} tulajdonság: \emph{\( \boldsymbol{\mu} \) szubadditív}.}
			
			\item\label{th:előmérték-szubadditívitás-kiterjesztése}
			Ha az \( (A_n) \) tagjai páronként diszjunktak
			és \( \bigcup\limits_{n=0}^{\infty} \! A_n \in \ring{G} \), akkor
			\[
				\sum\limits_{n=0}^{\infty} \mu( A_n ) \leq
				\mu \Biggl(\, \bigcup\limits_{n=0}^{\infty} \! A_n \Biggr).
			\]
		\end{enumerate}
	\end{theorem}
	\begin{proof}\,
		\begin{enumerate}
			\item Mivel \( A = B \cup (A \setminus B) \) diszjunkt felbontás és \( \mu \) nemnegatív, ezért
			\[
				\mu(A) = \mu(B) + \mu(A \setminus B) \geq \mu(B). 
				\tag{\( * \)}\label{eq:előmérték-monoton}
			\]
			
			\item Ha \( \mu(B) \) véges, akkor \eqref{eq:előmérték-monoton} átrendezésével adódik a belátandó állítás.
			
			\item Két esetet különböztetünk meg.
			\begin{enumerate}
				\item 
				Amennyiben \( \mu(A \cap B) = +\infty \), 
				\marginnote{
					Hiszen elmondható, hogy
					\[
						A \cap B \subseteq A, B \subseteq A \cup B.
					\]
				}
				akkor a \hyperref[th:előmérték-monoton]{\( \mu \) monotonitása} miatt
				\[
					\mu(A \cap B) = \mu(A \cup B) = \mu(A) = \mu(B) = +\infty.\ \checkmark
				\]
				
				\item Ha most \( \mu(A \cap B) \) véges, 
				akkor az \( A \cup B = A \cup \bigl( B \setminus (A \cap B) \bigr) \) diszjunkt felbontás és \( \mu \) additivitása miatt
				\[
					\mu(A \cup B) = 
					\mu(A) + \mu \bigl( B \setminus (A \cap B) \bigr) \overset{\ref{th:előmérték-különbség-mértéke}}{=} 
					\mu(A) + \mu(B) - \mu( A \cap B).\ \checkmark
				\]
			\end{enumerate}
			
			\item Az állítás teljes indukcióval igazolható, 
			lásd \hyperref[th:előmérték-szita-formula]{szita-formula}.
			
			\item Mivel \( A_0, \dots, A_n \ (n \in \N) \) páronként diszjunktak 
			és \( \mu \) additív, ezért
			\[
				\sum_{n=0}^{\infty} \mu( A_n ) =
				\lim_{n \to \infty} \sum_{k=0}^{n} \mu( A_k ) =
				\lim_{n \to \infty} \mu \Biggl( \, \bigcup_{k=0}^n \! A_k \Biggr).
			\]
			Ugyanakkor az
			\( \bigcup\limits_{k=0}^n \! A_k \subseteq \bigcup\limits_{k=0}^{\infty} \! A_k \)
			tartalmazás és \hyperref[th:előmérték-monoton]{\( \mu \) monotonitása} miatt
			\[
				\sum_{n=0}^{\infty} \mu( A_n ) =
				\lim_{n \to \infty} \mu \Biggl( \, \bigcup_{k=0}^n \! A_k \Biggr) \leq
				\lim_{n \to \infty} \mu \Biggl( \, \bigcup_{k=0}^{\infty} \! A_k \Biggr) =
				\mu \Biggl( \, \bigcup_{n=0}^{\infty} \! A_n \Biggr).
			\]
		\end{enumerate}
	\end{proof}
	
	\section{Kiterjesztések}
	
	A későbbiek során megmutatjuk, 
	hogy minden gyűrűn értelmezett előmértéket ki lehet kiterjeszteni kvázimértékké, majd azt követően mértékké. 
	Először is belátjuk, hogy tetszőleges félgyűrűn értelmezett nemnegatív, 
	additív és az üres halmazhoz nullát rendelő halmazfüggvényt ki tudjuk terjeszteni előmértékké.
	
	\begin{lemma}{}{halmazfüggvény-félgyűrűn}
		Legyen \( \ring{H} \subseteq \powerset{X} \) félgyűrű, 
		továbbá \( \func{m}{\ring{H}}{[0, +\infty]} \) halmazfüggvény és
		%
		\begin{enumerate}[label=\roman*)]
			\item az \( m \) additív és \( m(\emptyset) = 0 \);
			\item \( n \in \N, \ H_0, \dots, H_n \in \ring{H} \) páronként diszjunktak;
			\item \( s \in \N, \ Q_0, \dots, Q_s \in \ring{H} \) páronként diszjunktak.
		\end{enumerate}
		%
		Ekkor
		\[
			\bigcup_{k=0}^n \! H_k = \bigcup_{\ell=0}^s Q_\ell
			\qquad \Longrightarrow \qquad
			\sum_{k=0}^{n} m( H_k ) = \sum_{\ell=0}^{s} m( Q_\ell ).
		\]
	\end{lemma}
	\begin{proof}
		Mivel a metszetképzés disztributív az unióra, ezért
		\begin{alignat*}{5}
			H_k &= 
			H_k \cap \Biggl( \, \bigcup_{\ell=0}^s Q_\ell \Biggr) &&=
			\bigcup_{\ell=0}^s \bigl( H_k \cap Q_\ell \bigr) \qquad &&(k=0,\dots,n) \\[6pt]
			Q_\ell &= 
			Q_\ell \cap \Biggl( \, \bigcup_{k=0}^n H_k \Biggr) &&=
			\bigcup_{k=0}^n \bigl( Q_\ell \cap H_k \bigr) &&(\ell=0,\dots,s)
		\end{alignat*}
		páronként diszjunkt halmazrendszerek, ezért az \( m \) additivitása miatt
		\[
			\sum_{k=0}^n m( H_k ) =
		%	\sum_{k=0}^n m \Biggl( \bigcup_{\ell=0}^s \bigl( H_k \cap Q_\ell \bigr) \Biggr) =
			\sum_{k=0}^n \sum_{\ell=0}^s m \bigl( H_k \cap Q_\ell \bigr) =
			\sum_{\ell=0}^s \sum_{k=0}^n  m \bigl( Q_\ell \cap H_k \bigr) =
			\sum_{\ell=0}^s m( Q_\ell ).
		\]
	\end{proof}
	
	\noindent
	\Aref{lem:halmazfüggvény-félgyűrűn} lemma jogalapjául az alábbi lemma szolgál.
	
	\begin{lemma}{Félgyűrűvel generált halmazgyűrű szerkezete}{félgyűrűvel-generált-gyűrű}
		Amennyiben a \( \ring{H} \subseteq \powerset{X} \) egy félgyűrű, 
		akkor az általa generált gyűrű
		\[
			\mathcal{G}(\ring{H}) = 
			\setc[\Bigg]{ \bigcup_{k=0}^n H_k }
			            { H_0, \dots, H_n \in \ring{H} \text{ páronként diszjunktak} \ (n \in \N)}.
		\]
	\end{lemma}
	
	\noindent
	\Aref{lem:félgyűrűvel-generált-gyűrű} lemma értelmében a továbbiakban feltesszük, hogy
	\[
		A \in \mathcal{G}(\ring{H})
		\quad \iff \quad
		A = \bigcup_{k=0}^{n} Q_k
	\]
	fennáll valamilyen \( n \in \N \) és \( Q_0, \dots, Q_n \in \ring{H} \) páronként diszjunkt halmazok esetén.
		
	\newpage
	\begin{theorem}{}{}		
		Legyen \( \ring{H} \subseteq \powerset{X} \) félgyűrű, 
		továbbá \( \func{m}{\ring{H}}{[0, +\infty]} \) additív és \( m(\emptyset) = 0 \).\\[6pt]
		Definiáljuk az alábbi halmazfüggvényt:
		\[
			\func{\mu}{\mathcal{G}(\ring{H})}{[0, +\infty]}, \qquad
			\mu(A) \coloneq \sum_{k=0}^{n} m( Q_k ).
		\]
		Ekkor
		\begin{enumerate}
			\item \( \mu \) előmérték, valamint \( \restr{\mu}{\ring{H}} = m \);
			
			\item ha \( \lambda \) előmérték \( \mathcal{G}(\ring{H}) \)-n
			\marginnote{
				Tehát \( \func{\lambda}{\mathcal{G}(\ring{H})}{[0, +\infty]} \) alakú.
			}
			és \( \restr{\lambda}{\ring{H}} = m \), akkor \( \lambda = \mu \);
			
			\item ha \( m \) szigma-additív, akkor \( \mu \) kvázi-mérték.
		\end{enumerate}
	\end{theorem}
	\begin{proof}\,
		\begin{enumerate}
			\item Az állítás nyilvánvalóan igaz, lásd \ref{lem:halmazfüggvény-félgyűrűn} lemma.
			
			\item Az lemma felhasználásával 
			
			\item Legyenek \( B_n \in \mathcal{G}(\ring{H}) \) páronként diszjunkt halmazok 
			\marginnote{Igazolni kell, hogy \( \mu \) szigma-additív.}
			\( (n \in \N) \) és
			\[
				B \coloneq \bigcup_{n=0}^{\infty} B_n \in \mathcal{G}(\ring{H}).
			\]
			Ekkor alkalmas \( H_0, \dots, H_s \in \ring{H} \) páronként diszjunkt halmazokkal
			\marginnote{
				Lásd \hyperref[lem:félgyűrűvel-generált-gyűrű]{félgyűrűvel generált gyűrű}.
			}
			\[
				B = \bigcup_{k=0}^{s} H_k
			\]
			Ugyan ez elmondható a \( B_n \) halmazokra is, vagyis
			\marginnote{
				Az eddigieket összevetve kapjuk, hogy
				\[
					B = \bigcup_{n=0}^{\infty} \bigcup_{j=0}^{p_n} H_{nj}.
				\]
			}
			\[
				B_n = \bigcup_{j=0}^{p_n} H_{nj} \quad (n \in \N)
			\]
			valamilyen \( H_{n0}, \dots, H_{np_n} \in \ring{H} \) páronként diszjunkt halmazokkal.
			\marginnote{
				Ugyanis minden \( k = 0,\dots,s \) indexre
				\[
					H_k = 
					H_k \cap B =
				%	H_k \cap \bigcup_{n=0}^{\infty} \bigcup_{j=0}^{p_n} H_{nj}.
					\bigcup_{n=0}^{\infty} \bigcup_{j=0}^{p_n} (H_{nj} \cap H_k).
				\]
				Ez pedig páronként diszjunkt felbontás.
			}
			Így
			\begin{align*}
				\mu(B)
				 = \sum_{k=0}^{s} m( H_k )
				&= \sum_{k=0}^{s} \sum_{n=0}^{\infty} \sum_{j=0}^{p_n} m( H_{nj} \cap H_k ) \\
				&= \sum_{n=0}^{\infty} \sum_{j=0}^{p_n} \sum_{k=0}^{s} m( H_{nj} \cap H_k ) 
				%= \sum_{n=0}^{\infty} m( B_n )
				 = \sum_{n=0}^{\infty} \mu( B_n ).
			\end{align*}
		\end{enumerate}
	\end{proof}
	
\end{document}
