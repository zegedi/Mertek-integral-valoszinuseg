\documentclass[
%parspace, % Add vertical space between paragraphs
%noindent, % No indentation of first lines in each paragraph
%nohyp,	   % No hyphenation of words
%twoside,  % Double sided format
%draft,    % Quicker draft compilation without rendering images
%final,    % Set final to hide todos
]{elteikthesis}[2024/04/26]


% The minted package is also supported for source highlighting
% See elteikthesis_minted.tex for example
%\usepackage[newfloat]{minted}
\usepackage{enumitem}

\makeatletter
\def\biggg{\bBigg@{3}}
\def\bigggm{\mathrel\biggg}
\def\bigggl{\mathopen\biggg}
\def\bigggr{\mathclose\biggg}
\def\Biggg{\bBigg@{3.5}}
\def\Bigggm{\mathrel\Biggg}
\def\Bigggl{\mathopen\Biggg}
\def\Bigggr{\mathclose\Biggg}
\makeatother

% Document's metadata
\title{Mérték, integrál, valószínűség} % title
\date{2024} % year of defense

% Author's metadata
\author{Egedi Viktor}
\degree{programtervező informatikus BSc}

% Superivsor(s)' metadata
\supervisor{Dr. Simon Péter} % internal supervisor's name
\affiliation{egyetemi tanár} % internal supervisor's affiliation

% University's metadata
\university{Eötvös Loránd Tudományegyetem} % university's name
\faculty{Informatikai Kar} % faculty's name
\department{Numerikus Analízis Tanszék} % department's name
\city{Budapest} % city
\logo{elte_cimer_szines} % logo

% The document
\begin{document}
	
	% Set document language
	\documentlang{hungarian}
	
	Az (absztrakt) halmazok mérését (a mértéküknek az értelmezését) egy
	\[
		\funcin{\varphi}{\powerset{X}}{[0, +\infty]}
	\]
	függvény segítségével végezzük majd, ahol az \( X \) adott alaphalmaz.
	Il
	
	\begin{definition}{Véges- és szigma-additív halmazfüggvény}{}
		Azt mondjuk, hogy a \( \funcin{\varphi}{\powerset{X}}{[0, +\infty]} \) halmazfüggvény
		\begin{enumerate}
			\item \emph{(véges) additív}, ha
			\[
				\varphi \Biggl(\, \bigcup_{k=0}^n A_k \Biggr) = \sum_{k=0}^n \varphi(A_k)
			\]
			minden olyan \( A_0, \dots, A_n \in \dom{\varphi} \) páronként diszjunkt halmazrendszerre fennáll, amelynek az egyesítésére
			\( A_0 \cup \cdots \cup A_n \in \dom{\varphi} \) teljesül;
			
			\item \emph{szigma-additív} (\( \sigma \)-additív), ha
			\[
				\varphi \Biggl(\, \bigcup_{n=0}^\infty A_n \Biggr) = 
				\sum_{n=0}^\infty \varphi(A_n)
			\]
			minden olyan \( A_n \in \dom{\varphi} \ (n \in \N) \) páronként diszjunkt halmazsorozatra fennáll, amelynek az egyesítésére 
			\( \bigcup\limits_{n=0}^{\infty} A_n \in \dom{\varphi} \) teljesül.
		\end{enumerate}
	\end{definition}
	
	\begin{stat*}
		Legyen \( \varphi \) egy additív halmazfüggvény, 
		amire \( \emptyset \in \dom{\varphi} \) fennáll.
		\begin{enumerate}
			\item Ha \( \varphi \) additív és \( \varphi(\emptyset) \) véges, 
			akkor \( \varphi(\emptyset) = 0 \).
			
			\item Ha \( \varphi \) szigma-additív és \( \varphi(\emptyset) \) véges, 
			akkor \( \varphi(\emptyset) = 0 \) és \( \varphi \) additív is.
		\end{enumerate}
	\end{stat*}
	\begin{proof*}\,
		\begin{enumerate}
			\item Mivel \( \emptyset = \emptyset \cup \emptyset = \emptyset \cap \emptyset \), ezért alkalmazhatjuk a \( \varphi \) additív tulajdonságát
			\[
				\varphi(\emptyset) = 
				\varphi(\emptyset) + \varphi(\emptyset)
				\qquad \Longrightarrow \qquad
				\varphi(\emptyset) = 0.
			\]
			
			\item Amennyiben \( \varphi \) szigma-additív, akkor a
			\[
				\varphi(\emptyset) =
				\varphi \Biggl(\, \bigcup_{n=0}^\infty \emptyset \Biggr) =
				\sum_{n=0}^{\infty} \varphi(\emptyset)
			\]
			sorösszeg pontosan akkor lesz véges, ha \( \varphi(\emptyset) = 0 \).
			Ezért bárhogyan véve egy véges \( A_0, \dots, A_n \in \dom{\varphi}\ (n \in \N) \) és páronként diszjunkt halmazrendszert, akkor az 
			\( A_k \coloneq \emptyset \ (k > n) \) választás mellett
			\[
				\varphi \Biggl(\, \bigcup_{k=0}^n        A_k \Biggr) =
				\varphi \Biggl(\, \bigcup_{k=0}^{\infty} A_k \Biggr) =
				\sum_{n=0}^{\infty} \varphi(A_k) =
				\sum_{n=0}^n        \varphi(A_k).
			\]
			Ezt pedig azt jelenti, hogy \( \varphi \) additív.
		\end{enumerate}
	\end{proof*}
	
	\begin{definition}{Mérték, kvázimérték, előmérték}{}
		Azt mondjuk, hogy a \( \funcin{\mu}{\powerset{X}}{[0, +\infty]} \) halmazfüggvény egy
		\begin{enumerate}
			\item \emph{mérték}, ha \( \dom{\mu} \) szigma-algebra, \( \mu(\emptyset) = 0 \), és a \( \mu \) szigma-additív;
			
			\item \emph{kvázimérték}, ha \( \dom{\mu} \) halmazgyűrű, 
			\( \mu(\emptyset) = 0 \), és a \( \mu \) szigma-additív;
			
			\item \emph{előmérték}, ha \( \dom{\mu} \) halmazgyűrű, 
			\( \mu(\emptyset) = 0 \), és a \( \mu \) additív.
		\end{enumerate}
	\end{definition}
	
	\newpage
	
	\begin{theorem}{Az előmérték tulajdonságai}{}
		Legyen \( \mu \) előmérték a \( \ring{G} \subseteq \powerset{X} \) gyűrűn,
		\marginnote{Tehát \( \func{\mu}{\ring{G}}{[0, +\infty]} \) egy előmérték.}
		továbbá \( A, B, A_n \in \ring{G} \ (n \in \N) \).
		
		\begin{enumerate}
			\item\label{th:előmérték-monoton}
			\( \mu \) monoton, azaz \( B \subseteq A \) esetén \( \mu(B) \leq \mu(A) \).
			
			\item\label{th:előmérték-szita-formula}
			\( \mu(A \cup B) + \mu(A \cap B) = \mu(A) + \mu(B) \).
			
			\item\label{th:előmérték-különbség-mértéke}
			Ha \( \mu(B) \) véges és \( B \subseteq A \),
			akkor \( \mu(A \setminus B) = \mu(A) - \mu(B) \).
			
			\item\label{th:előmérték-véges-szubadditív}
			Minden \( n \in \N \) indexre 
			\( \mu \biggl(\, \bigcup\limits_{k=0}^n \! A_k \biggr) \leq \sum\limits_{k=0}^n \mu( A_k ) \).
			
			\item\label{th:előmérték-szubadditívitás-kiterjesztése}
			Ha az \( (A_n) \) tagjai páronként diszjunktak
			és \( \bigcup\limits_{n=0}^{\infty} \! A_n \in \ring{G} \), akkor
			\[
				\mu \Biggl(\, \bigcup\limits_{n=0}^{\infty} \! A_n \Biggr) \leq 
				\sum\limits_{n=0}^{\infty} \mu( A_n ).
			\]
		\end{enumerate}
	\end{theorem}
	\begin{proof}\,
		\begin{enumerate}
			\item Mivel \( A = B \cup (A \setminus B) \) diszjunkt felbontás és \( \mu \) nemnegatív, ezért
			\[
				\mu(A) = \mu(B) + \mu(A \setminus B) \geq \mu(B). 
				\tag{\( * \)}\label{eq:előmérték-monoton}
			\]
			
			\item Ha \( \mu(B) \) véges, akkor \eqref{eq:előmérték-monoton} átrendezésével adódik a belátandó állítás.
			
			\item Két esetet különböztetünk meg.
			\begin{enumerate}
				\item Amennyiben \( \mu(A \cap B) = +\infty \), 
				akkor az \( A \cap B \subseteq A, B \subseteq A \cup B \) tartalmazások,
				valamint a \hyperref[th:előmérték-monoton]{\( \mu \) monotonitása} alapján
				\[
					\mu(A \cap B) = \mu(A \cup B) = \mu(A) = \mu(B) = +\infty.\ \checkmark
				\]
				
				\item Ha most \( \mu(A \cap B) \) véges, 
				akkor az \( A \cup B = A \cup \bigl( B \setminus (A \cap B) \bigr) \) diszjunkt felbontás és \( \mu \) additivitása miatt
				\[
					\mu(A \cup B) = 
					\mu(A) + \mu \bigl( B \setminus (A \cap B) \bigr) \overset{\ref{th:előmérték-különbség-mértéke}}{=} 
					\mu(A) + \mu(B) - \mu( A \cap B).\ \checkmark
				\]
			\end{enumerate}
			
			\item Az állítás teljes indukcióval igazolható, 
			felhasználva \ref{th:előmérték-szita-formula}-t.
			
			\item Mivel \( A_0, \dots, A_n \ (n \in \N) \) páronként diszjunktak 
			és \( \mu \) additív, ezért
			\[
				\sum_{n=0}^{\infty} \mu( A_n ) =
				\lim_{n \to \infty} \sum_{k=0}^{n} \mu( A_k ) =
				\lim_{n \to \infty} \mu \Biggl( \, \bigcup_{k=0}^n \! A_k \Biggr).
			\]
			Ugyanakkor az
			\( \bigcup\limits_{k=0}^n \! A_k \subseteq \bigcup\limits_{k=0}^{\infty} \! A_k \)
			tartalmazás és \hyperref[th:előmérték-monoton]{\( \mu \) monotonitása} miatt
			\[
				\sum_{n=0}^{\infty} \mu( A_n ) =
				\lim_{n \to \infty} \mu \Biggl( \, \bigcup_{k=0}^n \! A_k \Biggr) \leq
				\lim_{n \to \infty} \mu \Biggl( \, \bigcup_{k=0}^{\infty} \! A_k \Biggr) =
				\mu \Biggl( \, \bigcup_{n=0}^{\infty} \! A_n \Biggr).
			\]
		\end{enumerate}
	\end{proof}
	
	\begin{theorem}{Kvázimértékek ekvivalens jellemzése}{}
		Legyen \( \mu \) egy előmérték a \( \ring{G} \subseteq \powerset{X} \) gyűrűn,
		és vegyük az alábbi állításokat.
		%
		\begin{enumerate}[label=\alph*)]
			\item\label{eq:kvázimérték-jellemzése-01} A \( \mu \) kvázimérték.
			
			\item\label{eq:kvázimérték-jellemzése-02} 
			Minden \( \ring{G} \)-beli \( A_n \subseteq A_{n + 1} \ (n \in \N) \) 
			monoton bővülő halmazsorozatra
			\[
				A \coloneq \bigcup_{n=0}^{\infty} A_n \in \ring{G}
				\qquad \Longrightarrow \qquad
				\mu(A) = \lim_{n \to \infty} \mu(A_n).
			\]

			\item\label{eq:kvázimérték-jellemzése-03} 
			Minden \( \ring{G} \)-beli \( B_{n + 1} \subseteq B_n \ (n \in \N) \)
			halmazsorozatra, ha \( \mu(B_n) < +\infty \)
			\[
				B \coloneq \bigcap_{n=0}^{\infty} B_n \in \ring{G}
				\qquad \Longrightarrow \qquad
				\mu(B) = \lim_{n \to \infty} \mu(B_n).
			\]

			\item\phantomsection\label{eq:kvázimérték-jellemzése-04} 
			Minden \( \ring{G} \)-beli \( C_{n + 1} \subseteq C_n \ (n \in \N) \)
			halmazsorozatra, ha \( \mu(C_n) < +\infty \)
			\[
				\emptyset = \bigcap_{n=0}^{\infty} C_n \in \ring{G}
				\qquad \Longrightarrow \qquad
				\mu(\emptyset) = \lim_{n \to \infty} \mu(C_n) = 0.
			\]
		\end{enumerate}
		%	
		%			\begin{alignat*}{4}
			%				A &\coloneq \bigcup_{n=0}^{\infty} A_n \in \ring{G}
			%				\qquad &&\Longrightarrow \qquad
			%				& \mu(A) &= \lim_{n \to \infty} \mu(A_n).
			%			%
			%				\intertext{%
				%					\item\label{eq:kvázimérték-jellemzése-03} 
				%					Minden \( \ring{G} \)-beli \( B_{n + 1} \subseteq B_n \ (n \in \N) \)
				%					halmazsorozatra, ha \( \mu(B_n) < +\infty \)}
			%			%
			%				B &\coloneq \bigcap_{n=0}^{\infty} B_n \in \ring{G}
			%				\qquad &&\Longrightarrow \qquad
			%				& \mu(B) &= \lim_{n \to \infty} \mu(B_n).
			%			%
			%				\intertext{%
				%					\item\phantomsection\label{eq:kvázimérték-jellemzése-04} 
				%					Minden \( \ring{G} \)-beli \( C_{n + 1} \subseteq C_n \ (n \in \N) \)
				%					halmazsorozatra, ha \( \mu(C_n) < +\infty \)}
			%			%
			%				\emptyset &= \bigcap_{n=0}^{\infty} C_n \in \ring{G}
			%				\qquad &&\Longrightarrow \qquad
			%				& \mu(\emptyset) &= \lim_{n \to \infty} \mu(C_n) = 0.
			%			\end{alignat*}
		Ekkor
		\begin{enumerate}
			\item 
			\ref{eq:kvázimérték-jellemzése-01} \( \Longleftrightarrow \)
			\ref{eq:kvázimérték-jellemzése-02} \( \Longrightarrow \)
			\ref{eq:kvázimérték-jellemzése-03} \( \Longleftrightarrow \)
			\ref{eq:kvázimérték-jellemzése-04};
			
			\item ha \( \mu \) véges, akkor még 
			\ref{eq:kvázimérték-jellemzése-02} \( \Longleftrightarrow \)
			\ref{eq:kvázimérték-jellemzése-03} is fennáll.
			\marginnote{Tehát \( \mu(Z) < +\infty \ (Z \in \ring{G}) \).}
		\end{enumerate}
	\end{theorem}
	\begin{proof} \,\\[6pt]
		
		\fbox{\ref{eq:kvázimérték-jellemzése-01} \( \Longrightarrow \)
			  \ref{eq:kvázimérték-jellemzése-02}}
		%
		Tekintsük az \( A \) ``határhalmaznak'' az
		\[
			A = A_0 \cup (A_1 \setminus A_0) \cup (A_2 \setminus A_1) \cup \cdots
		\]
		páronként diszjunkt halmazokból álló felbontását.
		Mivel \( \mu \) szigma-additív, ezért
		%
		\begin{align*}
			\mu(A)
			&= \mu(A_0) + \mu(A_1 \setminus A_0) + \mu(A_2 \setminus A_1) + \cdots \\
			&= \lim_{n \to \infty} \Bigl( 
			\mu(A_0) + \mu(A_1 \setminus A_0) + \cdots + \mu(A_n \setminus A_{n-1})
			\Bigr)\\
			&= \lim_{n \to \infty} \mu \bigl( 
			A_0 \cup (A_1 \setminus A_0) \cup \cdots \cup (A_n \setminus A_{n-1})
			\bigr) \\
			&= \lim_{n \to \infty} \mu (A_n).
		\end{align*}
	\end{proof}
	
	\newpage
	
	\begin{theorem}{}{}
		Legyen \( \ring{H} \subseteq \powerset{X} \) félgyűrű, 
		továbbá \( \func{m}{\ring{H}}{[0, +\infty]} \) halmazfüggvény és
		%
		\begin{enumerate}
			\item az \( m \) additív és \( m(\emptyset) = 0 \);
			\item \( n \in \N, \ H_0, \dots, H_n \in \ring{H} \) páronként diszjunktak;
			\item \( s \in \N, \ Q_0, \dots, Q_s \in \ring{H} \) páronként diszjunktak.
		\end{enumerate}
		%
		Ekkor
		\[
			\bigcup_{k=0}^n \! H_k = \bigcup_{\ell=0}^s Q_\ell
			\qquad \Longrightarrow \qquad
			\sum_{k=0}^{n} m( H_k ) = \sum_{\ell=0}^{s} m( Q_\ell ).
		\]
	\end{theorem}
	\begin{proof}
		Mivel a metszetképzés disztributív az unióra, ezért
		\begin{alignat*}{5}
			H_k &= 
			H_k \cap \Biggl( \, \bigcup_{\ell=0}^s Q_\ell \Biggr) &&=
			\bigcup_{\ell=0}^s \bigl( H_k \cap Q_\ell \bigr) \qquad &&(k=0,\dots,n) \\[6pt]
			Q_\ell &= 
			Q_\ell \cap \Biggl( \, \bigcup_{k=0}^n H_k \Biggr) &&=
			\bigcup_{k=0}^n \bigl( Q_\ell \cap H_k \bigr) &&(\ell=0,\dots,s)
		\end{alignat*}
		páronként diszjunkt halmazrendszerek, ezért az \( m \) additivitása miatt
		\[
			\sum_{k=0}^n m( H_k ) =
		%	\sum_{k=0}^n m \Biggl( \bigcup_{\ell=0}^s \bigl( H_k \cap Q_\ell \bigr) \Biggr) =
			\sum_{k=0}^n \sum_{\ell=0}^s m \bigl( H_k \cap Q_\ell \bigr) =
			\sum_{\ell=0}^s \sum_{k=0}^n  m \bigl( Q_\ell \cap H_k \bigr) =
			\sum_{\ell=0}^s m( Q_\ell ).
		\]
	\end{proof}
	
	\newpage
	\begin{theorem}{}{}
		Legyen \( \ring{H} \subseteq \powerset{X} \) félgyűrű, 
		továbbá \( \func{m}{\ring{H}}{[0, +\infty]} \) additív és \( m(\emptyset) = 0 \).
		Definiáljuk az 
		\[
			\func{\mu}{\mathcal{G}(\ring{H})}{[0, +\infty]}, \qquad
			\mu(A) \coloneq \sum_{k=0}^{n} m( H_k ).
		\]
		Ekkor
		\begin{enumerate}
			\item \( \mu \) előmérték, valamint \( \restr{\mu}{\ring{H}} = m \);
			
			\item ha \( \lambda \) előmérték \( \mathcal{G}(\ring{H}) \)-n
			\marginnote{Tehát \( \func{\lambda}{\mathcal{G}(\ring{H})}{[0, +\infty]} \) alakú.}
			és \( \restr{\lambda}{\ring{H}} = m \), akkor \( \lambda = \mu \);
			
			\item ha \( m \) szigma-additív, akkor \( \mu \) kvázi-mérték.
		\end{enumerate}
	\end{theorem}
	\begin{proof}\,
		\begin{enumerate}
			\item Az állítás nyilvánvalóan igaz.
			
			\item Az lemma felhasználásával 
			
			\item Legyenek \( A_n \in \mathcal{G}(\ring{H}) \) páronként diszjunkt halmazok 
			\( (n \in \N) \) és
			\[
				A \coloneq \bigcup_{n=0}^{\infty} A_n \in \mathcal{G}(\ring{H}).
			\]
			Ekkor vannak olyan \( H_0, \dots, H_s \in \ring{H} \) páronként diszjunkt halmazok, hogy
			\[
				A = \bigcup_{k=0}^{s} H_k \in \mathcal{G}(\ring{H}).
			\]
			Ugyan ez elmondható az \( A_n \) halmazokra is, vagyis
			\[
				A_n = \bigcup_{j=0}^{p_n} H_{nj} \quad (n \in \N)
			\]
			valamilyen \( H_{n0}, \dots, H_{np_n} \in \ring{H} \) páronként diszjunkt halmazokkal.
			\begin{align*}
				\mu(A)
				 = \sum_{k=0}^{s} m( H_k )
				&= \sum_{k=0}^{s} \sum_{n=0}^{\infty} \sum_{j=0}^{p_n} m( H_{nj} \cap H_k ) \\
				&= \sum_{n=0}^{\infty} \sum_{j=0}^{p_n} \sum_{k=0}^{s} m( H_{nj} \cap H_k ) 
				%= \sum_{n=0}^{\infty} m( A_n )
				 = \sum_{n=0}^{\infty} \mu( A_n ).
			\end{align*}
		\end{enumerate}
	\end{proof}
	
\end{document}
