\documentclass[
%parspace, % Add vertical space between paragraphs
%noindent, % No indentation of first lines in each paragraph
%nohyp,	   % No hyphenation of words
%twoside,  % Double sided format
%draft,    % Quicker draft compilation without rendering images
%final,    % Set final to hide todos
]{elteikthesis}[2024/04/26]


% The minted package is also supported for source highlighting
% See elteikthesis_minted.tex for example
%\usepackage[newfloat]{minted}
\usepackage{enumitem}

% Document's metadata
\title{Mérték, integrál, valószínűség} % title
\subtitle{3. Vizsgatétel}

% The document
\begin{document}
	
	% Set document language
	\documentlang{hungarian}
	
	\section{Borel-lefedés}
	
	A \hyperref[lem:borel]{Borel-lemma} azt állítja, 
	hogy egy korlátos és zárt valós intervallum tetszőleges lefedéséből kiválasztható véges lefedés is.
	
	\begin{lemma}{Borel-féle lefedési lemma}{borel}
		Legyen \( [a, b] \subset \R \) egy korlátos és zárt intervallum, 
		vagyis \( a, b \in \R, \ a < b \).\\[6pt]
		Ha van olyan \( \Gamma \neq \emptyset \) indexhalmaz, hogy az
		\( I_\gamma \ (\gamma \in \Gamma) \) nyílt intervallumokra
		\[
			[a, b] \subseteq \bigcup_{\gamma \in \Gamma} I_\gamma
		\]
		teljesül, akkor kiválasztható olyan \( \Gamma_0 \subseteq \Gamma \) véges indexhalmaz, amellyel
		\[
			[a, b] \subseteq \bigcup_{\gamma \in \Gamma_0} I_\gamma.
		\]
	\end{lemma}	
	\begin{proof}
		Indirekt tegyük fel, hogy nincs ilyen \( \Gamma_0 \subseteq \Gamma \) véges halmaz.
		\\[6pt]
		%
		Felezzük meg az \( [a, b] \) intervallumot. 
		Ekkor valamelyik félintervallumot nem tudjuk lefedi véges sok \( I_\gamma \) felhasználásával, 
		mert ha mindkét rész lefedhető lenne, 
		akkor a lefedések egyesítése lefedné az \( [a, b] \) intervallumot.\\[6pt]
		%
		Hasonlóan, ezt a nem lefedhető félintervallumot újból megfelezve kapjuk, hogy
		legalább az egyik negyedintervallum nem fedhető le véges sok \( I_\gamma \)-val.\\[6pt]
		%
		Ezen konstruktív módon definiált \( (J_n) \) zárt intervallumsorozatra
		\[
			J_{n + 1} \subset J_n \subset [a, b], \qquad
			\abs{J_n} = \frac{b - a}{2^n} \qquad (n \in \N).
		\]
		Ekkor a \hyperref[th:cantor]{Cantor-tétel} alapján egyetlen 
		olyan \( \alpha \in [a, b] \) szám létezik, hogy
		\marginnote{
			\begin{theo*}[Cantor-tétel]\phantomsection\label{th:cantor}
				Amennyiben a
				\[
					J_{n + 1} \subseteq J_n, \quad
					\abs{J_n} \to 0
					\quad (n \to \infty)
				\]
				egy korlátos és zárt intervallumsorozat,
				akkor létezik olyan \( A \in \R \) szám, hogy
				\[
					\bigcap_{n = 0}^{\infty} J_n = \{ A \}.
				\]
			\end{theo*}
		}[-1.5\baselineskip]
		\[
			\bigcap_{n=0}^{\infty} J_n = \{ \alpha \}.
		\]
		Mivel az \( [a, b] \) intervallum lefedhető, ezért van olyan \( \delta \in \Gamma \) index, hogy
		\[
			\alpha \in I_\delta, \quad
		%	\qquad \text{és} \qquad 
			\alpha \in J_n \qquad (n \in \N).
		\]
		Viszont a \( (J_n) \) intervallumok hossza nullához tart, ezért
		\[
			\exists n \in \N \ \colon \, J_n \subset I_\delta.
		\]
		Ez pedig ellentmondás, hiszen a konstrukciója miatt \( J_n \) nem lefedhető 
		véges sok \( I_\gamma \) segítségével, 
		ennek ellenére az egyetlen \( I_\delta \) intervallum lefedi.
	\end{proof}
	
	\section{A Riesz-féle felépítés}
	
	\indent 
	Bizonyos szempontból a legsúlyosabb hiányossága a Riemann-integrálnak
	a határátmenettel szembeni ``nehézkes'' viselkedése. 
	Ezt a szempontot helyezte a középpontba Riesz Frigyes, 
	amikor a Lebesgue-féle gondolat egy új interpretálását fogalmazta meg. 
	Az alábbiakban röviden vázoljuk a Riesz-féle felépítés alapgondolatát.
	
	\newpage
	\begin{definition}{Lépcsősfüggvény}{}
		Legyen \( [a,b] \subseteq \R \) korlátos és zárt intervallum, az
		\( \func{f}{[a, b]}{\R} \) korlátos.\\[6pt]		
		Azt mondjuk, hogy az \( f \) egy \emph{lépcsősfüggvény}, ha van olyan
		\[
			a = x_0 < x_1 < \cdots < x_n = b
		\]
		felosztás és \( c_0, \dots, c_{n-1} \in \R \) konstansok, 
		hogy minden \( k = 0,\dots,n-1 \)-ra
		\[
			f(x) = c_k \qquad ( x_k < x < x_{k+1} ).
		\]
		Ekkor az előbbi \( f \) lépcsősfüggvény \emph{integrálja} legyen
		\[
			\int_a^b f \coloneq \sum_{k=0}^{n-1} c_k {\cdot} (x_{k + 1} - x_k) \in \R.
		\]
	\end{definition}
	
	\begin{notes}
		\item 
		Az osztópontokban felvett \( h(x_k) \) helyettesítési értékek tetszőlegesek lehetnek.
		
		\item
		Világos, hogy minden lépcsősfüggvény Riemann-integrálható és az integrál definíciója megegyezik a Riemann-integrál értékével.
	\end{notes}

	\noindent	
	A továbbiakban legyen a lépcsősfüggvények halmaza
	\[
		C_0 \coloneq
		\setc[\big]{ \func{f}{[a,b]}{\R} }{ f \text{ lépcsősfüggvény} }.
	\]
	
	\begin{lemma}{A-lemma}{riesz-a}
		Legyen \( (h_n) \) egy olyan \( C_0 \)-beli függvénysorozat, amelyre
		\begin{enumerate}[label=\roman*)]
			\item\label{eq:riesz-a-lemma-01}
			minden \( x \in [a, b] \) helyen és \( n \in \N \) indexre
			\( 0 \leq h_{n+1}(x) \leq h_n(x) \);
%			\item nemnegatív és monoton csökken, azaz minden \( n \in \N \) indexre
%			\[
%				0 \leq h_{n+1}(x) \leq h_n(x) \qquad \bigl( x \in [a, b] \bigr).
%			\]

			\item\label{eq:riesz-a-lemma-02}
			valamilyen nullamértékű \( \mathcal{N} \subset [a,b] \) halmazzal
			\[
				\lim_{n \to \infty} h_n(x) = 0 
				\qquad \bigl( x \in [a, b] \setminus \mathcal{N} \bigr).
			\]
		\end{enumerate}
		Ekkor létezik 
		\[
			\lim_{n \to \infty} \int_a^b h_n = 0.
		\]
	\end{lemma}
	\begin{proof}
		Mivel a \( (h_n) \) tagok osztópontjai legfeljebb megszámlálhatóan sokan vannak, 
		így ezek nullamértékű halmazt alkotnak.
		Egyesítsük ezeket a pontokat \( \mathcal{N} \)-el.
		Az így kapott nullamértékű halmazt a továbbiakban \( \mathcal{R} \) jelöli.
		Vagyis amennyiben az \( \varepsilon > 0 \) rögzített, 
		akkor létezik olyan \( (I_n) \) nyílt intervallumsorozat, hogy
		\[
			\mathcal{R} \subseteq \bigcup_{n=0}^{\infty} I_n
			\qquad \text{és} \qquad
			\sum_{n=0}^{\infty} \abs{I_n} < \varepsilon.
		\]		
		Ekkor \aref{eq:riesz-a-lemma-02} feltétel szerint 
		minden \( x \in [a,b] \setminus \mathcal{R} \) helyen
		\[
			\lim_{n \to \infty} h_n(x) = 0.
%			\qquad \Longrightarrow \qquad
%			\exists N_x \in \N \ \colon \,
%			h_n(x) < \varepsilon
%			\qquad (\N \ni n \geq N_x).
		\]
		A konvergencia definíciója alapján van olyan \( N_x \in \N \) küszöbindex, hogy
		\[
			h_n(x) < \varepsilon
			\qquad (N_x \leq n \in \N).
		\]
		Ugyanakkor az \( x \) nem osztópontja \( h_{N_x} \)-nek, 
		ezért van olyan \( J_x \subset [a, b] \) nyílt intervallum, 
		ahol a \( \restr{h_{N_x}}{J_x} \) függvény konstans.
		Továbbá \aref{eq:riesz-a-lemma-01} feltétel miatt
		%
		\begin{equation}\label{eq:riesz-a-lemma-03}
			h_n(t) \leq h_{N_x}(t) < \varepsilon \qquad (n > N_x, \ t \in J_x).
			\tag{\( * \)}
		\end{equation}
		%
		is feltehető. Világos, hogy ekkor
		\[
			[a, b] \subseteq
			\mathopen{\raisebox{-0.9ex}{$ \biggl( $}}\,
			\bigcup_{n \in \N} I_n
			\mathopen{\raisebox{-0.9ex}{$ \biggr) $}}
			\, \cup \,
			\mathopen{\raisebox{-0.9ex}{$ \biggl( $}}\,
			\bigcup_{x \in \mathcal{R}^c} J_x
			\mathopen{\raisebox{-0.9ex}{$ \biggr) $}}.
		\]
		ezért a \hyperref[lem:borel]{Borel-lemma} szerint vannak olyan
		\[
			A \subset \N, \quad
			B \subset [a, b] \setminus \mathcal{R}
		\]
		véges halmazok, amelyekkel
		\[
			[a, b] \subseteq
			\mathopen{\raisebox{-0.9ex}{$ \biggl( $}}\,
			\bigcup_{n \in A} I_n
			\mathopen{\raisebox{-0.9ex}{$ \biggr) $}}
			\, \cup \,
			\mathopen{\raisebox{-0.9ex}{$ \biggl( $}}\,
			\bigcup_{x \in B} J_x
			\mathopen{\raisebox{-0.9ex}{$ \biggr) $}}.
		\]
		Az előbbi véges lefedésében szereplő intervallumok \( [a, b] \)-beli végpontjai 
		(ha	szükséges, akkor az \( a, b \) pontok hozzátételével) egy
		\[
			a = z_0 < z_1 < \cdots < z_s = b
		\]
		felosztást határoznak meg valamilyen \( s \in \N \) mellett. Legyen
		\[
			\mathcal{I} \coloneq 
			\setc[\big]{ k = 0,\dots,s-1 }
			           { \exists n \in A \ \colon \, (z_i, z_{i+1}) \subseteq I_n }, \quad
			\mathcal{J} \coloneq
			\{ 0, \dots, s-1 \} \setminus \mathcal{I}.
		\]
		Végül legyenek
		\[
			N \coloneq \max \{ N_x \ \colon \, x \in B \}, \quad
			N < n \in \N, \quad
			h_n \leq C \in \R.
		\]
		Ekkor a soron következő becslés van érvényben
		\begin{align*}
			0 \leq \int_a^b h_n
			&= \sum_{i=0}^{s-1} \int_{z_i}^{z_{i+1}} h_n 
			 = \sum_{i \in \mathcal{I}} \int_{z_i}^{z_{i+1}} h_n 
			 + \sum_{j \in \mathcal{J}} \int_{z_j}^{z_{j+1}} h_n \\[3pt]
			&\leq C \cdot \sum_{i \in \mathcal{I}} (z_{i + 1} - z_i)
			+ \varepsilon \cdot \sum_{j \in \mathcal{J}} (z_{j + 1} - z_j) \\
			&< C \cdot \sum_{n=0}^{\infty} \abs{I_n}
			+ \varepsilon \cdot (b - a) \\
			&= \varepsilon \cdot (C + b - a).
		\end{align*}
		Mindez azt jelenti, hogy valóban létezik 
		a \( \lim \Bigl( \int_a^b h_n \Bigr) = 0 \) határérték.
	\end{proof}
	
	\begin{lemma}{B-lemma}{riesz-b}
		Legyen \( (h_n) \) egy olyan \( C_0 \)-beli függvénysorozat, amelyre
		\begin{enumerate}[label=\roman*)]
			\item
			minden \( x \in [a, b] \) helyen és \( n \in \N \) indexre
			\( h_n(x) \leq h_{n + 1}(x) \);
			
			\item
			az integrálok \( \Bigl( \int_a^b h_n \Bigr) \) sorozata korlátos.
		\end{enumerate}
		Ekkor van olyan nullamértékű \( \mathcal{M} \subset [a, b] \) halmaz, hogy
		\[
			\lim_{n \to \infty} h_n(x) < +\infty
			\qquad \bigl( x \in [a, b] \setminus \mathcal{M} \bigr).
		\]
	\end{lemma}
	
%	\begin{notes}
%		\item 
%		A Riemann-integrál monotonitása alapján világos, hogy az
%		\[
%			\biggl( \int_a^b h_n \biggr)
%		\]
%		integrálsorozat monoton növekedő, ezért a
%		\[
%			\sup_{n \in \N} \int_a^b h_n = \lim_{n \to \infty} \int_a^b h_n 
%		\]
%		
%		\item
%		Belátható, hogy amennyiben a \hyperref[lem:riesz-b]{B-lemma} feltételeinek
%		\[
%			\lim_{n \to \infty} h_n(x) =
%			\lim_{n \to \infty} \widetilde{h}_n(x)
%			\qquad \bigl( \text{m.m. } x \in [a, b] \bigr),
%		\]
%		akkor az
%		\[
%			\lim_{n \to \infty} \int_a^b h_n =
%			\lim_{n \to \infty} \int_a^b \widetilde{h}_n.
%		\]
%	\end{notes}

	\newpage
%	\noindent
%	Legyen \( C_1 \) azoknak a \( \func{h}{[a, b]}{\R} \) függvényeknek a halmaza, 
%	amelyek majdnem mindenhol előállíthatóak egy alkalmas
%	\hyperref[lem:riesz-b]{B-lemmabeli} \( (h_n) \) függvénysorozat \( \lim(h_n) \) határfüggvényeként.
	
	\begin{definition}{}{}
		Ha a \( (h_n) \) függvénysorozat eleget tesz a \hyperref[lem:riesz-b]{B-lemma} feltételeinek, akkor legyen
		\[
			C_1 \coloneq
			\setc[\Big]{ \func{h}{[a, b]}{\R} }
	    			   { h(x) = \lim_{n \to \infty} h_n(x) \ 
	    			   	 \bigl(\text{m.m. } x \in [a,b] \bigr) }.
		\]
		Továbbá egy ilyen \( h \in C_1 \) függvény \emph{integrálja} legyen
		\[
			\int_a^b h \coloneq \lim_{n \to \infty} \int_a^b h_n.
		\]
	\end{definition}
	\begin{notes}
		\item
		Az integrál értéke nem függ a \( h \)-t közelítő sorozat megválasztástól.
		
		\item 
		Világos, hogy \( C_0 \subseteq C_1 \) fennáll, 
		valamint az integrál értéke változatlan.
	\end{notes}
	
%	A \hyperref[lem:riesz-b]{B-lemma} alapján tekintsük azoknak 
%	
%	Legyen
%	\[
%		C_1 \coloneq
%		\setc[\big]{ \func{f}{[a, b]}{\R} }
%		           { f(x) \coloneq \lim_{n \to \infty} f_n(x) \ (\text{m.m. } x \in [a,b])}
%	\]
%	Belátható, hogy amennyiben \( (\widetilde{f}_n) \) egy \( C_0 \)-beli olyan függvénysorozat, ami
	
	\begin{definition}{}{}
		Legyen a \emph{Lebesgue-integrálható} függvények halmaza
		\[
			C_2 \coloneq
			\setc[\big]{ f \coloneq g - h }{ g, h \in C_1 },
		\]
		valamint az ilyen függvények \emph{Lebesgue-integrálja} legyen
		\[
			\int_a^b f \coloneq \int_a^b g - \int_a^b h.
		\]
	\end{definition}
	
	\begin{notes}
		\item
		A Lebesgue-integrál értéké független az \( f \in C_2 \) előállításától, azaz ha
		\[
			f = g - h = G - H
		\]
		fennáll valamilyen \( g,h \in C_1 \) illetve \( G,H \in C_1 \) függvényekre, akkor
		\[
			\int_a^b f =
			\int_a^b g - \int_a^b h =
			\int_a^b G - \int_a^b H.
		\]
		
		\item 
		Világos, hogy \( C_1 \subseteq C_2 \) fennáll, 
		valamint az integrál értéke változatlan.
		
		\item
		Nem minden Lebesgue-integrálható függvény Riemann-integrálható, hiszen
		\[
			\func{f}{[0, 1]}{\R}, \qquad
			f(x) \coloneq
			\left\{
			\begin{aligned}
				1, & \quad \text{ha } x \in \Q, \\
				0, & \quad \text{ha } x \notin \Q
			\end{aligned}
			\right.
		\]
		nem Riemann-integrálható.
		Ugyanakkor, 
		ha tekintjük a \( [0, 1] \)-beli racionális számoknak egy \( (r_n) \) sorozatát, 
		akkor az
		\[
		%	\func{f_n}{[0, 1]}{\R}, \qquad
			f_n(x) \coloneq
			\left\{
			\begin{aligned}
				1, & \quad \text{ha } x \in \{ r_0, \dots, r_n \}, \\
				0, & \quad \text{ha } x \notin \{ r_0, \dots, r_n \}
			\end{aligned}
			\right.
			\qquad \bigl( x \in [0, 1], \ n \in \N \bigr)
		\]
		függvénysorozat eleget tesz a \hyperref[lem:riesz-b]{B-lemma} feltételeinek és
		\( f = \lim(f_n) \), ezért
		\[
			\int_a^b f = \lim_{n \to \infty} \int_a^b f_n = 0.
		\]
	\end{notes}
	
\end{document}